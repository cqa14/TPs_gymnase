% Conclure le rapport
% Qualité des résultats
% Source d'erreur
% Voies d'amélioration
En conclusion, la réalisation de filtres "passe-haut" et "passe-bas" a été faite avec succès. Néanmoins, les détails plus fin du travail ont rencontré différents problèmes menant à des valeurs très imprécises. L'erreur peut autant venir de l'observateur faisant la mesure que des appareils et composants utilisés. Dans ces conditions, des remarques sur une éventuelle approximation du modèle physique utilisé sont peu pertinentes. On aurait pu améliorer nos résultats en faisant une deuxième série de mesure sur le filtre "passe-haut", ainsi qu'en mesurant au multimètre chaque composant et pas seulement le condensateur. De plus, traiter les données de l'oscilloscope par ordinateur aurait pu faire gagner assez de précision sur les intervalles de temps réduit pour sensiblement améliorer nos résultats. Au vu des hypothèses sur la relation entre la fréquence de coupure et les différents paramètres, il serait aussi intéressant de mener des expériences pour les vérifier. En somme, malgré certaines imprécisions, nous sommes satisfaits des expériences menées.