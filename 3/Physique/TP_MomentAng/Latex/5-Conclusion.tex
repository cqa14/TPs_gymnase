% Conclure le rapport
% Qualité des résultats
% Source d'erreur
% Voies d'amélioration
Les résultats sont très satisfaisants et confirment bien la théorie de rotation des solides. Nous avons considérés un des modèles les plus simples possibles, ce qui a mené à des erreurs tels que la non-prise en compte des forces de frottement, mais qui a permis quand même de prendre des mesures d'assez bonne qualité. L'essai de prise au ralenti à néanmoins fait qu'un partie des mesures étaient dégradées mais quand même exploitables et donnant des résultats au final cohérents. L'algorithme de tracking fut de plus bien choisi, comme en témoigne les bonnes courbes de mesures ayant put être obtenues sans intervention de notre part dans le processus de tracking. On pourrait améliorer l'expérience en prenant les frottements en compte ou en essayant de les éliminer, par exemple avec un meilleur dispositif de coussin d'air ou en essayant de faire par un système magnétique, voir même avec un supraconducteur pour enlever tout frottement. On pourrait aussi imaginer un tracking matériel plutôt que par image qui serait plus précise. Finalement, une méthode de variation des paramètres plus large pourrait offrir encore plus de données. Malgré cela, cette expérience est très bonne pour confirmer la validité du modèle.
