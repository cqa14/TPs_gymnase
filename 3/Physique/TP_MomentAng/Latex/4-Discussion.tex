% Discuter les résultats, sources d'incertitudes, ...
Les résultats obtenus sont très satisfaisants. Comme on le voit en table \ref{tab:recap}, le facteur de proportionnalité est constant au cours des mesures ce qui nous permet d'affirmer que le moment d'inertie de notre disque autour de son axe centrale est autour de $0.015 \ [kg \cdot m^2]$ avec une bonne confiance. De plus, comme le moment de force est selon notre approximation seulement dépend de la force exercée par le contrepoids, il va augmenter seulement grâce à la masse (car la force pesante d'une masse est proportionnelle à celle-ci) et la distance entre le point d'application de la force et le pivot, \textit{id est} le rayon du disque. Comme le moment de force est proportionnel à l'accélération angulaire selon la théorie est que il est lui même proportionnel à ces facteurs, l'accélération angulaire devrait l'être. C'est ce que montrent les graphiques en figure \ref{fig:prop}.\\ \\
Dans cette figure, nous observons que la proportionnalité au rayon fonctionne bien, comme on le voit dans la corrélation entre la courbe de tendance en vert et les points de résultats en bleus. Mais pour la masse fixe cela semble plus compliqué. Le problème ne vient néanmoins pas de la loi physique mais de la prise de mesure. Observons en effet les vidéos des figures \ref{fig:p70}, \ref{fig:m70} et \ref{fig:g70} qui sont utilisées pour vérifier cette partie de la loi. Les graphes les plus intéressants sont pour la vitesse. En effet, nous voyons une grande divergence aux extrêmes, mais quand la vitesse change de direction et donc que le disque ralenti et repart dans l'autre sens, cela est moindre. Ce problème se voit également sur l'accélération, où dans les autres mesures elle est régulière dans le bruit mais pas dans celles-ci. Cela s'explique néanmoins assez simplement. Afin d'essayer d'avoir des mesures plus précises, nous avons pris les vidéos via le mode ralenti d'un \textit{iPhone}, ce qui nous offrait en théorie une prise d'image à 120 $[fps]$, en théorie donc meilleure que les 60 $[fps]$ des autres vidéos. Mais lors du traitement de celles-ci, un autre problème fut soulevé. En effet, le software \textit{iOS} ne fait pas un rendu linéaire de cette vidéo juste en réduisant le nombre d'image par seconde à l'encodage. Il va faire un rendu non-linéaire à 30 $[fps]$, afin d'avoir un aspect esthétique de décélération au début de la vidéo jusqu'au ralenti puis accélération à nouveau à la fin. De fait, l'utilisation naïve de ces vidéos comme du 120 $[fps]$ sans corriger ces effets car cela est trop coûteux pour le bénéfice, ce qui explique les graphes étranges pour ces trois mesures. On voit que malgré cela, les résultats demeurent cohérents avec les autres.\\
Une autre explication peut être que le plus grand nombre d'image par seconde fait que la précision de la méthode de tracking, ou dans ce cas son imprécision, est plus marquante car les résultats demandent dans ces cas plus de précision pour quantifier la vitesse précisément. Cette hypothèse semble plus probable au vu du fait que les résultats sont cohérents au globale avec les autres ce qui laisse à penser que l'imprécision des mesures instantanées ne se répercutent pas globalement.\\ \\
Il y a finalement un dernier point majeur à discuter, qui est visible principalement sur les graphes de la vitesse, mais explique également le léger décalage entre la courbe de tendance parabolique et l'angle en fonction du temps. En effet, la vitesse semble en deux parties, une avant qu'elle soit nulle où elle va diminuer plus rapidement qu'elle accélère ensuite. En théorie, si la seule force exerçant un moment était la force pesante $F_p$, cela ne devrait être qu'une droite. La séparation en deux cas s'explique néanmoins si l'on inclus une force de frottement $F_f$ s'opposant au mouvement par définition. De fait, quant on lance le disque dans l'autre direction que celle où tire le contrepoids, $F_p$ et $F_f$ vont donc s'exercer les deux contre le mouvement et le moment de force va donc en réalité plus dépendre de $F_{tot} = F_p + F_f$ alors que quand le disque s'arrête et repart dans la direction du contrepoids, les deux forces seront opposée et la force totale sera plutôt $F_{tot} = F_p - F_f$, ce qui explique les deux régimes de vitesse et donc de position. Néanmoins, les traités comme un seul régime pour les courbes de tendances n'est pas dérangeant pour observer les phénomènes désirés.\\ \\
Un graphique peut encore interpeller, la figure \ref{fig:g70}. En effet, la courbe et la vitesse ont un signe opposé à toutes les autre. Cela s'explique simplement par un point de départ pour les angles différents qui fait qu'avec la méthodes de calcul il y a eu ce changement de signe. On a donc pris dans la suite des calculs  l'accélération en valeur absolue.
