%! Author = cqa
%! Date = 02.05.23

\documentclass[11pt]{article}
% Basic Packages for Encoding (Input AND Output) and Langauge Support
\usepackage[utf8]{inputenc}
\usepackage[T1]{fontenc}
\usepackage[french]{babel}

% Change Layout with a User-Friendly Interface
\usepackage[margin=1in]{geometry}

% Include Pictures with a User-Friendly Interface
\usepackage{graphicx}
\usepackage{lmodern}
\usepackage{float}

% Extended Math Support from the Famous 'American Mathematical Society'
\usepackage{amsmath}
\usepackage{amsfonts}
\usepackage{amssymb}

% For physics
\usepackage{physics}
\usepackage[f]{esvect}

% Just for Demonstration Purposes
\usepackage[math]{blindtext}

% For the chemistry
\usepackage{chemist}

% For use on computer
\usepackage{hyperref}

% For table color
\usepackage{colortbl}
\usepackage[table]{xcolor}

% Tableau verticale
\usepackage{rotating}

% Figure dans figure
\usepackage{subfig}

% Multicolumn
\usepackage{multirow}

% Preuves
\usepackage{amsthm}

% Footnote
\usepackage[bottom]{footmisc}

% PsTricks
\usepackage[usenames,dvipsnames]{pstricks}
\usepackage{pstricks-add}
\usepackage{epsfig}
\usepackage{pst-grad} % For gradients
\usepackage{pst-plot} % For axes
\usepackage[space]{grffile} % For spaces in paths
\usepackage{etoolbox} % For spaces in paths
\makeatletter % For spaces in paths
\patchcmd\Gread@eps{\@inputcheck#1 }{\@inputcheck"#1"\relax}{}{}
\makeatother

\usepackage{auto-pst-pdf}

% Graphiques
\usepackage{pgfplots}
\DeclareUnicodeCharacter{2212}{−}
\usepgfplotslibrary{groupplots,dateplot}
\usetikzlibrary{patterns,shapes.arrows}
\pgfplotsset{compat=newest}

%\usepgfplotslibrary{external}
%\tikzexternalize

% Titre
\usepackage[affil-it]{authblk}
\usepackage{textcomp}
\usepackage{wasysym}
\title{\textbf{TP Ondes stationnaires}}
\author{Camille Yerly, Romain Blondel}
\affil{2M8, Gymnase Auguste Piccard}

% Document
\begin{document}
    \maketitle

    \section{But}\label{sec:but}

    Le but de ce TP est de comprendre le fonctionnement d'une onde, à l'aide d'une corde vibrante, en mesurant
    les longueurs d'ondes et les fréquences pour les différents modes, afin de vérifier les formules de vitesse
    des ondes.

    \section{Introduction}\label{sec:introduction}

    Les ondes stationnaires furent découvertes au 19\up{ème} siècle par Franz Melde.
    Il développa une expérience semblable à celle que nous avons reproduit afin de les observer.
    Les ondes stationnaires se forment lorsque deux ondes de même fréquence se superposent.
    À l'œil humain, elles semblent figées dans leur milieu.
    Pourtant, l'utilisation d'un stroboscope réglé à la même fréquence qu'elles, permet d'observer le
    mouvement de celles-ci.
    Dans l'expérience que nous avons mené, ces ondes stationnaires sont observées sur une corde.
    Les ondes progressives produites au départ se réfléchissant aux extrémités de la corde.
    Ainsi, en réglant la fréquence de ces ondes, il est possible de trouver des ondes stationnaires.\\
    Dans le cas d'une corde, elles sont caractérisées par des points particuliers: des nœuds et des ventres.
    Les nœuds représentant une superposition nulle des ondes et les ventres une superposition minimale ou
    maximale.
    Le nombre de ventres nous donne le mode de vibration.\\
    Les ondes stationnaires se trouvent dans beaucoup de phénomènes vibratoires au quotidien.
    Elles sont omniprésentes dans la musique, de nombreux instruments de musiques acoustiques illustrent
    ce principe; le fonctionnement d'une corde de violon ou d'un tuyau d'orgue par exemple.\\
    Dans ce travail pratique, nous avons cherché à vérifier expérimentalement la formule de la célérité
    des ondes stationnaires basée sur les caractéristiques de la corde en la comparant à la formule basique
    avec la fréquence et la longueur d'onde.\\
    La vitesse des ondes se calcule avec la longueur d'ondes $\lambda$ $[m]$ et la fréquence $f$ $[Hz]$
    selon la formule:
    \[c=\lambda \cdot f\]
    Dans le cas des ondes stationnaires sur une corde, il est aussi possible d'utiliser une formule basée
    uniquement sur les caractéristiques de la corde: la tension de la corde $F$ $[N]$ et sa masse linéaire
    $[\mu]$, c'est-à-dire sa masse $[kg]$ par mètre $[m]$, tel que:
    \[c=\sqrt{\dfrac{F}{\mu}}\]

    La démonstration de cette formule est assez intéressante.
    Prenons un tronçon de corde infinitésimal $AB$ avec $A(x;y)$ et $B(x+\dd{x};y+\dd{y})$, et l'angle entre la tangente
    à la corde et l'horizontal en ce point au temps $t$ donné par $\alpha (x,t)$ (respectivement $\alpha (x+\dd{x},t)$).
    En négligeant la force pesante, on a uniquement la force de tension qui agit sur la corde, $\vv*{T}{A}$ en $A$
    et $\vv*{T}{B}$ en $B$.
    % \usepackage[usenames,dvipsnames]{pstricks}
% \usepackage{pstricks-add}
% \usepackage{epsfig}
% \usepackage{pst-grad} % For gradients
% \usepackage{pst-plot} % For axes
% \usepackage[space]{grffile} % For spaces in paths
% \usepackage{etoolbox} % For spaces in paths
% \makeatletter % For spaces in paths
% \patchcmd\Gread@eps{\@inputcheck#1 }{\@inputcheck"#1"\relax}{}{}
% \makeatother
% 
\begin{figure}[H]
\centering
\psscalebox{0.8 0.8} % Change this value to rescale the drawing.
{
\begin{pspicture}(0,-4.1775)(17.043,2.2875)
\definecolor{colour1}{rgb}{0.6,0.6,0.6}
\rput(0.012999496,-1.9475){\psplot[linecolor=black, linewidth=0.04, plotstyle=curve, plotpoints=50, xunit=0.01, yunit=0.5, polarplot=false]{1.0}{900.0}{x sin}}
\psframe[linecolor=black, linewidth=0.04, linestyle=dotted, dotsep=0.10583334cm, dimen=outer](7.4129996,-1.7475)(6.8529997,-2.3075)
\psframe[linecolor=black, linewidth=0.04, linestyle=dotted, dotsep=0.10583334cm, dimen=outer](18,2.2875)(10,-4.1525)
\psline[linecolor=black, linewidth=0.04, linestyle=dotted, dotsep=0.10583334cm](7.4129996,-1.7675)(10,2.2525)
\psline[linecolor=black, linewidth=0.04, linestyle=dotted, dotsep=0.10583334cm](7.4129996,-2.3075)(10,-4.1475)
\psline[linecolor=black, linewidth=0.04, arrowsize=0.05291667cm 2.0,arrowlength=1.4,arrowinset=0.0]{->}(11.313,-3.0075)(16.213,-3.0075)
\psline[linecolor=black, linewidth=0.04, arrowsize=0.05291667cm 2.0,arrowlength=1.4,arrowinset=0.0]{->}(11.613,-3.3475)(11.613,1.6725)
\rput[bl](11.332999,-3.3675){$0$}
\psbezier[linecolor=black, linewidth=0.04](11.973,-2.3075)(13.933,-1.5875)(14.712999,-1.0075)(15.532999,-0.1075)
\psline[linecolor=green, linewidth=0.04, linestyle=dashed, dash=0.17638889cm 0.10583334cm](12.952999,-1.9475)(12.973,-3.0075)
\psline[linecolor=green, linewidth=0.04, linestyle=dashed, dash=0.17638889cm 0.10583334cm](14.153,-1.9675)(11.613,-1.9875)
\psline[linecolor=green, linewidth=0.04, linestyle=dashed, dash=0.17638889cm 0.10583334cm](14.733,-0.8875)(14.712999,-2.9675)
\psline[linecolor=green, linewidth=0.04, linestyle=dashed, dash=0.17638889cm 0.10583334cm](16.373,-0.8675)(11.613,-0.9075)
\rput[bl](12.952999,-3.4){$x$}
\rput[bl](11.193,-2.1075){$y$}
\rput[bl](10.3,-1.1075){$y+\dd{y}$}
\rput[bl](14.613,-3.4){$x+\dd{x}$}
\psline[linecolor=red, linewidth=0.04, arrowsize=0.05291667cm 2.0,arrowlength=1.4,arrowinset=0.0]{->}(12.853,-1.9875)(11.693,-2.5475)
\psline[linecolor=red, linewidth=0.04, arrowsize=0.05291667cm 2.0,arrowlength=1.4,arrowinset=0.0]{->}(14.733,-0.8875)(15.973,0.0325)
\psline[linecolor=red, linewidth=0.04, linestyle=dashed, dash=0.17638889cm 0.10583334cm](12.853,-1.9875)(13.792999,-1.6075)
\psarc[linecolor=colour1, linewidth=0.04, dimen=outer](13.592999,-1.8675){0.2}{325.30484}{84.289406}
\psarc[linecolor=colour1, linewidth=0.04, dimen=outer](15.153,-0.7275){0.2}{325.30484}{84.289406}
\rput[bl](13.792999,-1.9675){$\alpha (x,t)$}
\rput[bl](15.353,-0.7675){$\alpha (x+\dd{x},t)$}
\rput[bl](11.853,-2.9){$\vv*{T}{A}$}
\rput[bl](15.792999,0.0725){$\vv*{T}{B}$}
\end{pspicture}
}
\caption{Onde sur une corde et forces considérées.}
\label{fig:onde-corde}
\end{figure}


    Connaissant la masse linéique de la corde $\mu$, on peut écrire la relation suivante:
    \[
        \vv*{T}{A} + \vv*{T}{B} = \mu \dd{x} \vec{a}
    \]
    où $\vec{a} = \mqty(\pdv[2]{x}{t} \\ \pdv[2]{y}{t}) = \mqty(0 \\ \pdv[2]{y}{t}) $ est l'accélération de la corde
    (nulle horizontalement, car il n'y a qu'un déplacement vertical).
    On peut décomposer l'équation en deux équations, une selon l'axe horizontal $Ox$ et une selon l'axe verticale $Oy$.
    \[
        \begin{cases}
            Ox \ : \ - T_A \cos \alpha (x,t) + T_B \cos \alpha (x+\dd{x},t) = 0 \\
            Oy \ : \ - T_A \sin \alpha (x,t) + T_B \sin \alpha (x+\dd{x},t) = \mu \dd{x} \pdv[2]{y}{t}
        \end{cases}
    \]
    En utilisant les approximations du premier ordre $\sin \alpha \approx \alpha$ et $\cos \alpha \approx 1$, si l'on
    considère la variation crée par l'onde et donc les angles comme faibles, on obtient:
    \[
        \begin{cases}
            Ox \ : \ - T_A + T_B = 0 \\
            Oy \ : \ - T_A \alpha (x,t) + T_B \alpha (x+\dd{x},t) = \mu \dd{x} \pdv[2]{y}{t}
        \end{cases}
        \Rightarrow
        \begin{cases}
            T_A = T_B = T \\
            - T \alpha (x,t) + T \alpha (x+\dd{x},t) = \mu \dd{x} \pdv[2]{y}{t}
        \end{cases}
    \]
    De plus, nous pouvons considérer que $\alpha = \tan \alpha = \pdv{y}{x}$, d'où:
    \[
        -T \pdv{y}{x} \left( x,t \right) + T \pdv{y}{x} \left( x+\dd{x},t \right) = \mu \dd{x} \pdv[2]{y}{t} \left( x,t \right)
    \]
    En notant que
    \[
        \pdv{y}{x} \left( x+\dd{x},t \right) - \pdv{y}{x} \left( x,t \right) =
        \pdv{x} \left( y(x+\dd{x},t) - y(x,t) \right) = \pdv{x} \left(  \pdv{y}{x} \left( x,t \right) \dd{x} \right) =
        \pdv[2]{y}{x} \dd{x}
    \]
    on obtient
    \[
        T \pdv[2]{y}{x} \dd{x} = \mu \dd{x} \pdv[2]{y}{t} \Leftrightarrow \pdv[2]{y}{x} - \frac{\mu}{T} \pdv[2]{y}{t} = 0
    \]
    qui est une équation d'onde, ou équation de d'Alembert, de forme
    \[
        \pdv[2]{y}{x} - \frac{1}{c^2} \pdv[2]{y}{t} = 0
    \]
    où $c = \sqrt{\frac{T}{\mu}}$ est la célérité de l'onde.
    On peut également noter que la solution générale de cette équation qui nous intéresse est de la forme
    \[
        y(x,t) = A \cos \left( \omega t + \phi \right) \cos \left( kx + \psi \right)
    \]
    et via les conditions de bord $y(0,t)=0$ et $y(L,t)=0$, où L est la longueur de la corde, on obtient que
    $\psi = \frac{\pi}{2}$, donc on peut transformer le cosinus en sinus, et $k = \frac{n \pi}{L}$, où
    $n \in \mathbb{N}$.
    Finalement, on obtient une équation de la forme
    \[
        y(x,t) = A \cos \left( \omega t + \phi \right) \sin \left( \frac{n \pi}{L} x \right)
    \]
    avec $A$ l'amplitude de l'onde, $\omega$ la pulsation, $\phi$ la phase et $n$ le mode de vibration.
    On voit également que cette équation peut être écrite de manière indépendante du temps, sous la forme
    \[
        y(x) = A \sin \left( \frac{n \pi}{L} x \right)
    \]
    qui est la forme générale d'une onde stationnaire, utile à la visualisation des modes de vibration.

    \section{Principe de mesure et description}\label{sec:principe-de-mesure-et-description}

    \subsection{Matériel}\label{subsec:materiel}

    \begin{itemize}
        \item Corde
        \item Vibreur
        \item Générateur de fonction
        \item Multimètre réglé pour mesurer la fréquence
        \item Coinceur et poulie pour la corde
        \item Plusieurs masses
        \item Serre-joint
        \item Balance
        \item Mètre
    \end{itemize}

    \subsection{Déroulement}\label{subsec:deroulement}

    \begin{enumerate}
        \item Tout d'abord, nous avons relevé la masse de la corde, à l'aide de la balance.
        \item Nous avons fait passer la corde dans le vibreur, puis avons réglé sa hauteur sur la table à
            l'aide du coinceur et de la poulie.
        \item Nous avons ensuite, fixé les masses au bout de la corde, du côté de la poulie, afin de la
            tendre.
            Ainsi, en changeant les masses, il était possible de changer la tension de la corde.
        \item Avec une masse de 110 [g], nous avons cherché les modes 3, 5 et 7 (Un contrepoids de 110 [g]
            était idéal afin de trouver les différents modes, car il permettait d'avoir une corde ni trop ni
            trop peu tendue).
            Les modes impairs étant plus facilement observables que les modes paires.
            Nous avons mesuré la longueur d'onde avec le mètre (mesure de la distance entre trois nœuds) et
            avons relevé la fréquence d'oscillation du vibreur avec le multimètre pour chacun des trois modes.
        \item Puis, nous avons fait varier le contrepoids entre 60 [g] et 360 [g] avec des écarts de 50 [g]
            en mesurant a longueur d'onde et la fréquence pour les modes 5 et 7 à chaque fois.
    \end{enumerate}

    \subsection{Schéma}\label{subsec:schema}
    
    % \usepackage[usenames,dvipsnames]{pstricks}
% \usepackage{pstricks-add}
% \usepackage{epsfig}
% \usepackage{pst-grad} % For gradients
% \usepackage{pst-plot} % For axes
% \usepackage[space]{grffile} % For spaces in paths
% \usepackage{etoolbox} % For spaces in paths
% \makeatletter % For spaces in paths
% \patchcmd\Gread@eps{\@inputcheck#1 }{\@inputcheck"#1"\relax}{}{}
% \makeatother
% 
\begin{figure}[H]
\centering
\psscalebox{1.0 1.0} % Change this value to rescale the drawing.
{
\begin{pspicture}(0,-7.18)(13.95626,-1.7)
\definecolor{colour0}{rgb}{0.6,0.6,0.6}
\definecolor{colour1}{rgb}{0.5019608,0.5019608,0.5019608}
\definecolor{colour2}{rgb}{0.4,0.4,0.4}
\definecolor{colour3}{rgb}{1.0,0.6,0.4}
\definecolor{colour4}{rgb}{1.0,0.9019608,0.4}
\definecolor{colour5}{rgb}{0.0,0.2,0.2}
\definecolor{colour6}{rgb}{0.3019608,0.5019608,0.5019608}
\definecolor{colour7}{rgb}{1.0,0.8,0.4}
\definecolor{colour8}{rgb}{1.0,0.8,0.6}
\definecolor{colour9}{rgb}{0.3019608,0.3019608,0.3019608}
\definecolor{colour10}{rgb}{0.2,0.2,0.2}
\definecolor{colour11}{rgb}{0.8,0.6,0.2}
\definecolor{colour12}{rgb}{0.7019608,0.7019608,0.7019608}
\psline[linecolor=black, linewidth=0.04](0.32,-4.82)(0.3,-4.66)(0.04,-4.66)(0.04,-4.96)(0.2,-4.98)(0.2,-6.14)(0.2,-6.14)
\psframe[linecolor=black, linewidth=0.04, fillstyle=solid,fillcolor=colour0, dimen=outer](8.44,-3.22)(5.76,-4.5)
\pscircle[linecolor=black, linewidth=0.04, fillstyle=solid,fillcolor=colour1, dimen=outer](6.16,-3.88){0.32}
\pscircle[linecolor=black, linewidth=0.04, fillstyle=solid,fillcolor=colour2, dimen=outer](6.88,-3.7){0.1}
\pscircle[linecolor=black, linewidth=0.04, fillstyle=solid,fillcolor=colour2, dimen=outer](6.62,-3.72){0.1}
\pscircle[linecolor=black, linewidth=0.04, fillstyle=solid,fillcolor=colour2, dimen=outer](6.64,-4.04){0.1}
\pscircle[linecolor=black, linewidth=0.04, fillstyle=solid,fillcolor=colour2, dimen=outer](7.66,-3.66){0.1}
\psframe[linecolor=black, linewidth=0.04, fillstyle=solid,fillcolor=colour3, dimen=outer](8.28,-3.56)(8.12,-3.82)
\psbezier[linecolor=yellow, linewidth=0.04](7.38,-4.07)(7.2,-3.67)(7.08,-3.8)(6.94,-4.04)
\psbezier[linecolor=yellow, linewidth=0.04](7.38,-4.31)(7.2,-3.91)(7.08,-4.04)(6.94,-4.28)
\psframe[linecolor=black, linewidth=0.04, fillstyle=solid,fillcolor=colour4, dimen=outer](7.64,-1.7)(6.8,-3.24)
\psframe[linecolor=black, linewidth=0.04, fillstyle=gradient, gradlines=2000, gradbegin=colour5, gradend=colour6, dimen=outer](7.5,-1.92)(6.98,-2.26)
\pscircle[linecolor=black, linewidth=0.04, fillstyle=solid,fillcolor=colour7, dimen=outer](7.22,-2.62){0.26}
\psbezier[linecolor=red, linewidth=0.04](7.0,-3.08)(7.0,-3.88)(7.8,-4.0)(8.04,-4.24)
\psbezier[linecolor=blue, linewidth=0.04](7.36,-3.08)(7.36,-3.88)(7.38,-3.64)(8.0,-4.02)
\psframe[linecolor=black, linewidth=0.04, fillstyle=solid,fillcolor=colour8, dimen=outer](13.24,-4.48)(0.78,-4.78)
\psframe[linecolor=black, linewidth=0.04, fillstyle=solid,fillcolor=colour9, dimen=outer](13.1,-4.96)(12.98,-5.22)
\psframe[linecolor=black, linewidth=0.04, fillstyle=solid,fillcolor=colour9, dimen=outer](13.18,-5.22)(12.9,-5.38)
\psframe[linecolor=black, linewidth=0.04, fillstyle=solid,fillcolor=colour0, dimen=outer](13.42,-3.88)(13.28,-5.4)
\psframe[linecolor=black, linewidth=0.04, fillstyle=solid,fillcolor=colour9, dimen=outer](1.02,-4.96)(0.9,-5.22)
\psframe[linecolor=black, linewidth=0.04, fillstyle=solid,fillcolor=colour9, dimen=outer](1.1,-5.22)(0.82,-5.38)
\psframe[linecolor=black, linewidth=0.04, fillstyle=solid,fillcolor=colour0, dimen=outer](0.72,-3.88)(0.58,-5.4)
\psbezier[linecolor=green, linewidth=0.04](13.94,-3.76)(12.534804,-3.2969368)(12.359995,-3.3630722)(11.38,-3.3)(10.400004,-3.2369277)(0.85788774,-3.3750381)(0.66,-3.4)(0.46211225,-3.424962)(0.30646285,-3.3615456)(0.24,-3.76)(0.17353714,-4.1584544)(0.24028333,-3.9438028)(0.18,-4.82)
\psframe[linecolor=black, linewidth=0.04, fillstyle=solid,fillcolor=colour2, dimen=outer](0.38,-5.6)(0.0,-6.3)
\psframe[linecolor=black, linewidth=0.04, fillstyle=solid,fillcolor=colour8, dimen=outer](2.82,-4.78)(2.36,-7.18)
\psframe[linecolor=black, linewidth=0.04, fillstyle=solid,fillcolor=colour8, dimen=outer](11.44,-4.74)(10.98,-7.14)
\pspolygon[linecolor=black, linewidth=0.04, fillstyle=solid,fillcolor=colour10](1.08,-4.5)(0.78,-4.5)(0.78,-4.78)(1.08,-4.76)(1.08,-4.94)(0.52,-4.96)(0.5,-4.36)(1.06,-4.34)
\pspolygon[linecolor=black, linewidth=0.04, fillstyle=solid,fillcolor=colour10](12.92,-4.5)(13.22,-4.5)(13.22,-4.78)(12.92,-4.76)(12.92,-4.94)(13.48,-4.96)(13.5,-4.36)(12.94,-4.34)
\pscircle[linecolor=black, linewidth=0.04, fillstyle=solid,fillcolor=colour11, dimen=outer](0.64,-3.76){0.4}
\psframe[linecolor=black, linewidth=0.04, fillstyle=solid,fillcolor=colour12, dimen=outer](11.38,-3.15)(11.22,-3.76)
\psbezier[linecolor=blue, linewidth=0.04](7.96,-3.98)(9.54,-3.7)(10.4,-3.66)(10.82,-4.12)
\psbezier[linecolor=red, linewidth=0.04](8.04,-4.22)(9.38,-4.12)(9.76,-4.52)(10.88,-4.2)
\psframe[linecolor=black, linewidth=0.04, fillstyle=solid,fillcolor=black, dimen=outer](11.8,-3.74)(10.76,-4.5)
\psframe[linecolor=black, linewidth=0.04, fillstyle=solid,fillcolor=colour12, dimen=outer](13.58,-3.24)(13.12,-3.94)
\psrotate(13.345, -3.435){4.034072}{\pscircle[linecolor=black, linewidth=0.04, fillstyle=solid,fillcolor=black, dimen=outer](13.345,-3.435){0.115}}
\pscircle[linecolor=black, linewidth=0.04, fillstyle=solid,fillcolor=black, dimen=outer](13.34,-3.72){0.12}
\rput[bl](0.08,-7.0){Contrepoids}
\rput[bl](1.54,-4.16){Poulie}
\rput[bl](3.5,-2.1){Multimètre}
\rput[bl](0,-2.88){Générateur de fonction et amplificateur}
\rput[bl](6.02,-5.8){Serre-joint}
\rput[bl](9.32,-2.7){Vibreur}
\rput[bl](12.44,-2.2){Coinceur}
\rput[bl](4.4,-4.2){Corde}
\psline[linecolor=black, linewidth=0.04, arrowsize=0.05291667cm 2.0,arrowlength=1.4,arrowinset=0.0]{->}(5.42,-1.98)(6.68,-2.3)
\psline[linecolor=black, linewidth=0.04, arrowsize=0.05291667cm 2.0,arrowlength=1.4,arrowinset=0.0]{->}(4.84,-2.84)(5.72,-4.0)
\psline[linecolor=black, linewidth=0.04, arrowsize=0.05291667cm 2.0,arrowlength=1.4,arrowinset=0.0]{->}(4.52,-3.84)(4.26,-3.4)
\psline[linecolor=black, linewidth=0.04, arrowsize=0.05291667cm 2.0,arrowlength=1.4,arrowinset=0.0]{->}(1.54,-4.0)(1.14,-3.76)
\psline[linecolor=black, linewidth=0.04, arrowsize=0.05291667cm 2.0,arrowlength=1.4,arrowinset=0.0]{->}(0.88,-6.62)(0.42,-6.2)
\psline[linecolor=black, linewidth=0.04, arrowsize=0.05291667cm 2.0,arrowlength=1.4,arrowinset=0.0]{->}(5.84,-5.6)(1.2,-5.02)
\psline[linecolor=black, linewidth=0.04, arrowsize=0.05291667cm 2.0,arrowlength=1.4,arrowinset=0.0]{->}(7.8,-5.62)(12.76,-4.96)
\psline[linecolor=black, linewidth=0.04, arrowsize=0.05291667cm 2.0,arrowlength=1.4,arrowinset=0.0]{->}(12.82,-2.3)(13.06,-3.14)
\psline[linecolor=black, linewidth=0.04, arrowsize=0.05291667cm 2.0,arrowlength=1.4,arrowinset=0.0]{->}(9.72,-2.78)(10.88,-3.64)
\end{pspicture}
}
\caption{Schéma de l'expérience}
\label{fig:schema}
\end{figure}



    \section{Résultats et calculs}\label{sec:resultats-et-calculs}

    Ci-suit les résultats obtenus lors de l'expérience, ainsi que les différents calculs effectués.
    Pour commencer, nous avons mesuré la masse de la corde, ainsi que sa longueur totale, afin de pouvoir calculer
    la masse linéaire $\mu$ de la corde: elle pèse $M \approx 4 \ [g]$ et mesure $l \approx 2.638 \ [m]$, d'où
    $\mu = \frac{m}{L} \approx 1.516 \cdot 10^{-3} \ \left[\frac{kg}{m}\right]$.\\
    De plus, la longueur de la corde qui vibre est de $L \approx 1.475 \ [m]$.
    De celle-ci, nous pouvons représenter les différents modes de vibration, en fonction de la position $x$ sur la
    corde, comme illustré sur la Figure~\ref{fig:waves}.
    Pour ce faire, nous avons utilisé la formule $y(x) = A \sin \left( \frac{n \pi}{L} x \right)$, où $n$ est le mode
    de vibration et $A$ l'amplitude de l'onde, que nous avons fixé à $1$.
    Sur le graphe sont représentés les modes $n=3,5,7$, car ce sont les modes les plus facilement observables (car
    les modes paires paraissent instable à l'œil).
    De plus on y a ajouté les nœuds, qui sont les points où l'amplitude de l'onde est nulle, et donc une référence
    fixe afin de mesurer la longueur d'onde $\lambda$, qui correspond à l'espace entre trois nœuds, car cela
    correspond à un ventre en haut et un autre en bas, soit un cycle complet.

    \begin{figure}[H]
        \centering
        \resizebox{1.0\textwidth}{!}{% This file was created with tikzplotlib v0.10.1.
\begin{tikzpicture}

\definecolor{darkgray178}{RGB}{178,178,178}
\definecolor{firebrick166640}{RGB}{166,6,40}
\definecolor{lightgray204}{RGB}{204,204,204}
\definecolor{silver188}{RGB}{188,188,188}
\definecolor{slategray122104166}{RGB}{122,104,166}
\definecolor{steelblue52138189}{RGB}{52,138,189}
\definecolor{whitesmoke238}{RGB}{238,238,238}

\begin{axis}[
axis background/.style={fill=whitesmoke238},
axis line style={silver188},
height=3cm,
legend cell align={left},
legend style={
  fill opacity=0.8,
  draw opacity=1,
  text opacity=1,
  at={(1.05,1)},
  anchor=north west,
  draw=lightgray204,
  fill=whitesmoke238
},
tick pos=left,
width=15cm,
x grid style={darkgray178},
xlabel={Longueur \(\displaystyle \left[m\right]\)},
xmajorgrids,
xmin=0, xmax=1.475,
xtick style={color=black},
y grid style={darkgray178},
ylabel={Amplitude},
ymajorgrids,
ymin=-1.09999864021146, ymax=1.09999864021146,
ytick style={color=black}
]
\addplot [draw=black, fill=black, mark=*, only marks]
table{%
x  y
0 0
0.491666666666667 0
0.983333333333333 0
0.295 0
0.59 0
0.885 0
1.18 0
1.475 0
0.210714285714286 0
0.421428571428571 0
0.632142857142857 0
0.842857142857143 0
1.05357142857143 0
1.26428571428571 0
1.475 0
};
\addlegendentry{noeuds}
\addplot [thick, steelblue52138189]
table {%
0 0
0.0280530452728271 0.17829167842865
0.0428178310394287 0.270191669464111
0.0561060905456543 0.350870013237
0.0679179430007935 0.420479536056519
0.0782532691955566 0.479437112808228
0.0871120691299438 0.528318166732788
0.0959709882736206 0.575506925582886
0.104829788208008 0.620851993560791
0.112212181091309 0.657126426696777
0.119594573974609 0.691938877105713
0.12697696685791 0.725211977958679
0.134359359741211 0.756871700286865
0.141741752624512 0.786847710609436
0.147647619247437 0.809571146965027
0.153553605079651 0.831141948699951
0.159459471702576 0.851529121398926
0.1653653383255 0.870703935623169
0.171271324157715 0.888638854026794
0.17717719078064 0.905308485031128
0.181606650352478 0.916965961456299
0.186035990715027 0.927889108657837
0.190465450286865 0.938068866729736
0.194894909858704 0.947497248649597
0.199324369430542 0.956166744232178
0.203753709793091 0.964070320129395
0.208183169364929 0.971201777458191
0.212612628936768 0.977555274963379
0.217042088508606 0.983125686645508
0.219995021820068 0.986402153968811
0.222947955131531 0.989327430725098
0.225900888442993 0.991900444030762
0.228853821754456 0.994120359420776
0.231806755065918 0.995986342430115
0.23475980758667 0.997497797012329
0.237712740898132 0.998654127120972
0.240665674209595 0.999454975128174
0.243618607521057 0.999899864196777
0.24657154083252 0.999988794326782
0.249524474143982 0.999721884727478
0.252477407455444 0.999099016189575
0.255430459976196 0.998120307922363
0.258383393287659 0.99678635597229
0.261336326599121 0.995097637176514
0.264289259910583 0.993054628372192
0.267242193222046 0.990658044815063
0.270195245742798 0.987908720970154
0.27314817905426 0.984807729721069
0.276101112365723 0.981356143951416
0.280530571937561 0.975524187088013
0.284960031509399 0.968910694122314
0.289389371871948 0.96152138710022
0.293818831443787 0.953361749649048
0.298248291015625 0.944438457489014
0.302677631378174 0.93475866317749
0.307107090950012 0.924330234527588
0.311536550521851 0.913161277770996
0.315966010093689 0.9012610912323
0.321871876716614 0.884272813796997
0.327777743339539 0.866025447845459
0.333683729171753 0.846544861793518
0.339589595794678 0.825859069824219
0.345495462417603 0.803997159004211
0.351401329040527 0.780990481376648
0.357307314872742 0.756871700286865
0.364689707756042 0.725211977958679
0.372072100639343 0.691938877105713
0.379454493522644 0.657126426696777
0.386836767196655 0.620851993560791
0.395695686340332 0.575506925582886
0.404554605484009 0.528318166732788
0.413413405418396 0.479437112808228
0.423748731613159 0.420479536056519
0.434084057807922 0.359688758850098
0.445895910263062 0.288309216499329
0.459184169769287 0.206065654754639
0.476901888847351 0.0942022800445557
0.528578519821167 -0.233674645423889
0.541866779327393 -0.315290927886963
0.553678750991821 -0.385949611663818
0.564013957977295 -0.445986747741699
0.574349403381348 -0.504079580307007
0.583208203315735 -0.552133560180664
0.592067003250122 -0.598419070243835
0.600925922393799 -0.64278769493103
0.6083083152771 -0.678194284439087
0.6156907081604 -0.712092280387878
0.623073101043701 -0.74440598487854
0.630455493927002 -0.775063753128052
0.636361360549927 -0.798351526260376
0.642267227172852 -0.820502519607544
0.648173093795776 -0.841485261917114
0.654079079627991 -0.86126983165741
0.659985065460205 -0.879827976226807
0.66589093208313 -0.897133350372314
0.671796798706055 -0.913161277770996
0.676226139068604 -0.924330234527588
0.680655717849731 -0.93475866317749
0.68508505821228 -0.944438457489014
0.689514517784119 -0.953361749649048
0.693943977355957 -0.96152138710022
0.698373317718506 -0.968910694122314
0.702802777290344 -0.975524187088013
0.707232236862183 -0.981356143951416
0.710185170173645 -0.984807729721069
0.713138103485107 -0.987908720970154
0.716091156005859 -0.990658044815063
0.719043970108032 -0.993054628372192
0.721997022628784 -0.995097637176514
0.724949955940247 -0.99678635597229
0.727902889251709 -0.998120307922363
0.730855941772461 -0.999099016189575
0.733808755874634 -0.999721884727478
0.736761808395386 -0.999988794326782
0.739714741706848 -0.999899864196777
0.742667675018311 -0.999454975128174
0.745620608329773 -0.998654127120972
0.748573541641235 -0.997497797012329
0.751526594161987 -0.995986342430115
0.75447940826416 -0.994120359420776
0.757432460784912 -0.991900444030762
0.760385394096375 -0.989327430725098
0.763338327407837 -0.986402153968811
0.766291379928589 -0.983125686645508
0.769244194030762 -0.979499340057373
0.7736736536026 -0.973406314849854
0.778103113174438 -0.966533660888672
0.782532453536987 -0.958886623382568
0.786962032318115 -0.950471758842468
0.791391372680664 -0.941295504570007
0.795820832252502 -0.931365251541138
0.800250291824341 -0.920689105987549
0.80467963218689 -0.90927529335022
0.809109091758728 -0.897133350372314
0.815015077590942 -0.879827976226807
0.820920944213867 -0.86126983165741
0.826826810836792 -0.841485261917114
0.832732677459717 -0.820502519607544
0.838638663291931 -0.798351526260376
0.844544529914856 -0.775063753128052
0.851926922798157 -0.74440598487854
0.859309315681458 -0.712092280387878
0.866691708564758 -0.678194284439087
0.874074101448059 -0.64278769493103
0.88145637512207 -0.605950832366943
0.890315294265747 -0.559974789619446
0.899174213409424 -0.512204885482788
0.908033013343811 -0.462794303894043
0.918368339538574 -0.403286337852478
0.930180191993713 -0.333139777183533
0.943468451499939 -0.251978039741516
0.958233237266541 -0.159694910049438
0.980380296707153 -0.0188672542572021
1.01138639450073 0.17829167842865
1.02615118026733 0.270191669464111
1.03943943977356 0.350870013237
1.0512512922287 0.420479536056519
1.06158661842346 0.479437112808228
1.07044541835785 0.528318166732788
1.07930433750153 0.575506925582886
1.08816313743591 0.620851993560791
1.09554553031921 0.657126426696777
1.10292792320251 0.691938877105713
1.11031031608582 0.725211977958679
1.11769270896912 0.756871700286865
1.12507510185242 0.786847710609436
1.13098096847534 0.809571146965027
1.13688683509827 0.831141948699951
1.14279282093048 0.851529121398926
1.14869868755341 0.870703935623169
1.15460455417633 0.888638854026794
1.16051054000854 0.905308485031128
1.16493988037109 0.916965961456299
1.16936933994293 0.927889108657837
1.17379879951477 0.938068866729736
1.17822825908661 0.947497248649597
1.18265759944916 0.956166744232178
1.187087059021 0.964070320129395
1.19151651859283 0.971201777458191
1.19594597816467 0.977555274963379
1.20037531852722 0.983125686645508
1.20332837104797 0.986402153968811
1.20628130435944 0.989327430725098
1.2092342376709 0.991900444030762
1.21218717098236 0.994120359420776
1.21514010429382 0.995986342430115
1.21809303760529 0.997497797012329
1.22104609012604 0.998654127120972
1.2239990234375 0.999454975128174
1.22695195674896 0.999899864196777
1.22990489006042 0.999988794326782
1.23285782337189 0.999721884727478
1.23581075668335 0.999099016189575
1.2387638092041 0.998120307922363
1.24171674251556 0.99678635597229
1.24466967582703 0.995097637176514
1.24762260913849 0.993054628372192
1.25057554244995 0.990658044815063
1.25352847576141 0.987908720970154
1.25648152828217 0.984807729721069
1.25943446159363 0.981356143951416
1.26386392116547 0.975524187088013
1.26829326152802 0.968910694122314
1.27272272109985 0.96152138710022
1.27715218067169 0.953361749649048
1.28158164024353 0.944438457489014
1.28601098060608 0.93475866317749
1.29044044017792 0.924330234527588
1.29486989974976 0.913161277770996
1.2992992401123 0.9012610912323
1.30520522594452 0.884272813796997
1.31111109256744 0.866025447845459
1.31701695919037 0.846544861793518
1.32292294502258 0.825859069824219
1.32882881164551 0.803997159004211
1.33473467826843 0.780990481376648
1.34064066410065 0.756871700286865
1.34802305698395 0.725211977958679
1.35540544986725 0.691938877105713
1.36278784275055 0.657126426696777
1.37017011642456 0.620851993560791
1.37902903556824 0.575506925582886
1.38788783550262 0.528318166732788
1.3967467546463 0.479437112808228
1.40708208084106 0.420479536056519
1.41741740703583 0.359688758850098
1.42922925949097 0.288309216499329
1.44399404525757 0.196824908256531
1.46171176433563 0.084805965423584
1.47500002384186 0
};
\addlegendentry{$n=3$}
\addplot [thick, firebrick166640]
table {%
0 0
0.0191942453384399 0.202987432479858
0.0295295715332031 0.309316039085388
0.0383883714675903 0.397523045539856
0.0472472906112671 0.482194542884827
0.0546296834945679 0.549509048461914
0.0620119571685791 0.613428711891174
0.0679179430007935 0.661854267120361
0.0738238096237183 0.707662463188171
0.0797296762466431 0.750672340393066
0.0856356620788574 0.790713667869568
0.0900650024414062 0.818700790405273
0.0944944620132446 0.844866752624512
0.098923921585083 0.869153022766113
0.103353381156921 0.891505718231201
0.10778284072876 0.911875009536743
0.112212181091309 0.930215716362
0.115165114402771 0.941295504570007
0.118118166923523 0.951444387435913
0.121071100234985 0.960652589797974
0.124024033546448 0.968910694122314
0.12697696685791 0.976210832595825
0.129929900169373 0.982545614242554
0.132882833480835 0.987908720970154
0.135835886001587 0.992295026779175
0.138788819313049 0.995699882507324
0.140265226364136 0.997033357620239
0.141741752624512 0.998120307922363
0.143218278884888 0.998960494995117
0.144694685935974 0.999553799629211
0.14617121219635 0.999899864196777
0.147647619247437 0.999998807907104
0.149124145507812 0.999850511550903
0.150600552558899 0.999454975128174
0.152077078819275 0.998812198638916
0.153553605079651 0.997922658920288
0.155030012130737 0.99678635597229
0.156506538391113 0.99540376663208
0.1579829454422 0.993774890899658
0.160935878753662 0.98978066444397
0.163888931274414 0.984807729721069
0.166841864585876 0.978860974311829
0.169794797897339 0.971946239471436
0.172747731208801 0.964070320129395
0.175700664520264 0.955241203308105
0.178653717041016 0.945467352867126
0.181606650352478 0.93475866317749
0.18455958366394 0.923125624656677
0.188988924026489 0.903968334197998
0.193418383598328 0.882799863815308
0.197847843170166 0.859667539596558
0.202277302742004 0.834622740745544
0.206706762313843 0.807721138000488
0.211136102676392 0.779022574424744
0.21556556224823 0.74859094619751
0.221471428871155 0.705437064170837
0.227377414703369 0.659493565559387
0.233283281326294 0.610942125320435
0.239189147949219 0.559974789619446
0.24657154083252 0.493176102638245
0.25395393371582 0.423330664634705
0.262812852859497 0.336103200912476
0.27314817905426 0.230615854263306
0.286436438560486 0.0910710096359253
0.318918943405151 -0.251978039741516
0.329254269599915 -0.356752634048462
0.338113069534302 -0.443169832229614
0.345495462417603 -0.512204885482788
0.352877855300903 -0.578075766563416
0.358783721923828 -0.628220081329346
0.364689707756042 -0.675879955291748
0.370595574378967 -0.720867156982422
0.376501560211182 -0.763003826141357
0.38093090057373 -0.792634963989258
0.385360360145569 -0.820502519607544
0.389789819717407 -0.846544861793518
0.394219160079956 -0.870703935623169
0.398648619651794 -0.89292585849762
0.403078079223633 -0.913161277770996
0.407507538795471 -0.931365251541138
0.410460472106934 -0.942352533340454
0.413413405418396 -0.952407836914062
0.416366338729858 -0.96152138710022
0.41931939125061 -0.96968400478363
0.422272205352783 -0.976887941360474
0.425225257873535 -0.983125686645508
0.428178191184998 -0.988391399383545
0.43113112449646 -0.992679715156555
0.434084057807922 -0.995986342430115
0.435560584068298 -0.997270584106445
0.437036991119385 -0.998308181762695
0.438513517379761 -0.999099016189575
0.439990043640137 -0.999642848968506
0.441466450691223 -0.999939441680908
0.442942976951599 -0.999988794326782
0.444419384002686 -0.999791145324707
0.445895910263062 -0.999346137046814
0.447372436523438 -0.998654127120972
0.448848843574524 -0.99771523475647
0.4503253698349 -0.996529579162598
0.451801776885986 -0.995097637176514
0.454754710197449 -0.991496086120605
0.457707643508911 -0.986914157867432
0.460660696029663 -0.981356143951416
0.463613629341125 -0.974827766418457
0.466566562652588 -0.96733546257019
0.46951949596405 -0.958886623382568
0.472472429275513 -0.949489593505859
0.475425481796265 -0.939153671264648
0.478378415107727 -0.927889108657837
0.482807874679565 -0.90927529335022
0.487237215042114 -0.888638854026794
0.491666674613953 -0.866025447845459
0.496096134185791 -0.841485261917114
0.50052547454834 -0.815073132514954
0.504954934120178 -0.786847710609436
0.509384393692017 -0.756871700286865
0.515290260314941 -0.714296579360962
0.521196126937866 -0.668896913528442
0.527102112770081 -0.620851993560791
0.533007979393005 -0.570352077484131
0.540390372276306 -0.504079580307007
0.547772765159607 -0.434693098068237
0.556631565093994 -0.347923517227173
0.566967010498047 -0.242837190628052
0.580255270004272 -0.103590130805969
0.614214181900024 0.255020141601562
0.624549627304077 0.359688758850098
0.633408427238464 0.445986747741699
0.640790820121765 0.51490330696106
0.648173093795776 0.580638885498047
0.654079079627991 0.630663633346558
0.659985065460205 0.678194284439087
0.66589093208313 0.723043203353882
0.671796798706055 0.765032768249512
0.676226139068604 0.794548273086548
0.680655717849731 0.822296142578125
0.68508505821228 0.848214626312256
0.689514517784119 0.872246265411377
0.693943977355957 0.894337177276611
0.698373317718506 0.914438605308533
0.702802777290344 0.93250560760498
0.705755710601807 0.943400144577026
0.708708763122559 0.953361749649048
0.711661577224731 0.962380528450012
0.714614629745483 0.970447778701782
0.717567563056946 0.977555274963379
0.720520496368408 0.983696103096008
0.72347354888916 0.988864302635193
0.726426362991333 0.993054628372192
0.729379415512085 0.996262907981873
0.730855941772461 0.997497797012329
0.732332348823547 0.998486042022705
0.733808755874634 0.999227523803711
0.73528528213501 0.999721884727478
0.736761808395386 0.999969124794006
0.738238215446472 0.999969124794006
0.739714741706848 0.999721884727478
0.741191148757935 0.999227523803711
0.742667675018311 0.998486042022705
0.744144201278687 0.997497797012329
0.745620608329773 0.996262907981873
0.747097015380859 0.994781732559204
0.750050067901611 0.991081953048706
0.753003001213074 0.986402153968811
0.755955934524536 0.980746984481812
0.758908987045288 0.974121809005737
0.761861801147461 0.966533660888672
0.764814853668213 0.957989454269409
0.767767786979675 0.948498129844666
0.770720720291138 0.938068866729736
0.7736736536026 0.926711916923523
0.778103113174438 0.907962083816528
0.782532453536987 0.887192249298096
0.786962032318115 0.86444878578186
0.791391372680664 0.839782118797302
0.795820832252502 0.813247203826904
0.800250291824341 0.784903049468994
0.80467963218689 0.754812717437744
0.810585498809814 0.712092280387878
0.816491484642029 0.666555881500244
0.822397470474243 0.618383646011353
0.828303337097168 0.567766070365906
0.835685729980469 0.501361131668091
0.84306812286377 0.431858897209167
0.851926922798157 0.344973564147949
0.86226224899292 0.239785432815552
0.875550508499146 0.100461840629578
0.909509420394897 -0.258059501647949
0.91984486579895 -0.362621188163757
0.928703784942627 -0.448799133300781
0.936086058616638 -0.517596483230591
0.943468451499939 -0.583196401596069
0.949374437332153 -0.633100986480713
0.955280303955078 -0.680501937866211
0.961186170578003 -0.725211977958679
0.967092037200928 -0.767054200172424
0.971521496772766 -0.796453833580017
0.975950956344604 -0.824081659317017
0.980380296707153 -0.849876165390015
0.984809875488281 -0.873779892921448
0.98923921585083 -0.895739793777466
0.993668675422668 -0.915706872940063
0.998098134994507 -0.933636784553528
1.00105106830597 -0.944438457489014
1.00400400161743 -0.954306125640869
1.00695693492889 -0.963230133056641
1.00990986824036 -0.971201777458191
1.01286292076111 -0.978212952613831
1.01581585407257 -0.984256744384766
1.01876878738403 -0.989327430725098
1.0217217206955 -0.993419647216797
1.02467465400696 -0.996529579162598
1.02615118026733 -0.99771523475647
1.02762758731842 -0.998654127120972
1.0291041135788 -0.999346137046814
1.03058063983917 -0.999791145324707
1.03205704689026 -0.999988794326782
1.03353357315063 -0.999939441680908
1.03500998020172 -0.999642848968506
1.0364865064621 -0.999099016189575
1.03796291351318 -0.998308181762695
1.03943943977356 -0.997270584106445
1.04091596603394 -0.995986342430115
1.04239237308502 -0.994455933570862
1.04534530639648 -0.990658044815063
1.04829823970795 -0.985880374908447
1.0512512922287 -0.980127930641174
1.05420422554016 -0.973406314849854
1.05715715885162 -0.96572208404541
1.06011009216309 -0.957082867622375
1.06306302547455 -0.947497248649597
1.06601595878601 -0.936974763870239
1.06896901130676 -0.925525665283203
1.07339835166931 -0.906639814376831
1.07782781124115 -0.88573694229126
1.08225727081299 -0.862863540649414
1.08668673038483 -0.838070631027222
1.09111607074738 -0.811413168907166
1.09554553031921 -0.782950639724731
1.09997498989105 -0.752746224403381
1.10588085651398 -0.709880828857422
1.11178684234619 -0.66420841217041
1.11769270896912 -0.615909337997437
1.12359857559204 -0.565174579620361
1.13098096847534 -0.498637676239014
1.13836336135864 -0.429020404815674
1.14722228050232 -0.342020153999329
1.15755760669708 -0.236731171607971
1.17084586620331 -0.0973324775695801
1.20332837104797 0.245886564254761
1.21366369724274 0.350870013237
1.22252249717712 0.437523007392883
1.22990489006042 0.506793022155762
1.23728728294373 0.572932243347168
1.24319314956665 0.623314142227173
1.24909913539886 0.671231269836426
1.25500500202179 0.716493844985962
1.26091086864471 0.758923292160034
1.26534032821655 0.788784503936768
1.26976978778839 0.816890954971313
1.27419924736023 0.843180179595947
1.27862858772278 0.86759352684021
1.28305804729462 0.890076637268066
1.28748750686646 0.910579681396484
1.29191696643829 0.929057002067566
1.29486989974976 0.940229177474976
1.29782283306122 0.950471758842468
1.30077576637268 0.959774374961853
1.30372869968414 0.968127965927124
1.30668163299561 0.975524187088013
1.30963468551636 0.981955766677856
1.31258761882782 0.98741626739502
1.31554055213928 0.991900444030762
1.31849348545074 0.99540376663208
1.31997001171112 0.99678635597229
1.32144641876221 0.997922658920288
1.32292294502258 0.998812198638916
1.32439935207367 0.999454975128174
1.32587587833405 0.999850511550903
1.32735240459442 0.999998807907104
1.32882881164551 0.999899864196777
1.33030533790588 0.999553799629211
1.33178174495697 0.998960494995117
1.33325827121735 0.998120307922363
1.33473467826843 0.997033357620239
1.33621120452881 0.995699882507324
1.33768773078918 0.994120359420776
1.34064066410065 0.990224242210388
1.34359359741211 0.98534893989563
1.34654653072357 0.979499340057373
1.34949946403503 0.972681045532227
1.3524523973465 0.964900970458984
1.35540544986725 0.956166744232178
1.35835838317871 0.946486949920654
1.36131131649017 0.935871362686157
1.36426424980164 0.924330234527588
1.36869370937347 0.905308485031128
1.37312316894531 0.884272813796997
1.37755250930786 0.86126983165741
1.3819819688797 0.836350798606873
1.38641142845154 0.809571146965027
1.39084088802338 0.780990481376648
1.39527022838593 0.750672340393066
1.40117621421814 0.707662463188171
1.40708208084106 0.661854267120361
1.41298794746399 0.613428711891174
1.4188939332962 0.562577486038208
1.4262763261795 0.495909333229065
1.43365859985352 0.426177620887756
1.44251751899719 0.339063405990601
1.45285284519196 0.233674645423889
1.46614110469818 0.0942022800445557
1.47500002384186 -0
};
\addlegendentry{$n=5$}
\addplot [thick, slategray122104166]
table {%
0 0
0.016241192817688 0.239785432815552
0.0251001119613647 0.36555004119873
0.0324825048446655 0.465579748153687
0.0383883714675903 0.541602492332458
0.0442942380905151 0.613428711891174
0.0502002239227295 0.680501937866211
0.0546296834945679 0.727373600006104
0.0590590238571167 0.771074295043945
0.0634884834289551 0.811413168907166
0.0679179430007935 0.848214626312256
0.0723474025726318 0.881318330764771
0.0753003358840942 0.9012610912323
0.0782532691955566 0.919457197189331
0.081206202507019 0.935871362686157
0.0841591358184814 0.950471758842468
0.0871120691299438 0.963230133056641
0.0900650024414062 0.974121809005737
0.0930180549621582 0.983125686645508
0.0944944620132446 0.986914157867432
0.0959709882736206 0.990224242210388
0.097447395324707 0.993054628372192
0.098923921585083 0.99540376663208
0.100400447845459 0.997270584106445
0.101876854896545 0.998654127120972
0.103353381156921 0.999553799629211
0.104829788208008 0.999969124794006
0.106306314468384 0.999899864196777
0.10778284072876 0.999346137046814
0.109259247779846 0.998308181762695
0.110735774040222 0.99678635597229
0.112212181091309 0.994781732559204
0.113688707351685 0.992295026779175
0.115165114402771 0.989327430725098
0.116641640663147 0.985880374908447
0.118118166923523 0.981955766677856
0.121071100234985 0.972681045532227
0.124024033546448 0.96152138710022
0.12697696685791 0.948498129844666
0.129929900169373 0.933636784553528
0.132882833480835 0.916965961456299
0.135835886001587 0.898518085479736
0.138788819313049 0.878329038619995
0.141741752624512 0.856437683105469
0.14617121219635 0.820502519607544
0.150600552558899 0.780990481376648
0.155030012130737 0.738073587417603
0.159459471702576 0.691938877105713
0.163888931274414 0.64278769493103
0.169794797897339 0.572932243347168
0.175700664520264 0.498637676239014
0.181606650352478 0.420479536056519
0.188988924026489 0.318273782730103
0.197847843170166 0.190654635429382
0.211136102676392 -0.00628948211669922
0.225900888442993 -0.224491357803345
0.23475980758667 -0.350870013237
0.242142200469971 -0.451607227325439
0.248048067092896 -0.528318166732788
0.25395393371582 -0.600935697555542
0.259859800338745 -0.668896913528442
0.264289259910583 -0.716493844985962
0.268718719482422 -0.760967254638672
0.27314817905426 -0.802123188972473
0.277577638626099 -0.839782118797302
0.282006978988647 -0.873779892921448
0.284960031509399 -0.894337177276611
0.287912845611572 -0.913161277770996
0.290865898132324 -0.930215716362
0.293818831443787 -0.945467352867126
0.296771764755249 -0.958886623382568
0.299724698066711 -0.970447778701782
0.302677631378174 -0.980127930641174
0.305630683898926 -0.987908720970154
0.307107090950012 -0.991081953048706
0.308583617210388 -0.993774890899658
0.310060024261475 -0.995986342430115
0.311536550521851 -0.99771523475647
0.313013076782227 -0.998960494995117
0.314489483833313 -0.999721884727478
0.315966010093689 -0.999998807907104
0.317442417144775 -0.999791145324707
0.318918943405151 -0.999099016189575
0.320395469665527 -0.997922658920288
0.321871876716614 -0.996262907981873
0.3233482837677 -0.994120359420776
0.324824810028076 -0.991496086120605
0.326301336288452 -0.988391399383545
0.327777743339539 -0.984807729721069
0.330730676651001 -0.976210832595825
0.333683729171753 -0.96572208404541
0.336636662483215 -0.953361749649048
0.339589595794678 -0.939153671264648
0.34254252910614 -0.923125624656677
0.345495462417603 -0.905308485031128
0.348448514938354 -0.88573694229126
0.351401329040527 -0.86444878578186
0.355830788612366 -0.829389095306396
0.360260248184204 -0.790713667869568
0.364689707756042 -0.74859094619751
0.369119167327881 -0.70320463180542
0.37354850769043 -0.654752731323242
0.379454493522644 -0.585748076438904
0.385360360145569 -0.512204885482788
0.391266226768494 -0.434693098068237
0.398648619651794 -0.333139777183533
0.407507538795471 -0.206065654754639
0.420795798301697 -0.00943410396575928
0.437036991119385 0.230615854263306
0.445895910263062 0.356752634048462
0.453278303146362 0.457209825515747
0.459184169769287 0.533647775650024
0.465090036392212 0.605950832366943
0.470996022224426 0.673558950424194
0.475425481796265 0.720867156982422
0.479854822158813 0.765032768249512
0.484284281730652 0.80586314201355
0.48871374130249 0.843180179595947
0.493143081665039 0.876821279525757
0.496096134185791 0.897133350372314
0.499049067497253 0.915706872940063
0.502002000808716 0.93250560760498
0.504954934120178 0.947497248649597
0.507907867431641 0.960652589797974
0.510860919952393 0.971946239471436
0.513813734054565 0.981356143951416
0.515290260314941 0.98534893989563
0.516766786575317 0.988864302635193
0.518243312835693 0.991900444030762
0.51971971988678 0.994455933570862
0.521196126937866 0.996529579162598
0.522672653198242 0.998120307922363
0.524149179458618 0.999227523803711
0.525625705718994 0.999850511550903
0.527102112770081 0.999988794326782
0.528578519821167 0.999642848968506
0.530055046081543 0.998812198638916
0.531531572341919 0.997497797012329
0.533007979393005 0.995699882507324
0.534484505653381 0.993419647216797
0.535960912704468 0.990658044815063
0.537437438964844 0.98741626739502
0.53891396522522 0.983696103096008
0.541866779327393 0.974827766418457
0.544819831848145 0.964070320129395
0.547772765159607 0.951444387435913
0.550725698471069 0.936974763870239
0.553678750991821 0.920689105987549
0.556631565093994 0.902619123458862
0.559584617614746 0.882799863815308
0.562537550926208 0.86126983165741
0.566967010498047 0.825859069824219
0.571396350860596 0.786847710609436
0.575825810432434 0.74440598487854
0.580255270004272 0.698719024658203
0.584684610366821 0.649985909461975
0.590590596199036 0.580638885498047
0.59649658203125 0.506793022155762
0.602402448654175 0.429020404815674
0.609784841537476 0.327203035354614
0.618643641471863 0.199907183647156
0.631932020187378 0.00314474105834961
0.648173093795776 -0.236731171607971
0.657032012939453 -0.362621188163757
0.664414405822754 -0.462794303894043
0.670320272445679 -0.538956165313721
0.676226139068604 -0.610942125320435
0.682132124900818 -0.678194284439087
0.686561584472656 -0.725211977958679
0.690990924835205 -0.769068002700806
0.695420503616333 -0.809571146965027
0.699849843978882 -0.846544861793518
0.70427930355072 -0.879827976226807
0.707232236862183 -0.899893999099731
0.710185170173645 -0.918216109275818
0.713138103485107 -0.93475866317749
0.716091156005859 -0.949489593505859
0.719043970108032 -0.962380528450012
0.721997022628784 -0.973406314849854
0.724949955940247 -0.982545614242554
0.726426362991333 -0.986402153968811
0.727902889251709 -0.98978066444397
0.729379415512085 -0.992679715156555
0.730855941772461 -0.995097637176514
0.732332348823547 -0.997033357620239
0.733808755874634 -0.998486042022705
0.73528528213501 -0.999454975128174
0.736761808395386 -0.999939441680908
0.738238215446472 -0.999939441680908
0.739714741706848 -0.999454975128174
0.741191148757935 -0.998486042022705
0.742667675018311 -0.997033357620239
0.744144201278687 -0.995097637176514
0.745620608329773 -0.992679715156555
0.747097015380859 -0.98978066444397
0.748573541641235 -0.986402153968811
0.750050067901611 -0.982545614242554
0.753003001213074 -0.973406314849854
0.755955934524536 -0.962380528450012
0.758908987045288 -0.949489593505859
0.761861801147461 -0.93475866317749
0.764814853668213 -0.918216109275818
0.767767786979675 -0.899893999099731
0.770720720291138 -0.879827976226807
0.7736736536026 -0.858056783676147
0.778103113174438 -0.822296142578125
0.782532453536987 -0.782950639724731
0.786962032318115 -0.740191698074341
0.791391372680664 -0.694205760955811
0.795820832252502 -0.645193457603455
0.801726698875427 -0.575506925582886
0.807632684707642 -0.501361131668091
0.813538551330566 -0.423330664634705
0.820920944213867 -0.321253299713135
0.829779863357544 -0.193740725517273
0.84306812286377 0.00314474105834961
0.859309315681458 0.242837190628052
0.868168115615845 0.368475437164307
0.875550508499146 0.468360543251038
0.88145637512207 0.544243335723877
0.887362360954285 0.615909337997437
0.893268346786499 0.68280291557312
0.897697687149048 0.729528069496155
0.902127146720886 0.773072719573975
0.906556606292725 0.813247203826904
0.910985946655273 0.849876165390015
0.915415406227112 0.882799863815308
0.918368339538574 0.902619123458862
0.921321392059326 0.920689105987549
0.924274206161499 0.936974763870239
0.927227258682251 0.951444387435913
0.930180191993713 0.964070320129395
0.933133125305176 0.974827766418457
0.936086058616638 0.983696103096008
0.937562584877014 0.98741626739502
0.939038991928101 0.990658044815063
0.940515518188477 0.993419647216797
0.941992044448853 0.995699882507324
0.943468451499939 0.997497797012329
0.944944858551025 0.998812198638916
0.946421384811401 0.999642848968506
0.947897911071777 0.999988794326782
0.949374437332153 0.999850511550903
0.95085084438324 0.999227523803711
0.952327251434326 0.998120307922363
0.953803777694702 0.996529579162598
0.955280303955078 0.994455933570862
0.956756830215454 0.991900444030762
0.958233237266541 0.988864302635193
0.959709644317627 0.98534893989563
0.961186170578003 0.981356143951416
0.964139223098755 0.971946239471436
0.967092037200928 0.960652589797974
0.97004508972168 0.947497248649597
0.972998023033142 0.93250560760498
0.975950956344604 0.915706872940063
0.978903889656067 0.897133350372314
0.981856822967529 0.876821279525757
0.984809875488281 0.854809880256653
0.98923921585083 0.818700790405273
0.993668675422668 0.779022574424744
0.998098134994507 0.735948085784912
1.00252747535706 0.689665079116821
1.00695693492889 0.640375375747681
1.01286292076111 0.570352077484131
1.01876878738403 0.495909333229065
1.02467465400696 0.417624235153198
1.03205704689026 0.315290927886963
1.04091596603394 0.187566637992859
1.05568063259125 -0.0314421653747559
1.07044541835785 -0.248933553695679
1.07930433750153 -0.374315023422241
1.08668673038483 -0.473908185958862
1.09259259700775 -0.549509048461914
1.09849846363068 -0.620851993560791
1.10440444946289 -0.687384486198425
1.10883378982544 -0.733815312385559
1.11326324939728 -0.777046918869019
1.11769270896912 -0.816890954971313
1.12212216854095 -0.853173732757568
1.12507510185242 -0.875304937362671
1.12802803516388 -0.895739793777466
1.13098096847534 -0.914438605308533
1.1339339017868 -0.931365251541138
1.13688683509827 -0.946486949920654
1.13983988761902 -0.959774374961853
1.14279282093048 -0.971201777458191
1.14574575424194 -0.980746984481812
1.14869868755341 -0.988391399383545
1.15017521381378 -0.991496086120605
1.15165162086487 -0.994120359420776
1.15312814712524 -0.996262907981873
1.15460455417633 -0.997922658920288
1.15608108043671 -0.999099016189575
1.15755760669708 -0.999791145324707
1.15903401374817 -0.999998807907104
1.16051054000854 -0.999721884727478
1.16198694705963 -0.998960494995117
1.16346347332001 -0.99771523475647
1.16493988037109 -0.995986342430115
1.16641640663147 -0.993774890899658
1.16789293289185 -0.991081953048706
1.16936933994293 -0.987908720970154
1.17084586620331 -0.984256744384766
1.17379879951477 -0.975524187088013
1.17675173282623 -0.964900970458984
1.1797046661377 -0.952407836914062
1.18265759944916 -0.938068866729736
1.18561065196991 -0.921911954879761
1.18856358528137 -0.903968334197998
1.19151651859283 -0.884272813796997
1.1944694519043 -0.862863540649414
1.19889891147614 -0.827628135681152
1.20332837104797 -0.788784503936768
1.20775771141052 -0.746502161026001
1.21218717098236 -0.700965404510498
1.2166166305542 -0.652372598648071
1.22252249717712 -0.583196401596069
1.22842848300934 -0.509501457214355
1.23433434963226 -0.431858897209167
1.24171674251556 -0.330173015594482
1.25057554244995 -0.202987432479858
1.26386392116547 -0.00628948211669922
1.28010511398315 0.233674645423889
1.28896391391754 0.359688758850098
1.29634630680084 0.460004329681396
1.30225229263306 0.536304712295532
1.30815815925598 0.608449459075928
1.31406402587891 0.675879955291748
1.31849348545074 0.723043203353882
1.32292294502258 0.767054200172424
1.32735240459442 0.807721138000488
1.33178174495697 0.844866752624512
1.33621120452881 0.878329038619995
1.33916413784027 0.898518085479736
1.34211707115173 0.916965961456299
1.34507012367249 0.933636784553528
1.34802305698395 0.948498129844666
1.35097599029541 0.96152138710022
1.35392892360687 0.972681045532227
1.35688185691833 0.981955766677856
1.35835838317871 0.985880374908447
1.3598347902298 0.989327430725098
1.36131131649017 0.992295026779175
1.36278784275055 0.994781732559204
1.36426424980164 0.99678635597229
1.36574077606201 0.998308181762695
1.3672171831131 0.999346137046814
1.36869370937347 0.999899864196777
1.37017011642456 0.999969124794006
1.37164664268494 0.999553799629211
1.37312316894531 0.998654127120972
1.3745995759964 0.997270584106445
1.37607610225677 0.99540376663208
1.37755250930786 0.993054628372192
1.37902903556824 0.990224242210388
1.38050556182861 0.986914157867432
1.3819819688797 0.983125686645508
1.38493490219116 0.974121809005737
1.38788783550262 0.963230133056641
1.39084088802338 0.950471758842468
1.39379382133484 0.935871362686157
1.3967467546463 0.919457197189331
1.39969968795776 0.9012610912323
1.40265262126923 0.881318330764771
1.40560555458069 0.859667539596558
1.41003501415253 0.824081659317017
1.41446447372437 0.784903049468994
1.4188939332962 0.742302536964417
1.42332327365875 0.696465849876404
1.42775273323059 0.647592902183533
1.43365859985352 0.578075766563416
1.43956458568573 0.504079580307007
1.44547045230865 0.426177620887756
1.45285284519196 0.324229836463928
1.46171176433563 0.196824908256531
1.47500002384186 0
};
\addlegendentry{$n=7$}
\end{axis}

\end{tikzpicture}
}
        \caption{Ondes stationnaires selon différents modes.}
        \label{fig:waves}
    \end{figure}

    Ensuite, nous avons consigné nos mesures, ainsi que les quelques résultats de calculs nécessaires au traitement
    de ces données, dans la Table~\ref{tab:mesure}.\\
    Dans la première colonne est notée la masse du contrepoids $m$, en gramme, qui est utilisé pour tendre la corde.
    Juste à côté est consignée la tension $T$ de la corde, en Newton, qui est égale à la force pesante du contrepoids,
    $F_P = T = m \cdot g$, avec $g$ est l'accélération terrestre, qui vaut $9.81 \ \left[\frac{m}{s^2}\right]$.\\
    Les deux colonnes suivantes contiennent la fréquence $f$ de vibration de la corde, en Hertz, ainsi que le mode
    $n$ de vibration, qui est le nombre de ventres présents sur la corde.
    Ce sont les deux valeurs mesurées à proprement parler lors de l'expérience, avec la longueur d'onde.\\
    Celle-ci est consignée dans la colonne suivante, en mètre, et est consignée sous le symbole $\lambda$ pour celle
    que nous avons mesurée, et $\lambda_{th}$ pour celle que nous avons calculée théoriquement, via la formule
    $\lambda_{th} = \frac{2L}{n}$, où $L$ est la longueur de la corde qui vibre et $n$ le mode de vibration.\\
    Enfin, les trois dernières colonnes contiennent la célérité calculée via les différentes formules, soit
    $c = f \cdot \lambda$, $c_{th_1} = \sqrt{\frac{T}{\mu}}$ et $c_{th_2} = f \cdot \lambda_{th}$, en mètre par
    seconde.

    \begin{table}[H]
    \rowcolors{1}{lightgray}{} % Coloration des lignes
    \centering
        \begin{tabular}{|c|c||c|c||c|c||c|c|c|}
            \hline
            \rowcolor{gray} $m \ [g]$ & $T \ [N]$ & $f \ [Hz]$ & $n$ & $\lambda \ [m]$ & $\lambda_{th} \ [m]$ & $c \ \left[\frac{m}{s}\right]$ & $c_{th_1} \ \left[\frac{m}{s}\right]$ & $c_{th_2} \ \left[\frac{m}{s}\right]$ \\
            \hline
            110 & 1.079 & 45.65 & 5 & 0.590 & 0.590 & 26.93 & 26.68 & 26.93 \\
            \hline
            110 & 1.079 & 64.22 & 7 & 0.425 & 0.421 & 27.29 & 26.68 & 27.06 \\
            \hline
            110 & 1.079 & 26.26 & 3 & 1.010 & 0.983 & 26.52 & 26.68 & 25.82 \\
            \hline
            110 & 1.079 & 56.06 & 6 & 0.480 & 0.491 & 26.91 & 26.68 & 27.56 \\
            \hline
            60 & 5.886 & 34.00 & 5 & 0.585 & 0.590 & 19.89 & 19.70 & 20.06 \\
            \hline
            60 & 5.886 & 45.46 & 7 & 0.450 & 0.421 & 20.46 & 19.70 & 19.16 \\
            \hline
            160 & 1.570 & 57.09 & 5 & 0.587 & 0.590 & 33.51 & 32.17 & 33.68 \\
            \hline
            160 & 1.570 & 78.04 & 7 & 0.420 & 0.421 & 32.78 & 32.17 & 32.89 \\
            \hline
            210 & 2.060 & 64.06 & 5 & 0.565 & 0.590 & 36.19 & 36.86 & 37.80 \\
            \hline
            210 & 2.060 & 90.00 & 7 & 0.410 & 0.421 & 36.90 & 36.86 & 37.93 \\
            \hline
            260 & 2.551 & 67.71 & 5 & 0.590 & 0.590 & 39.95 & 41.01 & 39.95 \\
            \hline
            260 & 2.551 & 97.00 & 7 & 0.405 & 0.421 & 39.29 & 41.01 & 40.88 \\
            \hline
            310 & 3.041 & 75.90 & 5 & 0.590 & 0.590 & 44.78 & 44.78 & 44.78 \\
            \hline
            310 & 3.041 & 108.6 & 7 & 0.400 & 0.421 & 43.44 & 44.78 & 45.77 \\
            \hline
            360 & 3.532 & 82.00 & 5 & 0.580 & 0.590 & 47.56 & 48.26 & 48.38 \\
            \hline
            360 & 3.532 & 116.9 & 7 & 0.415 & 0.421 & 48.51 & 48.26 & 49.27 \\
            \hline
        \end{tabular}
        \caption{Mesures et calculs.}
        \label{tab:mesure}
        \rowcolors{1}{white}{} % Coloration des lignes
    \end{table}

    Finalement, nous avons représenté graphiquement les célérités calculées en fonction de la masse du contrepoids
    utilisé pour tendre la corde, dans la Figure~\ref{fig:lambda}.
    En plus des trois séries de points, nous avons également représenté la courbe correspondant à la célérité
    $c_{th} = \sqrt{\frac{T}{\mu}}$.

    \begin{figure}[H]
        \centering
        % This file was created with tikzplotlib v0.10.1.
\begin{tikzpicture}

\definecolor{darkgray178}{RGB}{178,178,178}
\definecolor{darkgreen}{RGB}{0,100,0}
\definecolor{firebrick166640}{RGB}{166,6,40}
\definecolor{lightgray204}{RGB}{204,204,204}
\definecolor{silver188}{RGB}{188,188,188}
\definecolor{slategray122104166}{RGB}{122,104,166}
\definecolor{steelblue52138189}{RGB}{52,138,189}
\definecolor{whitesmoke238}{RGB}{238,238,238}

\begin{axis}[
axis background/.style={fill=whitesmoke238},
axis line style={silver188},
height=10cm,
legend cell align={left},
legend style={
  fill opacity=0.8,
  draw opacity=1,
  text opacity=1,
  at={(0.97,0.03)},
  anchor=south east,
  draw=lightgray204,
  fill=whitesmoke238
},
tick pos=left,
width=10cm,
x grid style={darkgray178},
xlabel={\(\displaystyle m \ \left[kg\right]\)},
xmajorgrids,
xmin=0.06, xmax=0.36,
xtick style={color=black},
y grid style={darkgray178},
ylabel={\(\displaystyle c \ \left[\frac{m}{s}\right]\)},
ymajorgrids,
ymin=17.6528, ymax=50.7703428571429,
ytick style={color=black}
]
\addplot [draw=steelblue52138189, fill=steelblue52138189, mark=x, only marks]
table{%
x  y
0.11 26.9335
0.11 27.2935
0.11 26.5226
0.11 26.9088
0.0600000000000001 19.89
0.0600000000000001 20.457
0.16 33.51183
0.16 32.7768
0.21 36.1939
0.21 36.9
0.26 39.9489
0.26 39.285
0.31 44.781
0.31 43.44
0.36 47.56
0.36 48.5135
};
\addlegendentry{$c = f \cdot \lambda$}
\addplot [draw=firebrick166640, fill=firebrick166640, mark=x, only marks]
table{%
x  y
0.11 26.6770772387081
0.0600000000000001 19.7023272737004
0.16 32.173765710591
0.21 36.8596791901395
0.26 41.0136647960165
0.31 44.7839865353678
0.36 48.2606485658865
0.36 48.2606485658865
};
\addlegendentry{$c_{th_1} = \sqrt{\frac{T}{\mu}}$}
\addplot [draw=slategray122104166, fill=slategray122104166, mark=x, only marks]
table{%
x  y
0.11 26.9335
0.11 27.0641428571429
0.11 25.8223333333333
0.11 27.5628333333333
0.0600000000000001 20.06
0.0600000000000001 19.1581428571429
0.16 33.6831
0.16 32.8882857142857
0.21 37.7954
0.21 37.9285714285714
0.26 39.9489
0.26 40.8785714285714
0.31 44.781
0.31 45.7671428571429
0.36 48.38
0.36 49.265
};
\addlegendentry{$c_{th_2} = f \cdot \lambda_{th}$}
\addplot [thick, darkgreen]
table {%
0.059999942779541 19.7023277282715
0.0660605430603027 20.6734600067139
0.072121262550354 21.6009769439697
0.0781818628311157 22.4902763366699
0.0842424631118774 23.345724105835
0.0903030633926392 24.1709136962891
0.0963636636734009 24.9688472747803
0.102424263954163 25.7420597076416
0.108484864234924 26.4927139282227
0.114545464515686 27.222677230835
0.120606064796448 27.9335708618164
0.126666665077209 28.6268177032471
0.132727265357971 29.3036689758301
0.138787865638733 29.9652347564697
0.144848465919495 30.6125049591064
0.150909066200256 31.2463722229004
0.156969666481018 31.8676338195801
0.16303026676178 32.4770126342773
0.169090867042542 33.0751647949219
0.175151467323303 33.6626930236816
0.184242486953735 34.5252418518066
0.193333387374878 35.3667602539062
0.202424287796021 36.1887168884277
0.211515188217163 36.9924125671387
0.220606088638306 37.7790145874023
0.229696989059448 38.5495681762695
0.238787889480591 39.3050231933594
0.247878789901733 40.0462265014648
0.256969690322876 40.7739562988281
0.266060590744019 41.4889259338379
0.275151491165161 42.191780090332
0.284242391586304 42.8831176757812
0.293333292007446 43.5634841918945
0.302424192428589 44.2333869934082
0.311515092849731 44.8932952880859
0.320605993270874 45.543643951416
0.329697012901306 46.1848335266113
0.338787913322449 46.8172416687012
0.347878813743591 47.4412231445312
0.360000014305115 48.2606468200684
};
\addlegendentry{$c_{th} = \sqrt{\frac{T}{\mu}}$}
\end{axis}

\end{tikzpicture}

        \caption{Célérité en fonction de la masse.}
        \label{fig:lambda}
    \end{figure}

    \subsection{Incertitudes}

    Les plus grandes sources d'incertitudes se trouvent dans la mesure de la fréquence et de la longueur d'onde.
    L'incertitude sur la fréquence est très faible sur la lecture de l'appareil, mais la fréquence dans laquelle
    il est possible de trouver un mode spécifique varie d'environ $1$ $[Hz]$.
    Les fréquences allant de 45.7 $[Hz]$ à 116.9 $[Hz]$ l'incertitude varie entre $2.2\%$ et
    $0.9\%$ ($\dfrac{1}{65}=1.5\%$, $65$ $[Hz]$ étant une fréquence typique).
    L'incertitude sur la longueur d'onde quant à elle est de l'ordre de $1\%$, ($\dfrac{0.005}{0.45}=1.1\%$).
    Ainsi l'incertitude sur la célérité calculée selon ces paramètres,  $c=\lambda \cdot f$, est de l'ordre de
    $3\%$ ($1.5+1.1=2.6$).\\ Par contre, l'incertitude sur le calcul basée sur les caractéristiques de la corde,
    $c=\sqrt{\dfrac{F}{\mu}}$, est bien plus faible, de l'ordre de $0.2\%$.
    Ceci s'explique par une mesure précise de la masse linéaire de la corde.
    En comparant la célérité obtenue par les deux calculs, nous constatons que l'erreur ne dépasse pas l'écart
    imposé par l'incertitude.
    Nous pouvons donc dire que les mesures sont satisfaisantes.

    \section{Discussion des résultats}\label{sec:discussion-des-resultats}

    Nous pouvons voir que les célérités calculées via les différentes formules sont très proches les unes des autres,
    ce qui semble indiquer que les formules utilisées sont correctes.
    Tout d'abord, notons que malgré la difficulté à observer les modes pairs, nous avons tout de même pu avoir une
    mesure pour $n=6$.\\ \\
    Ensuite, nous observons sur la Figure~\ref{fig:lambda} que les célérités calculées via les propriétés de la corde
    ($c = \sqrt{\frac{T}{\mu}}$) sont très proches de la célérité mesurée ($c=f \cdot \lambda $) et celle calculée
    selon les conditions de l'expérience ($c = f \cdot \lambda_{th} $).
    La tendance générale est que celle mesurée est légèrement plus faible, et celle via les conditions de l'expérience
    légèrement plus élevée.
    Cela peut s'expliquer par plusieurs facteurs.\\ \\
    Tout d'abord à propos de la célérité propre aux propriétés de la corde, il faut tout d'abord mentionné que
    c'est un résultat admettant plusieurs hypothèses et approximations (voir Section~\ref{sec:introduction}), comme
    négliger la pesanteur, une tension uniforme, une perturbation de faible amplitude, etc.\\
    De plus, il faut noter que la masse linéique de la corde est calculée via la mesure de la masse de la corde,
    soumise à peu d'erreur, néanmoins le nombre de chiffres significatif est limité par la précision de la balance.
    D'un autre côté, la mesure de la longueur de la corde est également soumise à de l'erreur, donc la masse linéique
    comporte plusieurs sources d'erreur.\\
    Enfin, la tension sur la corde est peut-être légèrement différente de celle calculée, car la masse du contrepoids
    n'est pas vérifiée et peut être que le montage modifie légèrement la tension, via par exemple la poulie qui
    induit une force de frottement.\\ \\
    Ensuite, à propos de la célérité propre aux conditions de l'expérience, la principale source d'erreur est le
    montage de l'expérience.
    En effet, on a mesuré la distance entre la poulie et le vibreur en considérant que ce sont les bords de la corde
    qui crée l'onde et que ceux-ci sont fixes afin d'avoir la longueur $L$, or le vrai point fixe et le coinceur,
    le vibreur est néanmoins une bonne approximation de part sa fréquence élevée.\\ \\
    Finalement, la mesure de la longueur d'onde est assez approximative, car on ne peut pas coller la règle à la corde
    et il faut donc estimer la position du nœud.
    De plus, la fréquence est réglée manuellement via un générateur de fonction, dont le réglage se fait via une
    molette, ce qui n'est pas forcément très précis.

    \section{Conclusion}\label{sec:conclusion}

    En conclusion, nous sommes satisfaits des résultats obtenus, car ils sont cohérents avec les formules utilisées.
    L'expérience est très intéressante, car elle permet de visualiser les ondes stationnaires, et de plus elle permet
    de mettre en évidence que l'on peut calculer la célérité d'une onde en connaissant uniquement les propriétés
    de la corde, ce qui est très pratique, par exemple cela explique comment fonctionne un instrument à corde, et
    pourquoi il faut tendre ou détendre la corde pour l'accorder.
    Le montage pourrait néanmoins être amélioré, par exemple en créant la tension dans la corde via un système de
    mesure de la force, afin d'avoir une tension plus précise, ainsi qu'une fixation plus précise que la poulie pour
    mesurer la longueur de la corde.
    Deuxièmement, on pourrait attacher la corde au vibreur afin d'être plus exact sur la longueur de la corde.
    Ensuite, on pourrait utiliser un générateur de fonction avec un réglage plus fin, que l'on pourrait même
    brancher à un stroboscope afin de pouvoir le synchroniser avec la fréquence de l'onde et ainsi fixer l'onde
    et pouvoir mesurer la longueur d'onde plus précisément.
    Finalement, on pourrait monter le toute sur un support afin de pouvoir mesurer les longueurs d'onde de manière
    plus précise, et ainsi avoir une meilleure précision sur la célérité, par exemple en utilisant deux lasers que
    l'on pointerait sur les nœuds afin de mesurer précisément leur écartement.
    Malgré tout, cela demeurera une expérience très intéressante dans l'étude des ondes stationnaires.

\end{document}