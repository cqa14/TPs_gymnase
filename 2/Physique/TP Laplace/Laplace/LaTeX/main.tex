%! Author = cqa
%! Date = 02.05.23

\documentclass[11pt]{article}
% Basic Packages for Encoding (Input AND Output) and Langauge Support
\usepackage[utf8]{inputenc}
\usepackage[T1]{fontenc}
\usepackage[french]{babel}

% Change Layout with a User-Friendly Interface
\usepackage[margin=1in]{geometry}

% Include Pictures with a User-Friendly Interface
\usepackage{graphicx}
\usepackage{lmodern}
\usepackage{float}

% Extended Math Support from the Famous 'American Mathematical Society'
\usepackage{amsmath}
\usepackage{amsfonts}
\usepackage{amssymb}

% Just for Demonstration Purposes
\usepackage[math]{blindtext}

% For the chemistry
\usepackage{chemist}

% For use on computer
\usepackage{hyperref}

% For table color
\usepackage{colortbl}
\usepackage[table]{xcolor}

% Tableau verticale
\usepackage{rotating}

% Figure dans figure
\usepackage{subfig}

% Multicolumn
\usepackage{multirow}

% Preuves
\usepackage{amsthm}

% Footnote
\usepackage[bottom]{footmisc}

% PsTricks
\usepackage[usenames,dvipsnames]{pstricks}
\usepackage{pstricks-add}
\usepackage{epsfig}
\usepackage{pst-grad} % For gradients
\usepackage{pst-plot} % For axes
\usepackage[space]{grffile} % For spaces in paths
\usepackage{etoolbox} % For spaces in paths
\makeatletter % For spaces in paths
\patchcmd\Gread@eps{\@inputcheck#1 }{\@inputcheck"#1"\relax}{}{}
\makeatother

\usepackage{auto-pst-pdf}

% Graphiques
\usepackage{pgfplots}
\DeclareUnicodeCharacter{2212}{−}
\usepgfplotslibrary{groupplots,dateplot}
\usetikzlibrary{patterns,shapes.arrows}
\pgfplotsset{compat=newest}

%\usepgfplotslibrary{external}
%\tikzexternalize

% Titre
\usepackage[affil-it]{authblk}
\usepackage{textcomp}
\usepackage{wasysym}
\title{\textbf{TP Force de Laplace}}
\author{Camille Yerly, Romain Blondel}
\affil{2M8, Gymnase Auguste Piccard}

% Document
\begin{document}
    \rowcolors{1}{lightgray}{} % Coloration des lignes
    \maketitle

    \section{But}\label{sec:but}

    Le but de ce TP est de vérifier la validité de la formule de Laplace en variant les différents paramètres
    de celle-ci (le courant, la longueur et le champ magnétique (norme et direction)) et d'analyser les
    effets de ces paramètres sur ladite force.

    \section{Introduction}\label{sec:introduction}
    La force de Laplace est utilisée pour calculer l'effet des champs magnétiques sur les conducteurs
    électriques.
    La force de Laplace découle de la force de Lorentz qui décrit l'action du champ magnétique sur des
    particules chargées en mouvement.
    En raison de la facilité avec laquelle il est possible de générer des champs magnétiques et le fait que
    la force s'exerce à distante, on trouve de nombreuses applications pratiques à ce phénomène.
    Par exemple, l'imagerie médicale (IRM) repose sur ce principe.
    De plus, elle offre des applications indispensables dans notre vie de tous les jours, comme par
    exemple les moteurs électriques.
    En effet, dans ce cas, elle permet de créer une force perpendiculaire à la direction du courant, ce qui permet de
    faire tourner un moteur via un courant électrique dans un champ magnétique.\\
    Historiquement, cette force fut découverte en observant l'effet des courant électriques sur les champs
    magnétiques.
    Par exemple, une boussole aura un comportement anormal en présence de courants électriques.
    Cette force fut observée par plusieurs scientifiques, mais il fallut plusieurs années avant que Coulomb
    ne parviennent, au moyen d'une balance de torsion, à démontrer ce phénomène.\\
    En effet, le champ magnétique applique force perpendiculaire au sens du courant.
    La force de Lorentz décrit la force d'un champs magnétique sur une charge électrique.
    Elle s'applique selon la formule suivante:
    \[\vec{F}=q\cdot\vec{v}\times\vec{B}\]
    Avec :
    \begin{itemize}
        \item $\vec{F}$ la force de Lorenz en $[N]$
        \item $q$ la charge en $[C]$
        \item $\vec{v}$ la vitesse de la particule en $[\frac{m}{s}]$
        \item $\vec{B}$ le champ magnétique en $[T]$
    \end{itemize}
    La force de Laplace peut être décrite comme la résultante de toutes les forces de Lorentz dans un
    conducteur.
    La force de Laplace se décrit donc, non avec le charges, mais en rapport avec le conducteur de ces
    charges.
    Cette force s'exprime ainsi:
    \[\vec{F}=I\cdot\vec{L}\times\vec{B}\]
    Avec :
    \begin{itemize}
        \item $\vec{F}$ la force de Laplace en $[N]$
        \item $I$ le courant en $[A]$
        \item $\vec{L}$ la longueur du tronçon du conducteur en $[m]$
        \item $\vec{B}$ le champ magnétique en $[T]$
    \end{itemize}
    et une fois normalisée:
    \[|\vec{F_{La}}|=I\cdot|\vec{L}|\cdot|\vec{B}|\cdot\sin\alpha\]
    En prenant $t$ le temps pendant lequel un conducteur est traversé par une charge, $q=It$ et la vitesse
    de cette charges est $\vec{v}=\frac{\vec{L}}{t}$.
    En combinant les deux, nous pouvons voir $q\cdot\vec{v}=I\cdot\vec{L}$.
    Par conséquent, nous pouvons faire le lien entre ces deux forces.

    \section{Principe de mesure et description}\label{sec:principe-de-mesure-et-description}

    \subsection{Matériel}\label{subsec:materiel}

    \begin{itemize}
        \item Support pour la plaque et la bobine
        \item Plaques avec des fils conducteurs de diverses longueur
        \item Balance de précision
        \item Plusieurs aimants (identiques) et leur support afin de former un ``aimant''
        \item Statif
        \item Générateur
        \item Bobine avec roue graduée
    \end{itemize}

    \subsection{Déroulement}\label{subsec:deroulement}

    Nous avons fait quatre expériences permettant chacune de varier un paramètre de la formule:
    \[|\vec{F_{La}}|=I\cdot|\vec{L}|\cdot|\vec{B}|\cdot\sin\alpha\]
    le courant $I$, la longueur du fil $L$ , la norme du champ magnétique $B$ et l'angle $\alpha$
    entre $\vec{B}$ et $\vec{L}$.
    \subsubsection{Expérience 1: Courant}
    \begin{enumerate}
        \item Nous avons placé les six aimants sur sur leur support que nous avons mis sur la balance de
        précision et avons placée la plaque de longueur maximale dans le support entre les aimants.
        \item Nous ensuite fait varier le courant de -5 $[A]$ à 5 $[A]$ en relevant la masse tous les
        0.5 $[A]$.
        Pour chaque mesure, nous avons éteint puis rallumé le générateur afin d'éviter que le système
        chauffe et que les fils fondent.
    \end{enumerate}
    \subsubsection{Expérience 2: Longueur du fil}
    En gardant le même montage (même nombres d'aimants), nous avons cette fois fait changer les plaques,
    afin de faire varier la longueur du fil.
    Nous avons effectué chaque mesure à 5 $[A]$.
    Nous avons effectué les mesures pour chacune des six plaques.
    \subsubsection{Expérience 3: Norme du champ}
    \begin{enumerate}
        \item Nous avons remis la plaque de longueur maximale et avons gardé le courant à 5 $[A]$.
        \item Nous avons varier le nombre d'aimants, de 1 à 6, en faisant attention de les centrer à chaque
        fois et avons relevé des mesures pour chaque variation.
    \end{enumerate}
    \subsubsection{Expérience 4: Direction du champ}
    \begin{enumerate}
        \item Cette fois, nous avons remplacé la plaque par une bobine avec roue graduée (rapporteur) et
        l'avons centré de sorte à ce qu'elle ne dérange pas le support à aimant dans sa rotation et pour que,
        lorsque le rapporteur indiquait $0 \ [^\circ]$ la masse soit nulle.
        \item Avec un courant fixé à 2 $[V]$, nous avons fait varier l'angle entre la bobine et le champs
        magnétique de $-90 \ [^\circ]$ à $+90 \ [^\circ]$ en relevant la masse tous les $10 \ [^\circ]$.
    \end{enumerate}

    \subsection{Schémas}\label{subsec:schemas}

    \begin{minipage}{.45\textwidth}
        % \usepackage[usenames,dvipsnames]{pstricks}
% \usepackage{pstricks-add}
% \usepackage{epsfig}
% \usepackage{pst-grad} % For gradients
% \usepackage{pst-plot} % For axes
% \usepackage[space]{grffile} % For spaces in paths
% \usepackage{etoolbox} % For spaces in paths
% \makeatletter % For spaces in paths
% \patchcmd\Gread@eps{\@inputcheck#1 }{\@inputcheck"#1"\relax}{}{}
% \makeatother
% 
\begin{figure}[H]
\centering
\psscalebox{0.9 0.9} % Change this value to rescale the drawing.
{
\begin{pspicture}(0,-5.73)(7.47,1.37)
\definecolor{colour0}{rgb}{0.7019608,0.7019608,0.7019608}
\definecolor{colour1}{rgb}{0.4,0.4,0.4}
\definecolor{colour2}{rgb}{0.2,0.2,0.2}
\definecolor{colour3}{rgb}{0.8,0.8,0.8}
\definecolor{colour4}{rgb}{0.6,0.6,0.6}
\psframe[linecolor=colour0, linewidth=0.04, fillstyle=solid,fillcolor=colour0, dimen=outer](3.52,-5.31)(0.54,-5.71)
\psframe[linecolor=colour0, linewidth=0.04, fillstyle=solid,fillcolor=colour0, dimen=outer](2.66,1.11)(2.4,-5.35)
\psframe[linecolor=black, linewidth=0.04, fillstyle=solid,fillcolor=black, dimen=outer](3.94,-3.59)(2.14,-4.01)
\psframe[linecolor=colour1, linewidth=0.04, fillstyle=solid,fillcolor=colour1, dimen=outer](5.76,-3.69)(3.94,-3.89)
\psframe[linecolor=yellow, linewidth=0.04, fillstyle=solid,fillcolor=yellow, dimen=outer](5.84,-3.61)(5.74,-4.69)
\psframe[linecolor=black, linewidth=0.01, fillstyle=solid,fillcolor=red, dimen=outer](5.7,-4.33)(5.52,-4.97)
\psframe[linecolor=black, linewidth=0.01, fillstyle=solid, dimen=outer](6.1,-4.34)(5.92,-4.97)
\psframe[linecolor=colour2, linewidth=0.04, fillstyle=solid,fillcolor=colour2, dimen=outer](6.1,-4.85)(5.49,-5.2)
\psframe[linecolor=black, linewidth=0.04, fillstyle=solid,fillcolor=colour0, dimen=outer](6.38,-5.17)(5.2,-5.27)
\psframe[linecolor=black, linewidth=0.04, fillstyle=solid,fillcolor=black, dimen=outer](6.7,-5.23)(4.9,-5.73)
\psframe[linecolor=black, linewidth=0.04, fillstyle=gradient, gradlines=2000, dimen=outer](6.12,-5.39)(5.52,-5.63)
\psframe[linecolor=black, linewidth=0.04, fillstyle=solid,fillcolor=colour3, dimen=outer](5.92,0.93)(4.44,-2.13)
\psframe[linecolor=black, linewidth=0.04, fillstyle=solid,fillcolor=black, dimen=outer](5.64,0.51)(4.74,-0.11)
\pscircle[linecolor=black, linewidth=0.04, fillstyle=solid,fillcolor=colour4, dimen=outer](5.565,-0.675){0.155}
\pscircle[linecolor=black, linewidth=0.04, fillstyle=solid,fillcolor=colour4, dimen=outer](5.585,-1.195){0.155}
\psbezier[linecolor=blue, linewidth=0.04](5.06,-1.81)(3.94,-2.15)(1.56,-0.99)(4.44,-3.69)
\psbezier[linecolor=red, linewidth=0.04](5.54,-1.69)(5.54,-2.49)(3.14,-0.93)(4.8,-3.71)
\psline[linecolor=black, linewidth=0.04, arrowsize=0.05291667cm 2.0,arrowlength=1.4,arrowinset=0.0]{->}(0.72,0.75)(2.26,-0.15)
\psline[linecolor=black, linewidth=0.04, arrowsize=0.05291667cm 2.0,arrowlength=1.4,arrowinset=0.0]{->}(6.1,-3.39)(5.92,-3.95)
\psline[linecolor=black, linewidth=0.04, arrowsize=0.05291667cm 2.0,arrowlength=1.4,arrowinset=0.0]{->}(6.64,-4.21)(6.2,-4.57)
\psline[linecolor=black, linewidth=0.04, arrowsize=0.05291667cm 2.0,arrowlength=1.4,arrowinset=0.0]{->}(4.48,-4.77)(4.82,-5.15)
\psline[linecolor=black, linewidth=0.04, arrowsize=0.05291667cm 2.0,arrowlength=1.4,arrowinset=0.0]{->}(3.64,1.05)(4.32,0.11)
\psline[linecolor=black, linewidth=0.04, arrowsize=0.05291667cm 2.0,arrowlength=1.4,arrowinset=0.0]{->}(1.34,-2.77)(3.76,-3.53)
\rput[bl](0.0,0.77){Statif}
\rput[bl](0.32,-2.79){Support}
\rput[bl](5.56,-3.41){Plaque}
\rput[bl](2.84,1.09){Générateur}
\rput[bl](6.28,-4.15){Aimant}
\rput[bl](3.72,-4.73){Balance}
\end{pspicture}
}
\caption{Schéma du montage principal}
\label{fig:schema-principal}
\end{figure}


    \end{minipage}
    \hfill
    \begin{minipage}{.45\textwidth}
        % \usepackage[usenames,dvipsnames]{pstricks}
% \usepackage{pstricks-add}
% \usepackage{epsfig}
% \usepackage{pst-grad} % For gradients
% \usepackage{pst-plot} % For axes
% \usepackage[space]{grffile} % For spaces in paths
% \usepackage{etoolbox} % For spaces in paths
% \makeatletter % For spaces in paths
% \patchcmd\Gread@eps{\@inputcheck#1 }{\@inputcheck"#1"\relax}{}{}
% \makeatother
% 
\begin{figure}[H]
\centering
\psscalebox{0.8 0.8} % Change this value to rescale the drawing.
{
\begin{pspicture}(0,-5.98)(9.35,2.4)
\definecolor{colour3}{rgb}{0.8,0.8,0.8}
\definecolor{colour1}{rgb}{0.4,0.4,0.4}
\definecolor{colour5}{rgb}{0.5019608,0.5019608,0.5019608}
\definecolor{colour6}{rgb}{0.8,0.5019608,0.2}
\definecolor{colour4}{rgb}{0.6,0.6,0.6}
\definecolor{colour2}{rgb}{0.2,0.2,0.2}
\psframe[linecolor=black, linewidth=0.04, fillstyle=solid,fillcolor=colour3, dimen=outer](2.28,2.11)(0.32,-1.75)
\psbezier[linecolor=blue, linewidth=0.04](0.82,-1.31)(2.68,-2.13)(4.16,-0.91)(5.56,-1.81)
\psbezier[linecolor=red, linewidth=0.04](1.56,-1.37)(2.78,-0.39)(4.38,-0.79)(5.84,-1.83)
\psframe[linecolor=colour3, linewidth=0.04, fillstyle=solid,fillcolor=colour3, dimen=outer](5.18,-5.59)(1.54,-5.97)
\psframe[linecolor=colour3, linewidth=0.04, fillstyle=solid,fillcolor=colour3, dimen=outer](4.24,2.4)(3.88,-5.63)
\psframe[linecolor=black, linewidth=0.04, fillstyle=solid,fillcolor=black, dimen=outer](4.94,-1.65)(2.96,-2.15)
\psframe[linecolor=colour1, linewidth=0.04, fillstyle=solid,fillcolor=colour1, dimen=outer](6.66,-1.81)(4.94,-2.05)
\psframe[linecolor=black, linewidth=0.04, fillstyle=solid,fillcolor=black, dimen=outer](7.8,-0.67)(6.66,-3.43)
\psframe[linecolor=colour5, linewidth=0.04, fillstyle=solid,fillcolor=colour5, dimen=outer](8.32,-2.49)(6.14,-2.61)
\psrotate(7.235, -4.055){-180.0}{\pstriangle[linecolor=black, linewidth=0.04, fillstyle=solid,fillcolor=black, dimen=outer](7.235,-4.72)(1.19,1.33)}
\psframe[linecolor=black, linewidth=0.04, fillstyle=vlines*,fillcolor=colour6, hatchwidth=0.028222222, hatchangle=0.0, hatchsep=0.0412, dimen=outer](7.38,-4.27)(7.06,-4.91)
\psframe[linecolor=black, linewidth=0.04, fillstyle=solid,fillcolor=black, dimen=outer](8.67,-5.35)(5.84,-5.98)
\psframe[linecolor=black, linewidth=0.04, fillstyle=solid,fillcolor=colour4, dimen=outer](8.16,-5.21)(6.34,-5.37)
\psframe[linecolor=black, linewidth=0.012, fillstyle=solid,fillcolor=red, dimen=outer](6.9,-4.81)(6.64,-5.21)
\psframe[linecolor=black, linewidth=0.012, dimen=outer](7.88,-4.8)(7.64,-5.23)
\psframe[linecolor=black, linewidth=0.04, fillstyle=solid,fillcolor=colour2, dimen=outer](7.89,-4.96)(6.61,-5.22)
\pscircle[linecolor=black, linewidth=0.04, fillstyle=solid,fillcolor=colour4, dimen=outer](1.765,0.325){0.205}
\pscircle[linecolor=black, linewidth=0.04, fillstyle=solid,fillcolor=colour5, dimen=outer](1.745,-0.195){0.205}
\psframe[linecolor=black, linewidth=0.04, fillstyle=solid,fillcolor=black, dimen=outer](1.94,1.69)(0.68,0.83)
\psframe[linecolor=black, linewidth=0.04, fillstyle=gradient, gradlines=2000, dimen=outer](7.68,-5.49)(6.92,-5.85)
\psline[linecolor=black, linewidth=0.04, arrowsize=0.05291667cm 2.0,arrowlength=1.4,arrowinset=0.0]{->}(0.96,-2.43)(1.12,-1.79)
\psline[linecolor=black, linewidth=0.04, arrowsize=0.05291667cm 2.0,arrowlength=1.4,arrowinset=0.0]{->}(2.4,-4.77)(3.68,-5.11)
\psline[linecolor=black, linewidth=0.04, arrowsize=0.05291667cm 2.0,arrowlength=1.4,arrowinset=0.0]{->}(2.74,-3.01)(4.8,-2.23)
\rput[bl](0.0,-2.69){Générateur}
\rput[bl](1.56,-4.73){Statif}
\rput[bl](1.92,-3.41){Support}
\psline[linecolor=black, linewidth=0.04, arrowsize=0.05291667cm 2.0,arrowlength=1.4,arrowinset=0.0]{->}(5.36,-4.81)(5.8,-5.29)
\psline[linecolor=black, linewidth=0.04, arrowsize=0.05291667cm 2.0,arrowlength=1.4,arrowinset=0.0]{->}(5.7,-3.59)(6.58,-4.81)
\rput[bl](4.42,-4.75){Balance}
\rput[bl](4.68,-3.53){Aimant}
\psline[linecolor=black, linewidth=0.04, arrowsize=0.05291667cm 2.0,arrowlength=1.4,arrowinset=0.0]{->}(5.66,0.19)(6.58,-1.15)
\rput[bl](4.38,0.17){Bobine avec roue graduée}
\end{pspicture}
}
\caption{Schéma du montage pour faire varier l'angle}
\label{fig:schema-angle}
\end{figure}


    \end{minipage}


    \section{Résultats et calculs}\label{sec:resultats-et-calculs}

    Ci-dessous, les résultats obtenus lors de l'expérience.
    Afin de mesurer la force de Laplace, nous avons utilisé la troisième loi de Newton.
    En effet, la force de Laplace exercée sur le conducteur est égale à la force exercée
    par le conducteur sur le champ magnétique.
    En d'autres termes, la force de Laplace que subit la plaque est égale à la force de
    exercée sur l'aimant (au signe près, mais cela est négligeable dans le but de vérifier la formule).
    En mettant l'aimant sur une balance, nous avons pu mesurer cette dernière, en mettant la balance à 0,
    puis en relevant la masse affichée lorsque du courant passe dans la plaque.
    Comme la balance convertit la force en masse, nous avons dû opérer l'opération inverse afin d'obtenir
    la force de réponse qui est égale à la force de Laplace : $ m = \frac{F}{g} \Leftrightarrow F = m \cdot g $,
    avec $ g = 9.81 \left[\frac{m}{s^2}\right] $.\\ \\
    Les paramètres de l'expérience sont maximaux sauf pour celui qui est étudié.
    Cela veut donc dire que le courant est de $5 \ [A]$, il y a les 6 aimants pour créer le champ magnétique,
    la plaque est à 90 degrés et la longueur du fil est de $8.4 \ [cm]$.
    De plus, le voltage du circuit est plafonné entre $3 \text{ et } 4 \ [V]$. \\
    À noter encore que la balance est sensible à $0.01 \ [g]$ près, et que l'on considère le champ magnétique
    comme étant homogène.
    Finalement, dans les tableaux ci-dessous, les valeurs de $F$ sont arrondies à $10^{-3} \ [N]$ près,
    cela juste afin de donner un ordre d'idée, néanmoins les calculs et graphiques sont faits avec les valeurs
    exactes. \\ \\
    Les deux premiers tableaux ci-dessous montrent les mesures en faisant varier le courant.
    Le premier contient les valeurs de courant recommender par le protocole, et le second est fait avec des
    valeurs de courant $ 10 \times$ plus petites dû à une mauvaise manipulation, mais les résultats sont
    pertinents et donc intégrés dans le rapport. \\
    Le troisième tableau montre les mesures en avec un angle différent.
    Afin de modifier ce facteur, nous avons dû utilisé un montage différent, comme montré sur le
    schéma de la figure~\ref{fig:schema-angle}.
    De plus, le courant sur cette mesure est fixé à $2 \ [A]$ afin de ne pas abimer le matériel.

    \begin{minipage}{.6\textwidth}
        \begin{table}[H]
            \centering
            \begin{tabular}{|c|c|c|}
                \hline
                \rowcolor{gray} $I \ [A]$ & $m \ [g]$ & $F \ [N]$ \\
                \hline
                -5 & -3.30 & -0.032 \\
                \hline
                -4.5 & -2.97 & -0.029 \\
                \hline
                -4 & -2.65 & -0.026 \\
                \hline
                -3.5 & -2.33 & -0.023 \\
                \hline
                -3 & -2.00 & -0.020 \\
                \hline
                -2.5 & -1.67 & -0.016 \\
                \hline
                -2 & -1.34 & -0.013 \\
                \hline
                -1.5 & -1.01 & -0.010 \\
                \hline
                -1 & -0.66 & -0.006 \\
                \hline
                -0.5 & -0.31 & -0.003 \\
                \hline
                0 & 0.00 & 0.000 \\
                \hline
                0.5 & 0.32 & 0.003 \\
                \hline
                1 & 0.65 & 0.006 \\
                \hline
                1.5 & 0.98 & 0.010 \\
                \hline
                2 & 1.30 & 0.013 \\
                \hline
                2.5 & 1.63 & 0.016 \\
                \hline
                3 & 1.95 & 0.019 \\
                \hline
                3.5 & 2.29 & 0.022 \\
                \hline
                4 & 2.60 & 0.026 \\
                \hline
                4.5 & 2.93 & 0.029 \\
                \hline
                5 & 3.26 & 0.032 \\
                \hline
            \end{tabular}
            \hspace{1cm}
            \begin{tabular}{|c|c|c|}
                \hline
                \rowcolor{gray} $I \ [A]$ & $m \ [g]$ & $F \ [N]$ \\
                \hline
                -0.5 & -0.31 & -0.003 \\
                \hline
                -0.45 & -0.29 & -0.003 \\
                \hline
                -0.4 & -0.26 & -0.003 \\
                \hline
                -0.35 & -0.24 & -0.002 \\
                \hline
                -0.3 & -0.21 & -0.002 \\
                \hline
                -0.25 & -0.16 & -0.002 \\
                \hline
                -0.2 & -0.12 & -0.001 \\
                \hline
                -0.15 & -0.10 & -0.001 \\
                \hline
                -0.1 & -0.06 & -0.001 \\
                \hline
                -0.05 & -0.02 & -0.000 \\
                \hline
                0 & 0.00 & 0.000 \\
                \hline
                0.05 & 0.03 & 0.000 \\
                \hline
                0.1 & 0.06 & 0.001 \\
                \hline
                0.15 & 0.09 & 0.001 \\
                \hline
                0.2 & 0.12 & 0.001 \\
                \hline
                0.25 & 0.16 & 0.002 \\
                \hline
                0.3 & 0.20 & 0.002 \\
                \hline
                0.35 & 0.22 & 0.002 \\
                \hline
                0.4 & 0.26 & 0.003 \\
                \hline
                0.45 & 0.29 & 0.003 \\
                \hline
                0.5 & 0.32 & 0.003 \\
                \hline
            \end{tabular}
            \caption{Mesures de la force de Laplace en fonction de l'intensité du courant}
            \label{tab:mesure-cour}
        \end{table}
    \end{minipage}
    \hfill
    \begin{minipage}{.3\textwidth}
        \begin{table}[H]
            \centering
            \begin{tabular}{|c|c|c|c|}
                \hline
                \rowcolor{gray} $\alpha \ [^\circ]$ & $m \ [g]$ & $F \ [N]$ \\
                \hline
                -90 & -0.99 & -0.010 \\
                \hline
                -80 & -0.97 & -0.010 \\
                \hline
                -70 & -0.94 & -0.009 \\
                \hline
                -60 & -0.85 & -0.008 \\
                \hline
                -50 & -0.76 & -0.007 \\
                \hline
                -40 & -0.64 & -0.006 \\
                \hline
                -30 & -0.50 & -0.005 \\
                \hline
                -20 & -0.35 & -0.003 \\
                \hline
                -10 & -0.17 & -0.002 \\
                \hline
                0 & 0.00 & 0.000 \\
                \hline
                10 & 0.15 & 0.001 \\
                \hline
                20 & 0.32 & 0.003 \\
                \hline
                30 & 0.48 & 0.005 \\
                \hline
                40 & 0.62 & 0.006 \\
                \hline
                50 & 0.73 & 0.007 \\
                \hline
                60 & 0.83 & 0.008 \\
                \hline
                70 & 0.89 & 0.009 \\
                \hline
                80 & 0.92 & 0.009 \\
                \hline
                90 & 0.95 & 0.009 \\
                \hline
            \end{tabular}
            \caption{Mesures de la force de Laplace en fonction de l'angle}
            \label{tab:mesure-angle}
        \end{table}
    \end{minipage}

    Les deux derniers tableau sont à nouveau sur le premier montage, figure~\ref{fig:schema-principal}.
    Les plaques étant nommée, ceux-ci sont reporté dans le tableau~\ref{tab:mesure-longueur}, et pour
    les autres nous avons uniquement utilisé celle ``SF 42''.\\
    La table~\ref{tab:mesure-aimants} relève les résultats de la mesure de la force de Laplace en fonction
    du nombre de petits aimants composant l'aimant qui crée le champ magnétique.
    Il est toutefois à noter que nous considérons dans tout les cas le champ magnétique comme homogène.

    \begin{minipage}{.45\textwidth}
        \begin{table}[H]
            \centering

            \begin{tabular}{|c|c|c|c|}
                \hline
                \rowcolor{gray} $l \ [cm]$ & \# SF & $m \ [g]$ & $F \ [N]$ \\
                \hline
                1.2 & 40 & 0.42 & 0.004 \\
                \hline
                2.2 & 37 & 0.89 & 0.009 \\
                \hline
                3.2 & 39 & 1.31 & 0.013 \\
                \hline
                4.2 & 38 & 1.74 & 0.017 \\
                \hline
                6.4 & 41 & 2.50 & 0.025 \\
                \hline
                8.4 & 42 & 3.26 & 0.032 \\
                \hline
            \end{tabular}
            \caption{Mesures de la force de Laplace en fonction de la longueur du fil}
            \label{tab:mesure-longueur}
        \end{table}
    \end{minipage}
    \hfill
    \begin{minipage}{.45\textwidth}
        \begin{table}[H]
            \centering
            \begin{tabular}{|c|c|c|c|}
                \hline
                \rowcolor{gray} \# aimants & $m \ [g]$ & $F \ [N]$ \\
                \hline
                1 & 0.76 & 0.007 \\
                \hline
                2 & 1.20 & 0.012 \\
                \hline
                3 & 1.84 & 0.018 \\
                \hline
                4 & 2.47 & 0.024 \\
                \hline
                5 & 3.24 & 0.032 \\
                \hline
                6 & 3.26 & 0.032 \\
                \hline
            \end{tabular}
            \caption{Mesures de la force de Laplace en fonction du nombre d'aimants}
            \label{tab:mesure-aimants}
        \end{table}
    \end{minipage}\\

    Afin de pouvoir étudié les résultats, nous avons représenté les données sous forme de graphique.
    Sur chaque graphique, ont a reporté nos mesures, et afin de simplifier la lecture, nous avons
    ajouté une courbe de tendance, obtenue par la méthode des moindres carrés.
    Pour tout les graphes sauf la figure~\ref{fig:angle}, c'est une fonction linéaire, soit d'équation
    $ f(x) = a \cdot x $, dont le facteur $a$ est indiqué dans la légende, arrondi au millième.
    Pour la dernière, c'est une fonction linéaire du sinus, soit d'équation $ f(x) = a \cdot \sin(x) $,
    également indiqué dans la légende. \\ \\
    Sur les graphes des figures~\ref{fig:courant},~\ref{fig:longueur} et~\ref{fig:aimants}, nous avons
    également ajouté une droite obtenue en calculant via les valeurs théoriques de la force de Laplace :
    $ F = I \cdot L \cdot B $, avec pour les fois où c'est fixe $L = 8.4 \ [cm]$ et $I = 5 \ [A]$.
    Le champs $B$ est obtenu via la mesure ``\# 6'' de la table~\ref{tab:mesure-aimants}, via la formule
    $ F = I \cdot L \cdot B \Leftrightarrow B = \frac{F}{I \cdot L} \approx 0.076 \ [T] $.
    Pour la figure~\ref{fig:aimants}, nous avons exprimé le nombre d'aimant $N$, puis pour la courbe en trait
    non-continu, nous avons utilisé la formule $F = I \cdot L \cdot \frac{B \cdot N}{6} $ pour modéliser
    la variation du champ magnétique selon le nombre de petits aimants.\\
    Pour la figure~\ref{fig:angle}, nous ne pouvions pas faire de même, ne connaissant pas la longueur
    de la bobine.\\ \\
    Pour la variation de courant, figure~\ref{fig:courant}, nous avons séparé les graphes afin de pouvoir
    mieux voir les résultats, et nous avons également ajouté un graphe avec toutes les mesures, afin de
    pouvoir bien constater la tendance générale.\\

    \begin{figure}[H]
        \centering
        \subfloat[Mesure normale]{% This file was created with tikzplotlib v0.10.1.
\begin{tikzpicture}

\definecolor{darkgray178}{RGB}{178,178,178}
\definecolor{firebrick166640}{RGB}{166,6,40}
\definecolor{lightgray204}{RGB}{204,204,204}
\definecolor{silver188}{RGB}{188,188,188}
\definecolor{slategray122104166}{RGB}{122,104,166}
\definecolor{steelblue52138189}{RGB}{52,138,189}
\definecolor{whitesmoke238}{RGB}{238,238,238}

\begin{axis}[
axis background/.style={fill=whitesmoke238},
axis line style={silver188},
height=9cm,
legend cell align={left},
legend style={
  fill opacity=0.8,
  draw opacity=1,
  text opacity=1,
  at={(0.03,0.97)},
  anchor=north west,
  draw=lightgray204,
  fill=whitesmoke238
},
tick pos=left,
width=16cm,
x grid style={darkgray178},
xlabel={Courant \(\displaystyle [A]\)},
xmajorgrids,
xmin=-5, xmax=5,
xtick style={color=black},
y grid style={darkgray178},
ylabel={Force \(\displaystyle [N]\)},
ymajorgrids,
ymin=-0.0356034202597424, ymax=0.03546582545459,
ytick style={color=black}
]
\addplot [thick, steelblue52138189, mark=x, mark size=3, mark options={solid}, only marks]
table {%
-5 -0.0323729515075684
-4.5 -0.0291357040405273
-4 -0.0259964466094971
-3.5 -0.0228573083877563
-3 -0.0196199417114258
-2.5 -0.0163826942443848
-2 -0.0131454467773438
-1.5 -0.00990808010101318
-1 -0.00647461414337158
-0.5 -0.00304114818572998
0 0
0.5 0.00313925743103027
1 0.00637650489807129
1.5 0.0096137523651123
2 0.0127530097961426
2.5 0.0159902572631836
3 0.0191295146942139
3.5 0.0224648714065552
4 0.0255060195922852
4.5 0.0287432670593262
5 0.0319806337356567
};
\addlegendentry{Mesures}
\addplot [thick, firebrick166640]
table {%
-5 -0.0322353839874268
5 0.0322353839874268
};
\addlegendentry{$F = 0.006 \cdot I$}
\addplot [thick, slategray122104166, dashed]
table {%
-5 -0.0319806337356567
5 0.0319806337356567
};
\addlegendentry{$F = I \cdot L \cdot B$}
\end{axis}

\end{tikzpicture}
}\\
        \subfloat[Courant $10 \times$ plus petit]{% This file was created with tikzplotlib v0.10.1.
\begin{tikzpicture}

\definecolor{darkgray178}{RGB}{178,178,178}
\definecolor{firebrick166640}{RGB}{166,6,40}
\definecolor{lightgray204}{RGB}{204,204,204}
\definecolor{silver188}{RGB}{188,188,188}
\definecolor{slategray122104166}{RGB}{122,104,166}
\definecolor{steelblue52138189}{RGB}{52,138,189}
\definecolor{whitesmoke238}{RGB}{238,238,238}

\begin{axis}[
axis background/.style={fill=whitesmoke238},
axis line style={silver188},
height=8cm,
legend cell align={left},
legend style={
  fill opacity=0.8,
  draw opacity=1,
  text opacity=1,
  at={(0.03,0.97)},
  anchor=north west,
  draw=lightgray204,
  fill=whitesmoke238
},
tick pos=left,
width=8cm,
x grid style={darkgray178},
xlabel={Courant \(\displaystyle [A]\)},
xmajorgrids,
xmin=-0.5, xmax=0.5,
xtick style={color=black},
y grid style={darkgray178},
ylabel={Force \(\displaystyle [N]\)},
ymajorgrids,
ymin=-0.003517866, ymax=0.003517866,
ytick style={color=black}
]
\addplot [thick, steelblue52138189, mark=x, mark size=3, mark options={solid}, only marks]
table {%
-0.5 -0.00304114818572998
-0.450000047683716 -0.00284492969512939
-0.399999976158142 -0.00255060195922852
-0.350000023841858 -0.00235438346862793
-0.299999952316284 -0.00206005573272705
-0.25 -0.00156962871551514
-0.200000047683716 -0.00117719173431396
-0.149999976158142 -0.000980973243713379
-0.100000023841858 -0.000588655471801758
-0.0499999523162842 -0.000196218490600586
0 0
0.0499999523162842 0.000294327735900879
0.100000023841858 0.000588655471801758
0.149999976158142 0.000882863998413086
0.200000047683716 0.00117719173431396
0.25 0.00156962871551514
0.299999952316284 0.00196194648742676
0.350000023841858 0.00215816497802734
0.399999976158142 0.00255060195922852
0.450000047683716 0.00284492969512939
0.5 0.00313925743103027
};
\addlegendentry{Mesures}
\addplot [thick, firebrick166640]
table {%
-0.5 -0.00315701961517334
0.5 0.00315701961517334
};
\addlegendentry{$F = 0.006 \cdot I$}
\addplot [thick, slategray122104166, dashed]
table {%
-0.5 -0.00319802761077881
0.5 0.00319802761077881
};
\addlegendentry{$F = I \cdot L \cdot B$}
\end{axis}

\end{tikzpicture}
}
        \hfill
        \subfloat[Toutes les mesures]{% This file was created with tikzplotlib v0.10.1.
\begin{tikzpicture}

\definecolor{darkgray178}{RGB}{178,178,178}
\definecolor{firebrick166640}{RGB}{166,6,40}
\definecolor{lightgray204}{RGB}{204,204,204}
\definecolor{silver188}{RGB}{188,188,188}
\definecolor{slategray122104166}{RGB}{122,104,166}
\definecolor{steelblue52138189}{RGB}{52,138,189}
\definecolor{whitesmoke238}{RGB}{238,238,238}

\begin{axis}[
axis background/.style={fill=whitesmoke238},
axis line style={silver188},
height=8cm,
legend cell align={left},
legend style={
  fill opacity=0.8,
  draw opacity=1,
  text opacity=1,
  at={(0.03,0.97)},
  anchor=north west,
  draw=lightgray204,
  fill=whitesmoke238
},
tick pos=left,
width=8cm,
x grid style={darkgray178},
xlabel={Courant \(\displaystyle [A]\)},
xmajorgrids,
xmin=-5, xmax=5,
xtick style={color=black},
y grid style={darkgray178},
ylabel={Force \(\displaystyle [N]\)},
ymajorgrids,
ymin=-0.0356034202597424, ymax=0.03546582545459,
ytick style={color=black}
]
\addplot [thick, steelblue52138189, mark=x, mark size=3, mark options={solid}, only marks]
table {%
-5 -0.0323729515075684
-4.5 -0.0291357040405273
-4 -0.0259964466094971
-3.5 -0.0228573083877563
-3 -0.0196199417114258
-2.5 -0.0163826942443848
-2 -0.0131454467773438
-1.5 -0.00990808010101318
-1 -0.00647461414337158
-0.5 -0.00304114818572998
0 0
1 0.00637650489807129
1.5 0.0096137523651123
2 0.0127530097961426
2.5 0.0159902572631836
3 0.0191295146942139
3.5 0.0224648714065552
4 0.0255060195922852
4.5 0.0287432670593262
5 0.0319806337356567
-0.450000047683716 -0.00284492969512939
-0.399999976158142 -0.00255060195922852
-0.350000023841858 -0.00235438346862793
-0.299999952316284 -0.00206005573272705
-0.25 -0.00156962871551514
-0.200000047683716 -0.00117719173431396
-0.149999976158142 -0.000980973243713379
-0.100000023841858 -0.000588655471801758
-0.0499999523162842 -0.000196218490600586
0.0499999523162842 0.000294327735900879
0.100000023841858 0.000588655471801758
0.149999976158142 0.000882863998413086
0.200000047683716 0.00117719173431396
0.25 0.00156962871551514
0.299999952316284 0.00196194648742676
0.350000023841858 0.00215816497802734
0.399999976158142 0.00255060195922852
0.450000047683716 0.00284492969512939
0.5 0.00313925743103027
};
\addlegendentry{Mesures}
\addplot [thick, firebrick166640]
table {%
-5 -0.0322353839874268
5 0.0322353839874268
};
\addlegendentry{$F = 0.006 \cdot I$}
\addplot [thick, slategray122104166, dashed]
table {%
-5 -0.0319806337356567
5 0.0319806337356567
};
\addlegendentry{$F = I \cdot L \cdot B$}
\end{axis}

\end{tikzpicture}
}
        \caption{Graphes des mesures selon l'intensité du courant}
        \label{fig:courant}
    \end{figure}

    \begin{figure}[H]
        \centering
        % This file was created with tikzplotlib v0.10.1.
\begin{tikzpicture}

\definecolor{darkgray178}{RGB}{178,178,178}
\definecolor{firebrick166640}{RGB}{166,6,40}
\definecolor{lightgray204}{RGB}{204,204,204}
\definecolor{silver188}{RGB}{188,188,188}
\definecolor{slategray122104166}{RGB}{122,104,166}
\definecolor{steelblue52138189}{RGB}{52,138,189}
\definecolor{whitesmoke238}{RGB}{238,238,238}

\begin{axis}[
axis background/.style={fill=whitesmoke238},
axis line style={silver188},
height=9cm,
legend cell align={left},
legend style={
  fill opacity=0.8,
  draw opacity=1,
  text opacity=1,
  at={(0.03,0.97)},
  anchor=north west,
  draw=lightgray204,
  fill=whitesmoke238
},
tick pos=left,
width=16cm,
x grid style={darkgray178},
xlabel={Longueur \(\displaystyle [cm]\)},
xmajorgrids,
xmin=1.2, xmax=8.4,
xtick style={color=black},
y grid style={darkgray178},
ylabel={Force \(\displaystyle [N]\)},
ymajorgrids,
ymin=0.275669686985619, ymax=3.450936573302,
ytick style={color=black}
]
\addplot [thick, steelblue52138189, mark=x, mark size=3, mark options={solid}, only marks]
table {%
1.20000004768372 0.419999957084656
2.20000004768372 0.889999985694885
3.20000004768372 1.30999994277954
4.19999980926514 1.74000000953674
6.40000009536743 2.5
8.39999961853027 3.25999999046326
};
\addlegendentry{Mesures}
\addplot [thick, firebrick166640]
table {%
1.20000004768372 0.47237229347229
8.39999961853027 3.30660629272461
};
\addlegendentry{$F = 0.394 \cdot L$}
\addplot [thick, slategray122104166, dashed]
table {%
1.20000004768372 0.456865787506104
8.39999961853027 3.19806003570557
};
\addlegendentry{$F = I \cdot L \cdot B$}
\end{axis}

\end{tikzpicture}

        \caption{Graphes des mesures selon de la longueur du fil}
        \label{fig:longueur}
    \end{figure}

    \begin{figure}[H]
        \centering
        % This file was created with tikzplotlib v0.10.1.
\begin{tikzpicture}

\definecolor{darkgray178}{RGB}{178,178,178}
\definecolor{firebrick166640}{RGB}{166,6,40}
\definecolor{lightgray204}{RGB}{204,204,204}
\definecolor{silver188}{RGB}{188,188,188}
\definecolor{slategray122104166}{RGB}{122,104,166}
\definecolor{steelblue52138189}{RGB}{52,138,189}
\definecolor{whitesmoke238}{RGB}{238,238,238}

\begin{axis}[
axis background/.style={fill=whitesmoke238},
axis line style={silver188},
height=9cm,
legend cell align={left},
legend style={
  fill opacity=0.8,
  draw opacity=1,
  text opacity=1,
  at={(0.03,0.97)},
  anchor=north west,
  draw=lightgray204,
  fill=whitesmoke238
},
tick pos=left,
width=16cm,
x grid style={darkgray178},
xlabel={Nombre d'aimants},
xmajorgrids,
xmin=1, xmax=6,
xtick style={color=black},
y grid style={darkgray178},
ylabel={Force \(\displaystyle [N]\)},
ymajorgrids,
ymin=0.00383986038526585, ymax=0.0366251319094172,
ytick style={color=black}
]
\addplot [thick, steelblue52138189, mark=x, mark size=3, mark options={solid}, only marks]
table {%
1 0.00745558738708496
2 0.0117720365524292
3 0.0180504322052002
4 0.0242307186126709
5 0.0317844152450562
6 0.0319806337356567
};
\addlegendentry{Mesures}
\addplot [thick, firebrick166640]
table {%
1 0.00585579872131348
6 0.0351349115371704
};
\addlegendentry{$F = 0.006 \cdot N$}
\addplot [thick, slategray122104166, dashed]
table {%
1 0.00533008575439453
6 0.0319806337356567
};
\addlegendentry{$F = I \cdot L \cdot B$}
\end{axis}

\end{tikzpicture}

        \caption{Graphes des mesures selon le nombre d'aimants}
        \label{fig:aimants}
    \end{figure}

    \begin{figure}[H]
        \subfloat[Graphe selon l'angle]{% This file was created with tikzplotlib v0.10.1.
\begin{tikzpicture}

\definecolor{darkgray178}{RGB}{178,178,178}
\definecolor{firebrick166640}{RGB}{166,6,40}
\definecolor{lightgray204}{RGB}{204,204,204}
\definecolor{silver188}{RGB}{188,188,188}
\definecolor{steelblue52138189}{RGB}{52,138,189}
\definecolor{whitesmoke238}{RGB}{238,238,238}

\begin{axis}[
axis background/.style={fill=whitesmoke238},
axis line style={silver188},
height=8cm,
legend cell align={left},
legend style={
  fill opacity=0.8,
  draw opacity=1,
  text opacity=1,
  at={(0.03,0.97)},
  anchor=north west,
  draw=lightgray204,
  fill=whitesmoke238
},
tick pos=left,
width=8cm,
x grid style={darkgray178},
xlabel={Angle \(\displaystyle [^\circ]\)},
xmajorgrids,
xmin=-90, xmax=90,
xtick style={color=black},
y grid style={darkgray178},
ylabel={Force \(\displaystyle [N]\)},
ymajorgrids,
ymin=-0.0106733725254142, ymax=0.0104790230336984,
ytick style={color=black}
]
\addplot [thick, steelblue52138189, mark=x, mark size=3, mark options={solid}, only marks]
table {%
-90 -0.0097118616104126
-80 -0.00951564311981201
-70 -0.00922143459320068
-60 -0.00833845138549805
-50 -0.00745558738708496
-40 -0.006278395652771
-30 -0.00490498542785645
-20 -0.0034334659576416
-10 -0.00166773796081543
0 0
10 0.00147151947021484
20 0.00313925743103027
30 0.00470876693725586
40 0.00608217716217041
50 0.00716125965118408
60 0.00814235210418701
70 0.00873088836669922
80 0.0090252161026001
90 0.00931954383850098
};
\addlegendentry{Mesures}
\addplot [thick, firebrick166640]
table {%
-90 -0.00951755046844482
-86.326530456543 -0.0094980001449585
-82.6530609130859 -0.00943946838378906
-78.9795913696289 -0.00934207439422607
-75.3061218261719 -0.00920629501342773
-71.6326522827148 -0.0090327262878418
-67.9591827392578 -0.00882196426391602
-64.2857131958008 -0.0085749626159668
-60.6122436523438 -0.0082927942276001
-56.9387741088867 -0.00797653198242188
-53.2653045654297 -0.00762748718261719
-49.5918350219727 -0.00724709033966064
-45.9183654785156 -0.00683689117431641
-42.2448997497559 -0.00639867782592773
-38.5714302062988 -0.00593411922454834
-34.8979606628418 -0.00544512271881104
-31.2244892120361 -0.00493383407592773
-27.5510196685791 -0.0044022798538208
-23.8775501251221 -0.0038524866104126
-20.2040824890137 -0.00328707695007324
-16.5306129455566 -0.00270795822143555
-12.8571424484253 -0.00211787223815918
-5.51020431518555 -0.000913858413696289
9.18367385864258 0.00151896476745605
12.8571424484253 0.00211787223815918
16.5306129455566 0.00270795822143555
20.2040824890137 0.00328707695007324
23.8775501251221 0.0038524866104126
27.5510196685791 0.0044022798538208
31.2244892120361 0.00493383407592773
34.8979606628418 0.00544512271881104
38.5714302062988 0.00593411922454834
42.2448997497559 0.00639867782592773
45.9183654785156 0.00683689117431641
49.5918350219727 0.00724709033966064
53.2653045654297 0.00762748718261719
56.9387741088867 0.00797653198242188
60.6122436523438 0.0082927942276001
64.2857131958008 0.0085749626159668
67.9591827392578 0.00882196426391602
71.6326522827148 0.0090327262878418
75.3061218261719 0.00920629501342773
78.9795913696289 0.00934207439422607
82.6530609130859 0.00943946838378906
86.326530456543 0.0094980001449585
90 0.00951755046844482
};
\addlegendentry{$F = 0.01 \cdot \sin \alpha$}
\end{axis}

\end{tikzpicture}
}
        \hfill
        \subfloat[Graphe selon le sinus de l'angle]{% This file was created with tikzplotlib v0.10.1.
\begin{tikzpicture}

\definecolor{darkgray178}{RGB}{178,178,178}
\definecolor{firebrick166640}{RGB}{166,6,40}
\definecolor{lightgray204}{RGB}{204,204,204}
\definecolor{silver188}{RGB}{188,188,188}
\definecolor{steelblue52138189}{RGB}{52,138,189}
\definecolor{whitesmoke238}{RGB}{238,238,238}

\begin{axis}[
axis background/.style={fill=whitesmoke238},
axis line style={silver188},
height=8cm,
legend cell align={left},
legend style={
  fill opacity=0.8,
  draw opacity=1,
  text opacity=1,
  at={(0.03,0.97)},
  anchor=north west,
  draw=lightgray204,
  fill=whitesmoke238
},
tick pos=left,
width=8cm,
x grid style={darkgray178},
xlabel={\(\displaystyle \sin \alpha\)},
xmajorgrids,
xmin=-1, xmax=1,
xtick style={color=black},
y grid style={darkgray178},
ylabel={Force \(\displaystyle [N]\)},
ymajorgrids,
ymin=-0.0106733725254142, ymax=0.0104790230336984,
ytick style={color=black}
]
\addplot [thick, steelblue52138189, mark=x, mark size=3, mark options={solid}, only marks]
table {%
-1 -0.0097118616104126
-0.984807729721069 -0.00951564311981201
-0.939692616462708 -0.00922143459320068
-0.866025447845459 -0.00833845138549805
-0.76604437828064 -0.00745558738708496
-0.64278769493103 -0.006278395652771
-0.5 -0.00490498542785645
-0.342020153999329 -0.0034334659576416
-0.173648118972778 -0.00166773796081543
0 0
0.173648118972778 0.00147151947021484
0.342020153999329 0.00313925743103027
0.5 0.00470876693725586
0.64278769493103 0.00608217716217041
0.76604437828064 0.00716125965118408
0.866025447845459 0.00814235210418701
0.939692616462708 0.00873088836669922
0.984807729721069 0.0090252161026001
1 0.00931954383850098
};
\addlegendentry{Mesures}
\addplot [thick, firebrick166640]
table {%
-1 -0.00951755046844482
1 0.00951755046844482
};
\addlegendentry{$F = 0.01 \cdot \sin \alpha$}
\end{axis}

\end{tikzpicture}
}
        \caption{Graphes des mesures selon l'angle}
        \label{fig:angle}
    \end{figure}

    \section{Discussion des résultats}\label{sec:discussion-des-resultats}

    Nous observons bien dans nos résultats les relations de proportionnalité entre la force de Laplace
    et l'intensité du courant, la longueur du fil, l'intensité du champ magnétique et le sinus de l'angle. \\ \\
    Notons tout d'abord que via les tableaux, nous pouvons voir des mesures très petites pour les forces,
    ce qui est normal, car la force de Laplace est très faible, mais demeure tout de même mesurable.
    Il faut quand même remarquer que pour la percevoir, la balance est très sensible, et il faut donc
    faire attention à ne pas la faire bouger, car cela fausserait les mesures.
    De plus, le calibrage de celle-ci à zéro au point neutre de la mesure n'est pas toujours aisé,
    et il oscillait parfois avec $\pm 0.01 \ [g]$. \\ \\
    Ensuite, nous pouvons assez clairement voir la relation de proportionnalité entre les résultats.
    Sur la variation de courant, figure~\ref{fig:courant}, la courbe de tendance est très proche pour les
    deux échelles de mesure, et cela est confirmé par le graphe avec toutes les mesures.
    De plus, la droite théorique est également très proche, ce qui indique que la mesure en faisant varier
    le courant est très précise. \\ \\
    Pour la variation de la longueur du fil, figure~\ref{fig:longueur}, la courbe de tendance est un peu
    moins proche des mesures, tout comme la droite théorique, mais cela peut être mis sur le fait que la mesure
    de la longueur du fil est lié au fabricant et donc potentiellement moins précis que la mesure du courant.\\
    Pour la variation du nombre d'aimants, figure~\ref{fig:aimants}, les mesures semblent être plus dispersées,
    surtout entre les 5 premières mesures et la 6ème.
    Cela peut être attribué aux aimants qui serait mal centré ou qui ne serait pas parfaitement aligné
    (mais on a répété plusieurs fois ces mesures avec les mêmes résultats, donc on doute que ce soit
    la meilleure explication), ou, également plus ou moins lié à cela, le fait que l'on considère que
    le champ magnétique est homogène.
    Ces différences sont également visible sur la droite théorique, qui est faite en considérant que le champ
    est strictement proportionnel au nombre d'aimants, ce qui ne semble pas être le cas.
    Néanmoins, de part l'aspect proche entre cette dernière courbe et le point 2, il n'est pas à exclure les
    hypothèses précédentes.\\ \\
    Enfin, pour la variation de l'angle, figure~\ref{fig:angle}, nous pouvons voir que la courbe de tendance
    colle bien aux mesures, et la relation de proportionnalité est parfaitement visible sur le graphe de la
    force en fonction du sinus de l'angle.

    \section{Conclusion}\label{sec:conclusion}

    En conclusion, nous avons pu observer expérimentalement comment s'exerce la force de Laplace sur un fil
    parcouru par un courant électrique, et nous avons pu constater que les résultats obtenus sont en accord
    avec les prédictions théoriques.
    L'expérience pourrait être améliorée, principalement en créant un champ magnétique plus homogène et
    mieux contrôlé, afin d'améliorer la mesure.
    Cela pourrait être fait en utilisant un électroaimant, ce qui améliorerait les deux points précédents.
    De plus, créer un montage permettant de mieux observer la variation de la force selon la longueur du fil
    pourrait être intéressant, comme c'est une des mesures pour laquelle on a le moins de points.
    Malgré tout, cela demeure une expérience intéressante, et qui permet de bien comprendre le phénomène
    de la force de Laplace.

\end{document}