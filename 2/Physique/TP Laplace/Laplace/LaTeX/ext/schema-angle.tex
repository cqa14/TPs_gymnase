% \usepackage[usenames,dvipsnames]{pstricks}
% \usepackage{pstricks-add}
% \usepackage{epsfig}
% \usepackage{pst-grad} % For gradients
% \usepackage{pst-plot} % For axes
% \usepackage[space]{grffile} % For spaces in paths
% \usepackage{etoolbox} % For spaces in paths
% \makeatletter % For spaces in paths
% \patchcmd\Gread@eps{\@inputcheck#1 }{\@inputcheck"#1"\relax}{}{}
% \makeatother
% 
\begin{figure}[H]
\centering
\psscalebox{0.8 0.8} % Change this value to rescale the drawing.
{
\begin{pspicture}(0,-5.98)(9.35,2.4)
\definecolor{colour3}{rgb}{0.8,0.8,0.8}
\definecolor{colour1}{rgb}{0.4,0.4,0.4}
\definecolor{colour5}{rgb}{0.5019608,0.5019608,0.5019608}
\definecolor{colour6}{rgb}{0.8,0.5019608,0.2}
\definecolor{colour4}{rgb}{0.6,0.6,0.6}
\definecolor{colour2}{rgb}{0.2,0.2,0.2}
\psframe[linecolor=black, linewidth=0.04, fillstyle=solid,fillcolor=colour3, dimen=outer](2.28,2.11)(0.32,-1.75)
\psbezier[linecolor=blue, linewidth=0.04](0.82,-1.31)(2.68,-2.13)(4.16,-0.91)(5.56,-1.81)
\psbezier[linecolor=red, linewidth=0.04](1.56,-1.37)(2.78,-0.39)(4.38,-0.79)(5.84,-1.83)
\psframe[linecolor=colour3, linewidth=0.04, fillstyle=solid,fillcolor=colour3, dimen=outer](5.18,-5.59)(1.54,-5.97)
\psframe[linecolor=colour3, linewidth=0.04, fillstyle=solid,fillcolor=colour3, dimen=outer](4.24,2.4)(3.88,-5.63)
\psframe[linecolor=black, linewidth=0.04, fillstyle=solid,fillcolor=black, dimen=outer](4.94,-1.65)(2.96,-2.15)
\psframe[linecolor=colour1, linewidth=0.04, fillstyle=solid,fillcolor=colour1, dimen=outer](6.66,-1.81)(4.94,-2.05)
\psframe[linecolor=black, linewidth=0.04, fillstyle=solid,fillcolor=black, dimen=outer](7.8,-0.67)(6.66,-3.43)
\psframe[linecolor=colour5, linewidth=0.04, fillstyle=solid,fillcolor=colour5, dimen=outer](8.32,-2.49)(6.14,-2.61)
\psrotate(7.235, -4.055){-180.0}{\pstriangle[linecolor=black, linewidth=0.04, fillstyle=solid,fillcolor=black, dimen=outer](7.235,-4.72)(1.19,1.33)}
\psframe[linecolor=black, linewidth=0.04, fillstyle=vlines*,fillcolor=colour6, hatchwidth=0.028222222, hatchangle=0.0, hatchsep=0.0412, dimen=outer](7.38,-4.27)(7.06,-4.91)
\psframe[linecolor=black, linewidth=0.04, fillstyle=solid,fillcolor=black, dimen=outer](8.67,-5.35)(5.84,-5.98)
\psframe[linecolor=black, linewidth=0.04, fillstyle=solid,fillcolor=colour4, dimen=outer](8.16,-5.21)(6.34,-5.37)
\psframe[linecolor=black, linewidth=0.012, fillstyle=solid,fillcolor=red, dimen=outer](6.9,-4.81)(6.64,-5.21)
\psframe[linecolor=black, linewidth=0.012, dimen=outer](7.88,-4.8)(7.64,-5.23)
\psframe[linecolor=black, linewidth=0.04, fillstyle=solid,fillcolor=colour2, dimen=outer](7.89,-4.96)(6.61,-5.22)
\pscircle[linecolor=black, linewidth=0.04, fillstyle=solid,fillcolor=colour4, dimen=outer](1.765,0.325){0.205}
\pscircle[linecolor=black, linewidth=0.04, fillstyle=solid,fillcolor=colour5, dimen=outer](1.745,-0.195){0.205}
\psframe[linecolor=black, linewidth=0.04, fillstyle=solid,fillcolor=black, dimen=outer](1.94,1.69)(0.68,0.83)
\psframe[linecolor=black, linewidth=0.04, fillstyle=gradient, gradlines=2000, dimen=outer](7.68,-5.49)(6.92,-5.85)
\psline[linecolor=black, linewidth=0.04, arrowsize=0.05291667cm 2.0,arrowlength=1.4,arrowinset=0.0]{->}(0.96,-2.43)(1.12,-1.79)
\psline[linecolor=black, linewidth=0.04, arrowsize=0.05291667cm 2.0,arrowlength=1.4,arrowinset=0.0]{->}(2.4,-4.77)(3.68,-5.11)
\psline[linecolor=black, linewidth=0.04, arrowsize=0.05291667cm 2.0,arrowlength=1.4,arrowinset=0.0]{->}(2.74,-3.01)(4.8,-2.23)
\rput[bl](0.0,-2.69){Générateur}
\rput[bl](1.56,-4.73){Statif}
\rput[bl](1.92,-3.41){Support}
\psline[linecolor=black, linewidth=0.04, arrowsize=0.05291667cm 2.0,arrowlength=1.4,arrowinset=0.0]{->}(5.36,-4.81)(5.8,-5.29)
\psline[linecolor=black, linewidth=0.04, arrowsize=0.05291667cm 2.0,arrowlength=1.4,arrowinset=0.0]{->}(5.7,-3.59)(6.58,-4.81)
\rput[bl](4.42,-4.75){Balance}
\rput[bl](4.68,-3.53){Aimant}
\psline[linecolor=black, linewidth=0.04, arrowsize=0.05291667cm 2.0,arrowlength=1.4,arrowinset=0.0]{->}(5.66,0.19)(6.58,-1.15)
\rput[bl](4.38,0.17){Bobine avec roue graduée}
\end{pspicture}
}
\caption{Schéma du montage pour faire varier l'angle}
\label{fig:schema-angle}
\end{figure}

