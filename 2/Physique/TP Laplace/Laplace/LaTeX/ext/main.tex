%! Author = cqa
%! Date = 02.05.23

\documentclass[11pt]{article}
% Basic Packages for Encoding (Input AND Output) and Langauge Support
\usepackage[utf8]{inputenc}
\usepackage[T1]{fontenc}
\usepackage[french]{babel}

% Change Layout with a User-Friendly Interface
\usepackage[margin=1in]{geometry}

% Include Pictures with a User-Friendly Interface
\usepackage{graphicx}
\usepackage{lmodern}
\usepackage{float}

% Extended Math Support from the Famous 'American Mathematical Society'
\usepackage{amsmath}
\usepackage{amsfonts}
\usepackage{amssymb}

% Just for Demonstration Purposes
\usepackage[math]{blindtext}

% For the chemistry
\usepackage{chemist}

% For use on computer
\usepackage{hyperref}

% For table color
\usepackage{xcolor,colortbl}

% Tableau verticale
\usepackage{rotating}

% Figure dans figure
\usepackage{subfig}

% Multicolumn
\usepackage{multirow}

% Preuves
\usepackage{amsthm}

% Footnote
\usepackage[bottom]{footmisc}

% PsTricks
\usepackage[usenames,dvipsnames]{pstricks}
\usepackage{pstricks-add}
\usepackage{epsfig}
\usepackage{pst-grad} % For gradients
\usepackage{pst-plot} % For axes
\usepackage[space]{grffile} % For spaces in paths
\usepackage{etoolbox} % For spaces in paths
\makeatletter % For spaces in paths
\patchcmd\Gread@eps{\@inputcheck#1 }{\@inputcheck"#1"\relax}{}{}
\makeatother

\usepackage{auto-pst-pdf}

% Graphiques
\usepackage{pgfplots}
\DeclareUnicodeCharacter{2212}{−}
\usepgfplotslibrary{groupplots,dateplot}
\usetikzlibrary{patterns,shapes.arrows}
\pgfplotsset{compat=newest}

%\usepgfplotslibrary{external}
%\tikzexternalize

% Titre
\usepackage[affil-it]{authblk}
\usepackage{textcomp}
\usepackage{wasysym}
\title{\textbf{TP Force de Laplace}}
\author{Camille Yerly, Romain Blondel}
\affil{2M8, Gymnase Auguste Piccard}

% Document
\begin{document}
    \maketitle

    \section{But}\label{sec:but}


    \section{Introduction}\label{sec:introduction}


    \section{Principe de mesure et description}\label{sec:principe-de-mesure-et-description}

    \subsection{Matériel}\label{subsec:materiel}

    \subsection{Déroulement}\label{subsec:deroulement}

    \subsection{Schémas}\label{subsec:schemas}


    \section{Résultats et calculs}\label{sec:resultats-et-calculs}
    
    \begin{figure}[H]
        \centering
        % This file was created with tikzplotlib v0.10.1.
\begin{tikzpicture}

\definecolor{darkgray178}{RGB}{178,178,178}
\definecolor{firebrick166640}{RGB}{166,6,40}
\definecolor{lightgray204}{RGB}{204,204,204}
\definecolor{silver188}{RGB}{188,188,188}
\definecolor{slategray122104166}{RGB}{122,104,166}
\definecolor{steelblue52138189}{RGB}{52,138,189}
\definecolor{whitesmoke238}{RGB}{238,238,238}

\begin{axis}[
axis background/.style={fill=whitesmoke238},
axis line style={silver188},
height=9cm,
legend cell align={left},
legend style={
  fill opacity=0.8,
  draw opacity=1,
  text opacity=1,
  at={(0.03,0.97)},
  anchor=north west,
  draw=lightgray204,
  fill=whitesmoke238
},
tick pos=left,
width=16cm,
x grid style={darkgray178},
xlabel={Courant \(\displaystyle [A]\)},
xmajorgrids,
xmin=-5, xmax=5,
xtick style={color=black},
y grid style={darkgray178},
ylabel={Force \(\displaystyle [N]\)},
ymajorgrids,
ymin=-0.0356034202597424, ymax=0.03546582545459,
ytick style={color=black}
]
\addplot [thick, steelblue52138189, mark=x, mark size=3, mark options={solid}, only marks]
table {%
-5 -0.0323729515075684
-4.5 -0.0291357040405273
-4 -0.0259964466094971
-3.5 -0.0228573083877563
-3 -0.0196199417114258
-2.5 -0.0163826942443848
-2 -0.0131454467773438
-1.5 -0.00990808010101318
-1 -0.00647461414337158
-0.5 -0.00304114818572998
0 0
0.5 0.00313925743103027
1 0.00637650489807129
1.5 0.0096137523651123
2 0.0127530097961426
2.5 0.0159902572631836
3 0.0191295146942139
3.5 0.0224648714065552
4 0.0255060195922852
4.5 0.0287432670593262
5 0.0319806337356567
};
\addlegendentry{Mesures}
\addplot [thick, firebrick166640]
table {%
-5 -0.0322353839874268
5 0.0322353839874268
};
\addlegendentry{$F = 0.006 \cdot I$}
\addplot [thick, slategray122104166, dashed]
table {%
-5 -0.0319806337356567
5 0.0319806337356567
};
\addlegendentry{$F = I \cdot L \cdot B$}
\end{axis}

\end{tikzpicture}

        \caption{Mesure de la force de Laplace en fonction de l'intensité du courant}
        \label{fig:courant}
    \end{figure}

    \begin{figure}[H]
        \centering
        \subfloat[Courant croissant]{
            % This file was created with tikzplotlib v0.10.1.
\begin{tikzpicture}

\definecolor{darkgray178}{RGB}{178,178,178}
\definecolor{firebrick166640}{RGB}{166,6,40}
\definecolor{lightgray204}{RGB}{204,204,204}
\definecolor{silver188}{RGB}{188,188,188}
\definecolor{slategray122104166}{RGB}{122,104,166}
\definecolor{steelblue52138189}{RGB}{52,138,189}
\definecolor{whitesmoke238}{RGB}{238,238,238}

\begin{axis}[
axis background/.style={fill=whitesmoke238},
axis line style={silver188},
height=8cm,
legend cell align={left},
legend style={
  fill opacity=0.8,
  draw opacity=1,
  text opacity=1,
  at={(0.03,0.97)},
  anchor=north west,
  draw=lightgray204,
  fill=whitesmoke238
},
tick pos=left,
width=8cm,
x grid style={darkgray178},
xlabel={Courant \(\displaystyle [A]\)},
xmajorgrids,
xmin=-0.5, xmax=0.5,
xtick style={color=black},
y grid style={darkgray178},
ylabel={Force \(\displaystyle [N]\)},
ymajorgrids,
ymin=-0.003517866, ymax=0.003517866,
ytick style={color=black}
]
\addplot [thick, steelblue52138189, mark=x, mark size=3, mark options={solid}, only marks]
table {%
-0.5 -0.00304114818572998
-0.450000047683716 -0.00284492969512939
-0.399999976158142 -0.00255060195922852
-0.350000023841858 -0.00235438346862793
-0.299999952316284 -0.00206005573272705
-0.25 -0.00156962871551514
-0.200000047683716 -0.00117719173431396
-0.149999976158142 -0.000980973243713379
-0.100000023841858 -0.000588655471801758
-0.0499999523162842 -0.000196218490600586
0 0
0.0499999523162842 0.000294327735900879
0.100000023841858 0.000588655471801758
0.149999976158142 0.000882863998413086
0.200000047683716 0.00117719173431396
0.25 0.00156962871551514
0.299999952316284 0.00196194648742676
0.350000023841858 0.00215816497802734
0.399999976158142 0.00255060195922852
0.450000047683716 0.00284492969512939
0.5 0.00313925743103027
};
\addlegendentry{Mesures}
\addplot [thick, firebrick166640]
table {%
-0.5 -0.00315701961517334
0.5 0.00315701961517334
};
\addlegendentry{$F = 0.006 \cdot I$}
\addplot [thick, slategray122104166, dashed]
table {%
-0.5 -0.00319802761077881
0.5 0.00319802761077881
};
\addlegendentry{$F = I \cdot L \cdot B$}
\end{axis}

\end{tikzpicture}

        }
        \subfloat[Courant décroissant]{
            % This file was created with tikzplotlib v0.10.1.
\begin{tikzpicture}

\definecolor{darkgray178}{RGB}{178,178,178}
\definecolor{firebrick166640}{RGB}{166,6,40}
\definecolor{lightgray204}{RGB}{204,204,204}
\definecolor{silver188}{RGB}{188,188,188}
\definecolor{slategray122104166}{RGB}{122,104,166}
\definecolor{steelblue52138189}{RGB}{52,138,189}
\definecolor{whitesmoke238}{RGB}{238,238,238}

\begin{axis}[
axis background/.style={fill=whitesmoke238},
axis line style={silver188},
height=8cm,
legend cell align={left},
legend style={
  fill opacity=0.8,
  draw opacity=1,
  text opacity=1,
  at={(0.03,0.97)},
  anchor=north west,
  draw=lightgray204,
  fill=whitesmoke238
},
tick pos=left,
width=8cm,
x grid style={darkgray178},
xlabel={Courant \(\displaystyle [A]\)},
xmajorgrids,
xmin=-5, xmax=5,
xtick style={color=black},
y grid style={darkgray178},
ylabel={Force \(\displaystyle [N]\)},
ymajorgrids,
ymin=-0.0356034202597424, ymax=0.03546582545459,
ytick style={color=black}
]
\addplot [thick, steelblue52138189, mark=x, mark size=3, mark options={solid}, only marks]
table {%
-5 -0.0323729515075684
-4.5 -0.0291357040405273
-4 -0.0259964466094971
-3.5 -0.0228573083877563
-3 -0.0196199417114258
-2.5 -0.0163826942443848
-2 -0.0131454467773438
-1.5 -0.00990808010101318
-1 -0.00647461414337158
-0.5 -0.00304114818572998
0 0
1 0.00637650489807129
1.5 0.0096137523651123
2 0.0127530097961426
2.5 0.0159902572631836
3 0.0191295146942139
3.5 0.0224648714065552
4 0.0255060195922852
4.5 0.0287432670593262
5 0.0319806337356567
-0.450000047683716 -0.00284492969512939
-0.399999976158142 -0.00255060195922852
-0.350000023841858 -0.00235438346862793
-0.299999952316284 -0.00206005573272705
-0.25 -0.00156962871551514
-0.200000047683716 -0.00117719173431396
-0.149999976158142 -0.000980973243713379
-0.100000023841858 -0.000588655471801758
-0.0499999523162842 -0.000196218490600586
0.0499999523162842 0.000294327735900879
0.100000023841858 0.000588655471801758
0.149999976158142 0.000882863998413086
0.200000047683716 0.00117719173431396
0.25 0.00156962871551514
0.299999952316284 0.00196194648742676
0.350000023841858 0.00215816497802734
0.399999976158142 0.00255060195922852
0.450000047683716 0.00284492969512939
0.5 0.00313925743103027
};
\addlegendentry{Mesures}
\addplot [thick, firebrick166640]
table {%
-5 -0.0322353839874268
5 0.0322353839874268
};
\addlegendentry{$F = 0.006 \cdot I$}
\addplot [thick, slategray122104166, dashed]
table {%
-5 -0.0319806337356567
5 0.0319806337356567
};
\addlegendentry{$F = I \cdot L \cdot B$}
\end{axis}

\end{tikzpicture}

        }
        \caption{Mesure de la force de Laplace en fonction de l'intensité du courant}
        \label{fig:courant-plus}
    \end{figure}

    \section{Discussion des résultats}\label{sec:discussion-des-resultats}


    \section{Conclusion}\label{sec:conclusion}

\end{document}