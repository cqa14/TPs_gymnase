%! Author = cqa
%! Date = 02.05.23

\documentclass[11pt]{article}
% Basic Packages for Encoding (Input AND Output) and Langauge Support
\usepackage[utf8]{inputenc}
\usepackage[T1]{fontenc}
\usepackage[french]{babel}

% Change Layout with a User-Friendly Interface
\usepackage[margin=1in]{geometry}

% Include Pictures with a User-Friendly Interface
\usepackage{graphicx}
\usepackage{lmodern}
\usepackage{float}

% Extended Math Support from the Famous 'American Mathematical Society'
\usepackage{amsmath}
\usepackage{amsfonts}
\usepackage{amssymb}

% Just for Demonstration Purposes
\usepackage[math]{blindtext}

% For the chemistry
\usepackage{chemist}

% For use on computer
\usepackage{hyperref}

% For table color
\usepackage{xcolor,colortbl}

% Tableau verticale
\usepackage{rotating}

% Figure dans figure
\usepackage{subfig}

% Multicolumn
\usepackage{multirow}

% Preuves
\usepackage{amsthm}

% Footnote
\usepackage[bottom]{footmisc}

% PsTricks
\usepackage[usenames,dvipsnames]{pstricks}
\usepackage{pstricks-add}
\usepackage{epsfig}
\usepackage{pst-grad} % For gradients
\usepackage{pst-plot} % For axes
\usepackage[space]{grffile} % For spaces in paths
\usepackage{etoolbox} % For spaces in paths
\makeatletter % For spaces in paths
\patchcmd\Gread@eps{\@inputcheck#1 }{\@inputcheck"#1"\relax}{}{}
\makeatother

\usepackage{auto-pst-pdf}

% Graphiques
\usepackage{pgfplots}
\DeclareUnicodeCharacter{2212}{−}
\usepgfplotslibrary{groupplots,dateplot}
\usetikzlibrary{patterns,shapes.arrows}
\pgfplotsset{compat=newest}

%\usepgfplotslibrary{external}
%\tikzexternalize

% Titre
\usepackage[affil-it]{authblk}
\usepackage{textcomp}
\usepackage{wasysym}
\title{\textbf{TP Acidité du Coca-Cola}}
\author{Manon Bruno, Romain Blondel}
\affil{2M8, Gymnase Auguste Piccard}

% Document
\begin{document}

\maketitle

\section{But}\label{sec:but}
Mesurer le taux d'acidité du Coca-Cola dû à l'acide phosphorique présent dedans au moyen du $pH$\@.

\section{Introduction}\label{sec:introduction}

Le Coca-Cola contient de l'acide carbonique que nous faisons évaporer et qui a donc peu d'influence dans
nos mesures et de l'acide phosphorique.
C'est un triacide de composition \chemform{H_3PO_4} ce qui veut dire qu'il est capable de céder tour à tour 3
protons \chemform{H^+}.

\begin{chemmath}
    H_3PO_4 + H_2O \rightleftharpoons H_3O^+ + H_2PO_4^-
\end{chemmath}
\begin{chemmath}
    H_2PO_4^- + H_2O \rightleftharpoons H_3O^+ + HPO_4^{2-}
\end{chemmath}
\begin{chemmath}
    HPO_4^{2-} + H_2O \rightleftharpoons H_3O^+ + PO_4^{3-}
\end{chemmath}

Voici les 3 réactions de décomposition de l'acide phosphorique dans l'eau.
On constate le caractère amphotère de l'acide phosphorique, c'est-à-dire qu'il peut se comporter comme un
acide ou une base, lors de sa décomposition en \chemform{H_2PO_4^-} et \chemform{HPO_4^{2-}}.
Elles peuvent être résumées par la réaction suivante :

\begin{chemmath}
    H_3PO_4 + 3H_2O \rightleftharpoons 3H_3O^+ + PO_4^{3-}
\end{chemmath}

Bien qu'il soit utilisé dans différents sodas, nous pouvons aussi le retrouver dans des engrais ou des détergents.
Il faut aussi noter que sa grande consommation peut augmenter les risques de calculs rénaux
(et plus généralement causer des problèmes aux reins).
Ce risque se retrouve donc lorsqu'une trop grande consommation de Coca-Cola a lieu au quotidien aux côtés
des risques de diabètes et d'obésité ce qui ne stoppe cependant pas cette boisson d'être très consommée
dans le monde.\\

Le taux d'acidité se mesure grâce au $pH$ (potentiel hydrogène).
C'est une échelle adimensionnelle qui va de 0 à 14.
Un $pH$ de 0 représente un acide fort, un de 7 une solution neutre telle que l'eau et un $pH$ de 14 représente
 une base forte.
Le Coca en possédant un d'environ 2, il est considéré comme acide.\\
Le $pH$ est la mesure de la concentration d'ion \chemform{H^+} dans une solution (dans les équations qui suivent, $C$
désigne la concentration).
On l'obtient grâce à la formule $pH= - \log(C_{H^+})$ et, pour un acide fort, cela correspond à
$pH= - \log(C_{HA})$, car sa réaction de décomposition peut être considéré comme à sens unique
\chemform{HA \rightarrow H^+ + A^-}, avec \chemform{HA} un acide.\\
Or quand on a un acide faible, comme l'acide phosphorique dans notre cas, la réaction de décomposition est
à double sens, c'est-à-dire que l'acide peut se recomposer à partir de ses ions
\chemform{HA \rightleftharpoons H^+ + A^-}.
Alors on obtient une nouvelle formule pour calculer le pH d'un acide faible $pH = \frac{1}{2} (pK_a - \log(C_{HA}))$.
Le $pK_a$ se calcule comme suit : $pK_a = - \log(K_a)$ avec $K_a = \frac{C_{H^+}C_{A^-}}{C_{HA}}$ la constante de la
vitesse de réaction de décomposition de l'acide.\\
Pour un polyacide on pourrait avoir une formule plus complexe, mais dans notre cas, nous pouvons négliger
les autres réactions de décomposition de l'acide phosphorique et donc le considérer comme un monoacide faible.\\
Il faut aussi noter qu'il existe de nombreuses façons de mesurer le $pH$\@.
Ici, nous avons fait appel à un pH-mètre, mais des indicateurs colorés sous forme de ruban ou du jus de
choux-rouge sont d'autres moyens de l'obtenir.

\section{Principe de mesure et description}\label{sec:principe-de-mesure-et-description}

Afin d'avoir les résultats les plus précis possibles, il est nécessaire d'enlever le gaz carbonique présent
dans le Coca-Cola.
On a donc commencé par le chauffer en le remuant, assez chaud pour que le gaz s'échappe, mais pas trop pour
éviter que le volume de Coca ne change trop à cause de l'évaporation.
Ensuite, on a mesuré le $pH$ du Coca-Cola avec le pH-mètre, puis reporté son évolution au fur et à mesure
de l'ajout de la solution titrante de NaOH\@.
Finalement, on exploite ces données en les reportant sur un graphique.

\subsection{Matériel}\label{subsec:materiel}
\begin{itemize}
\item $25 \ [mL]$ de Coca-cola
\item Pipette de $25 \ [ml]$
\item Bécher à large col de $100 \ [ml]$
\item Barreau magnétique
\item Agitateur magnétique
\item Burette de $50 \ [mL]$
\item Solution de NaOH à $0.020 \ [M]$
\item pH-mètre
\item Entonnoir
\item Bécher poubelle
\end{itemize}

\subsection{Montage}\label{subsec:montage}
\begin{figure}[H]
    \centering
    \psscalebox{1 1} % Change this value to rescale the drawing.
    {
        \begin{pspicture}(0,-12.514695)(15.79,0.53591245)
            \definecolor{colour7}{rgb}{0.8,0.8,0.8}
            \definecolor{colour11}{rgb}{0.9019608,0.9019608,0.9019608}
            \definecolor{colour0}{rgb}{0.4,0.4,0.4}
            \definecolor{colour1}{rgb}{0.8,1.0,1.0}
            \definecolor{colour2}{rgb}{0.5019608,0.5019608,0.5019608}
            \definecolor{colour3}{rgb}{0.2,0.2,0.2}
            \definecolor{colour4}{rgb}{1.0,0.4,0.4}
            \definecolor{colour5}{rgb}{0.7019608,0.7019608,0.7019608}
            \definecolor{colour6}{rgb}{0.6,0.3019608,0.0}
            \definecolor{colour8}{rgb}{0.5019608,0.6,1.0}
            \definecolor{colour9}{rgb}{0.8,0.9019608,1.0}
            \definecolor{colour10}{rgb}{0.0,0.101960786,0.5019608}
            \psrotate(8.06, -6.704696){-80.12114}{\psframe[linecolor=black, linewidth=0.04, fillstyle=solid,fillcolor=black, dimen=outer](10.96,-6.3546963)(5.16,-7.054696)}
            \psframe[linecolor=colour7, linewidth=0.04, fillstyle=solid,fillcolor=colour11, dimen=outer](8.52,-4.694696)(3.98,-5.554696)
            \psframe[linecolor=colour7, linewidth=0.04, fillstyle=solid,fillcolor=colour11, dimen=outer](5.18,-3.594696)(4.24,-10.054696)
            \psbezier[linecolor=black, linewidth=0.16](3.48,-9.574696)(1.08,-4.254696)(4.56,-0.65469605)(7.6,-3.974696044921875)
            \pspolygon[linecolor=black, linewidth=0.04, fillstyle=solid](8.88,0.505304)(9.55,0.505304)(9.55,-3.9946961)(9.306364,-7.234696)(9.154091,-7.234696)(8.88,-3.9946961)
            \pspolygon[linecolor=black, linewidth=0.04, fillstyle=solid,fillcolor=colour1](9.04,-5.694696)(9.46,-5.694696)(9.56,-4.134696)(9.54,-0.15469605)(8.9,-0.15469605)(8.86,-3.934696)
            \pspolygon[linecolor=colour0, linewidth=0.04, fillstyle=solid,fillcolor=colour0](8.76,-5.6944256)(8.78,-6.108615)(9.68,-6.131588)(9.7,-6.574696)(9.88,-6.576858)(9.86,-5.144696)(9.72,-5.1521287)(9.68,-5.703345)
            \rput(9.38,-5.174696){\psaxes[linecolor=black, linewidth=0.04, tickstyle=full, axesstyle=axes, labels=none, ticks=y, ticksize=0.1432cm, dx=1.0cm, dy=1.0cm, Dy=10.0](0,0)(0,0)(0,5)}
            \rput[bl](9.02,-0.27469605){0}
            \rput[bl](8.96,-5.174696){50}
            \psline[linecolor=black, linewidth=0.04, fillstyle=solid, arrowsize=0.05291667cm 2.0,arrowlength=1.4,arrowinset=0.0]{->}(8.1,-1.6946961)(9.06,-2.594696)
            \rput[bl](6.02,-1.654696){NaOH $0.020$ $[M]$}
            \psframe[linecolor=colour2, linewidth=0.04, fillstyle=solid,fillcolor=colour2, dimen=outer](10.91,-10.994696)(6.59,-11.374696)
            \psframe[linecolor=colour3, linewidth=0.04, fillstyle=solid,fillcolor=colour3, dimen=outer](11.1,-11.294696)(6.32,-12.514696)
            \psframe[linecolor=colour0, linewidth=0.04, fillstyle=solid,fillcolor=colour0, dimen=outer](9.42,-11.494696)(8.86,-11.774696)
            \pscircle[linecolor=colour4, linewidth=0.04, fillstyle=solid,fillcolor=colour4, dimen=outer](9.24,-11.6146965){0.06}
            \pscircle[linecolor=colour5, linewidth=0.04, fillstyle=solid,fillcolor=colour5, dimen=outer](10.38,-11.914696){0.38}
            \psline[linecolor=black, linewidth=0.04, fillstyle=solid, arrowsize=0.05291667cm 2.0,arrowlength=1.4,arrowinset=0.0]{->}(12.2,-11.634696)(11.12,-11.694696)
            \rput[bl](12.38,-11.674696){Agitateur magnétique}
            \pscircle[linecolor=colour5, linewidth=0.04, fillstyle=solid,fillcolor=colour5, dimen=outer](7.92,-11.954696){0.38}
            \psframe[linecolor=colour0, linewidth=0.04, fillstyle=solid,fillcolor=colour0, dimen=outer](7.32,-11.534696)(6.76,-11.814696)
            \pscircle[linecolor=colour4, linewidth=0.04, fillstyle=solid,fillcolor=colour4, dimen=outer](7.16,-11.674696){0.06}
            \rput[bl](12.4,-12.014696){et plaque chauffante}
            \psframe[linecolor=black, linewidth=0.04, dimen=outer](9.86,-7.274696)(7.48,-10.974696)
            \psframe[linecolor=black, linewidth=0.04, fillstyle=solid,fillcolor=colour6, dimen=outer](9.88,-9.354696)(7.5,-10.974696)
            \psellipse[linecolor=black, linewidth=0.04, fillstyle=gradient, gradlines=2000, gradbegin=colour0, gradend=colour5, dimen=outer](8.62,-10.834696)(0.56,0.12)
            \psframe[linecolor=black, linewidth=0.04, fillstyle=solid,fillcolor=colour7, dimen=outer](5.54,-9.254696)(0.0,-12.494696)
            \psframe[linecolor=black, linewidth=0.04, fillstyle=gradient, gradlines=2000, gradbegin=colour8, gradend=colour9, dimen=outer](3.88,-9.794696)(2.0,-10.874696)
            \psframe[linecolor=black, linewidth=0.04, fillstyle=solid,fillcolor=colour10, dimen=outer](2.66,-10.934696)(2.04,-11.574696)
            \psframe[linecolor=black, linewidth=0.04, fillstyle=solid,fillcolor=colour10, dimen=outer](3.84,-10.934696)(3.2,-11.574696)
            \psframe[linecolor=black, linewidth=0.04, fillstyle=solid,fillcolor=colour10, dimen=outer](2.66,-11.694696)(2.04,-12.334696)
            \psframe[linecolor=black, linewidth=0.04, fillstyle=solid,fillcolor=colour10, dimen=outer](3.84,-11.694696)(3.2,-12.334696)
            \psline[linecolor=black, linewidth=0.04, arrowsize=0.05291667cm 2.0,arrowlength=1.4,arrowinset=0.0]{->}(11.44,-2.854696)(9.68,-4.174696)
            \psline[linecolor=black, linewidth=0.04, arrowsize=0.05291667cm 2.0,arrowlength=1.4,arrowinset=0.0]{->}(1.34,-6.974696)(2.06,-9.074696)
            \psline[linecolor=black, linewidth=0.04, arrowsize=0.05291667cm 2.0,arrowlength=1.4,arrowinset=0.0]{->}(11.76,-5.934696)(9.98,-7.894696)
            \psline[linecolor=black, linewidth=0.04, arrowsize=0.05291667cm 2.0,arrowlength=1.4,arrowinset=0.0]{->}(10.78,-8.134696)(9.22,-9.914696)
            \psline[linecolor=black, linewidth=0.04, arrowsize=0.05291667cm 2.0,arrowlength=1.4,arrowinset=0.0]{->}(10.52,-10.634696)(9.22,-10.834696)
            \rput[bl](11.48,-2.914696){Burette}
            \rput[bl](10.78,-8.274696){Coca-Cola degazé}
            \rput[bl](11.78,-6.014696){Bécher}
            \rput[bl](0.52,-6.934696){pH-mètre}
            \rput[bl](10.5,-10.754696){Barre d'agitation}
        \end{pspicture}
    }
    \caption{Schéma de l'expérience}\label{fig:schema}
\end{figure}

\subsection{Mode opératoire}\label{subsec:mode-operatoire}
\begin{enumerate}
\item Récupérer $25 \ [mL]$ de Coca-cola à l'aide de la pipette et les verser dans le bécher de $100 \ [mL]$.
\item Mettre le barreau magnétique dans le bécher, placer ce dernier sur l'agitateur magnétique.
Faire chauffer le Coca-cola pendant 5 minutes à environ $80 \ [^\circ C]$.
\item À l'aide de l'entonnoir, remplir la burette de quelques millilitres de plus que la limite de 50 puis vider
 le surplus dans le bécher poubelle.
\item Brancher le pH-mètre, allumer le et placer la sonde dans le bécher de Coca-cola en veillant à ce qu'elle
 ne gêne pas la rotation du barreau magnétique, ensuite lancer l'agitateur en veillant à ce que le Coca-Cola
 n'éclabousse pas.
\item Placer l'ouverture de la burette sur le goutte à goutte puis noter le $pH$ tous les $0.5 \ [mL]$ jusqu'à
 atteindre un $pH$ de 11 ou ne plus avoir de liquide dans la burette.
\end{enumerate}

\section{Résultats}\label{sec:resultats}
Ci-dessous sont consignés les résultats obtenus lors de l'expérience.
Parmi les 100 mesures effectuées, certaines sont un peu faussées par des erreurs de manipulation.
En effet, à la mesure pour $41 \ [mL]$, il a été question d'arrêter les mesures avant de les relancer pour
les dernières.
Relancer la mesure nous a empêché de mesurer le $pH$ pour cette valeur et les quelques résultats suivants
semblent avoir été faussés (voir Figure~\ref{fig:data-plot}).

\input{ext/table-result}

Les mesures ont donc été mise en forme dans le graphique qui suit.
Les points de mesures ont été interpolés de manière linéaire afin de facilité la lecture du graphique et le
traitement des données.
De plus, on y a ajouté le tracé théorique avec les données théoriques.
On y a mis les valeurs initiales de l'expérience ainsi que la concentration d'acide phosphorique dans le
Coca-Cola\footnote{\href{https://fr.coca-cola.be/developpement-durable/boissons/ingredients/decouvrir-les-ingredients-de-nos-boissons}{Site de \textit{Coca-Cola}}}.
Il y est dit que le Coca-Cola contient $190 \ [mg]$ d'acide phosphorique pour $330 \ [mL]$ de Coca-Cola.
De là, on obtient sa concentration $\frac{190 \ [mg]}{330 \ [mL]} = 0.\overline{57} \ [g/L]$, soit environ
$0.006 \ [M]$ si l'on prend le rapport de $98 \ [g]$ de \chemform{H_3PO_4} correspond à $1 \ [mol]$\footnote{\href{https://www.laboratorynotes.com/molarity-of-85-percent-phosphoric-acid-h3po4/}{Poids de la molécule de \chemform{H_3PO_4} issu d'une référence externe}}.

\begin{figure}[H]
\centering
% This file was created with tikzplotlib v0.10.1.
\begin{tikzpicture}

  \definecolor{darkgray176}{RGB}{176,176,176}
  \definecolor{green}{RGB}{0,128,0}
  \definecolor{lightgray204}{RGB}{204,204,204}
  \definecolor{steelblue31119180}{RGB}{31,119,180}

  \begin{axis}[
  height=8cm,
  legend cell align={left},
  legend style={
    fill opacity=0.8,
    draw opacity=1,
    text opacity=1,
    at={(0.97,0.03)},
    anchor=south east,
    draw=lightgray204
  },
  tick align=outside,
  tick pos=left,
  width=15cm,
  x grid style={darkgray176},
  xlabel={Volume de titrant $[mL]$},
  xmajorgrids,
  xmin=0, xmax=50,
  xtick style={color=black},
  y grid style={darkgray176},
  ylabel={$pH$},
  ymajorgrids,
  ymin=1.10814181448934, ymax=12.360634064755,
  ytick style={color=black}
  ]
  \addplot [draw=steelblue31119180, fill=steelblue31119180, mark=x, only marks]
  table{%
    x  y
    0 2.42
    0.5 2.47
    1 2.52
    1.5 2.57
    2 2.63
    2.5 2.7
    3 2.77
    3.5 2.85
    4 2.94
    4.5 3.05
    5 3.19
    5.5 3.32
    6 3.51
    6.5 3.91
    7 4.5
    7.5 5.35
    8 5.87
    8.5 6.13
    9 6.33
    9.5 6.51
    10 6.64
    10.5 6.75
    11 6.87
    11.5 6.98
    12 7.09
    12.5 7.2
    13 7.33
    13.5 7.44
    14 7.59
    14.5 7.76
    15 7.97
    15.5 8.2
    16 8.42
    16.5 8.65
    17 8.82
    17.5 8.97
    18 9.11
    18.5 9.26
    19 9.35
    19.5 9.4
    20 9.49
    20.5 9.56
    21 9.62
    21.5 9.65
    22 9.7
    22.5 9.76
    23 9.8
    23.5 9.83
    24 9.86
    24.5 9.9
    25 9.94
    25.5 9.98
    26 10
    26.5 10.02
    27 10.05
    27.5 10.08
    28 10.1
    28.5 10.12
    29 10.14
    29.5 10.17
    30 10.18
    30.5 10.21
    31 10.23
    31.5 10.24
    32 10.26
    32.5 10.28
    33 10.3
    33.5 10.33
    34 10.33
    34.5 10.35
    35 10.35
    35.5 10.37
    36 10.39
    36.5 10.4
    37 10.43
    37.5 10.43
    38 10.45
    38.5 10.46
    39 10.47
    39.5 10.48
    40 10.49
    40.5 10.51
    41.5 10.67
    42 10.62
    42.5 10.62
    43 10.63
    43.5 10.64
    44 10.65
    44.5 10.66
    45 10.67
    45.5 10.68
    46 10.69
    46.5 10.69
    47 10.7
    47.5 10.71
    48 10.72
    48.5 10.73
    49 10.74
    49.5 10.75
    50 10.76
  };
  \addlegendentry{Mesures}
  \addplot [semithick, green]
  table {%
    0 2.42000007629395
    1.68400001525879 2.59208011627197
    2.18799996376038 2.65632009506226
    3.18700003623962 2.79992008209229
    3.68899989128113 2.88402009010315
    4.09499979019165 2.96090006828308
    4.56300020217896 3.06764006614685
    5.19500017166138 3.24070000648499
    5.53200006484985 3.33215999603271
    6.00899982452393 3.51719999313354
    6.51200008392334 3.92416000366211
    7.01100015640259 4.51870012283325
    7.51100015640259 5.36144018173218
    8.01000022888184 5.87519979476929
    8.53400039672852 6.14359998703003
    9.10099983215332 6.36636018753052
    9.53899955749512 6.52014017105103
    10.0970001220703 6.66134023666382
    10.6899995803833 6.79559993743896
    11.1940002441406 6.91268014907837
    12.5939998626709 7.22444009780884
    13.0959997177124 7.35111999511719
    13.5480003356934 7.45440006256104
    14.0970001220703 7.62298011779785
    14.548999786377 7.78058004379272
    15.1000003814697 8.01599979400635
    15.701000213623 8.28843975067139
    16.2080001831055 8.51568031311035
    16.5340003967285 8.66156005859375
    17.0979995727539 8.84939956665039
    17.6900005340576 9.02320003509521
    18.1919994354248 9.16759967803955
    18.5319995880127 9.26576042175293
    19.0470008850098 9.35470008850098
    19.5459995269775 9.40828037261963
    20.0949993133545 9.50329971313477
    20.6849994659424 9.582200050354
    21.0629997253418 9.62378025054932
    21.5909996032715 9.65909957885742
    22.1919994354248 9.7230396270752
    22.5919990539551 9.76735973358154
    23.1849994659424 9.81110000610352
    24.1919994354248 9.87535953521729
    25.5900001525879 9.98359966278076
    26.6900005340576 10.0313997268677
    27.6830005645752 10.0873203277588
    29.173999786377 10.1504402160645
    29.5949993133545 10.1718997955322
    30.0939998626709 10.185640335083
    30.6919994354248 10.217679977417
    31.1860008239746 10.2337198257446
    31.6879997253418 10.2475204467773
    33.173999786377 10.3104400634766
    33.5620002746582 10.3299999237061
    34.0929985046387 10.3337202072144
    34.5929985046387 10.3500003814697
    35.0929985046387 10.3537197113037
    36.185001373291 10.3936996459961
    36.5940017700195 10.4056396484375
    37.0610008239746 10.4300003051758
    37.5929985046387 10.4337196350098
    38.1860008239746 10.4537200927734
    40.1910018920898 10.4976396560669
    40.5309982299805 10.5149602890015
    41.5139999389648 10.6686000823975
    42.0369987487793 10.6199998855591
    42.6870002746582 10.623740196228
    46.1839981079102 10.6899995803833
    46.6870002746582 10.6937398910522
    49.9990005493164 10.7599802017212
  };
  \addlegendentry{Interpolation}
  \addplot [semithick, red]
  table {%
    0 2.40376305580139
    0.899999976158142 2.47757005691528
    1.70000004768372 2.54950833320618
    2.40000009536743 2.61889243125916
    3 2.68461728096008
    3.59999990463257 2.75793051719666
    4.09999990463257 2.82666993141174
    4.5 2.88825297355652
    4.90000009536743 2.95753884315491
    5.19999980926514 3.01607871055603
    5.5 3.08201670646667
    5.69999980926514 3.13123416900635
    5.90000009536743 3.18581533432007
    6.09999990463257 3.24719738960266
    6.30000019073486 3.31748127937317
    6.5 3.3999171257019
    6.59999990463257 3.44725155830383
    6.69999980926514 3.49992704391479
    6.80000019073486 3.5593581199646
    6.90000009536743 3.62760281562805
    7 3.70782279968262
    7.09999990463257 3.80523109436035
    7.19999980926514 3.92932343482971
    7.30000019073486 4.10010814666748
    7.40000009536743 4.3698878288269
    7.59999990463257 5.37670755386353
    7.69999980926514 5.65058946609497
    7.80000019073486 5.82565355300903
    7.90000009536743 5.95412635803223
    8 6.05600452423096
    8.10000038146973 6.14078187942505
    8.19999980926514 6.21367311477661
    8.30000019073486 6.27784156799316
    8.39999961853027 6.33534717559814
    8.5 6.38760805130005
    8.69999980926514 6.48019504547119
    8.89999961853027 6.56105041503906
    9.10000038146973 6.63345336914062
    9.30000019073486 6.69953584671021
    9.60000038146973 6.78991508483887
    9.89999961853027 6.87269639968872
    10.3000001907349 6.97508239746094
    10.8000001907349 7.09525394439697
    12.3000001907349 7.44974517822266
    12.6999998092651 7.55407810211182
    13 7.63906049728394
    13.3000001907349 7.732590675354
    13.5 7.80155324935913
    13.6999998092651 7.8777642250061
    13.8999996185303 7.96382188796997
    14 8.01173973083496
    14.1000003814697 8.06384754180908
    14.1999998092651 8.12113857269287
    14.3000001907349 8.1849946975708
    14.3999996185303 8.25740623474121
    14.5 8.34139537811279
    14.6000003814697 8.44184684753418
    14.6999998092651 8.56734752655029
    14.8000001907349 8.73470211029053
    14.8999996185303 8.97988224029541
    15.1000003814697 9.71573448181152
    15.1999998092651 9.95446395874023
    15.3000001907349 10.1150436401367
    15.3999996185303 10.2336673736572
    15.5 10.3271570205688
    15.6000003814697 10.4041004180908
    15.6999998092651 10.4693794250488
    15.8000001907349 10.5260105133057
    15.8999996185303 10.575982093811
    16 10.6206712722778
    16.2000007629395 10.6979169845581
    16.3999996185303 10.7630767822266
    16.6000003814697 10.8193626403809
    16.7999992370605 10.868857383728
    17.1000003814697 10.9333715438843
    17.3999996185303 10.9889726638794
    17.7000007629395 11.0377597808838
    18.1000003814697 11.0946321487427
    18.5 11.1442365646362
    19 11.1983776092529
    19.5 11.2457046508789
    20.1000003814697 11.2955007553101
    20.7999992370605 11.346076965332
    21.6000003814697 11.3962068557739
    22.5 11.4450540542603
    23.5 11.4920797348022
    24.6000003814697 11.5369653701782
    25.8999996185303 11.5828371047974
    27.3999996185303 11.6283006668091
    29.1000003814697 11.6723928451538
    31 11.7145004272461
    33.2000007629395 11.7560119628906
    35.7000007629395 11.7959756851196
    38.5999984741211 11.8350410461426
    41.9000015258789 11.872296333313
    45.7000007629395 11.9080715179443
    50 11.9416761398315
  };
  \addlegendentry{Courbe théorique}
  \end{axis}

\end{tikzpicture}

\caption[Mesure du $pH$ selon le titrant (\chemform{NaOH}) et valeurs théoriques avec du \chemform{OH^-}]{Mesure
du $pH$ selon le titrant (\chemform{NaOH}) et valeurs théoriques\protect\footnotemark avec du \chemform{OH^-}}
\label{fig:data-plot}
\end{figure}
\footnotetext{\href{http://www.cheminfo.org/flavor/gymnase/index.html}{Source : \textit{cheminfo.org}} -
paramètres : "Solution à titrer : \chemform{H_3PO_4}, $0.006 \ [M]$, $25 \ [mL]$" ; "Titrant : \chemform{OH^-},
$0.02 \ [M]$, $50 \ [mL]$"}

De ces données, on peut en déduire les points d'équivalence et de demi-équivalence.
Pour cela, on a utilisé la méthode des tangentes.
On a tracé une tangente de chaque côté des ``sauts'' dans la courbe, avec les deux la même pente, calculée
via la dérivée numérique $f'(x) = \frac{f(x + \Delta x) - f(x)}{\Delta x}$.
De fait, en prenant la moyenne des deux droites, le croisement avec le graphe correspond au point d'équivalence,
qui est au milieu du ``saut'' de $pH$\@.

\begin{figure}[H]
\centering
\subfloat[PE 1\label{fig:tan-pe1}]{% This file was created with tikzplotlib v0.10.1.
\begin{tikzpicture}

  \definecolor{darkgray176}{RGB}{176,176,176}
  \definecolor{green}{RGB}{0,128,0}
  \definecolor{lightgray204}{RGB}{204,204,204}
  \definecolor{steelblue31119180}{RGB}{31,119,180}

  \begin{axis}[
  legend cell align={left},
  legend style={
    fill opacity=0.8,
    draw opacity=1,
    text opacity=1,
    at={(0.03,0.97)},
    anchor=north west,
    draw=lightgray204
  },
  tick align=outside,
  tick pos=left,
  x grid style={darkgray176},
  xlabel={Volume de titrant \(\displaystyle [mL]\)},
  xmajorgrids,
  xmin=0, xmax=20,
  xtick style={color=black},
  y grid style={darkgray176},
  ylabel={$pH$},
  ymajorgrids,
  ymin=1.44650100000023, ymax=11.203479,
  ytick style={color=black}
  ]
  \addplot [semithick, green]
  table {%
    0 2.42000007629395
    1.65900003910065 2.58908009529114
    2.16300010681152 2.65282011032104
    3.16100001335144 2.79575991630554
    3.66499996185303 2.87969994544983
    4.08199977874756 2.95803999900818
    4.55499982833862 3.06539988517761
    5.16800022125244 3.23368000984192
    5.52799987792969 3.33064007759094
    6.00799989700317 3.51640009880066
    6.51000022888184 3.92179989814758
    7.00899982452393 4.51529979705811
    7.50899982452393 5.3593602180481
    8.00899982452393 5.87468004226685
    8.5310001373291 6.14239978790283
    9.08600044250488 6.36096000671387
    9.53400039672852 6.51883983612061
    10.0830001831055 6.65825986862183
    10.6680002212524 6.79031991958618
    11.1669998168945 6.90674018859863
    12.581000328064 7.22105979919434
    13.0830001831055 7.34825992584229
    13.5419998168945 7.45260000228882
    14.085000038147 7.61889982223511
    14.5419998168945 7.77763986587524
    15.0880002975464 8.01047992706299
    15.6759996414185 8.27744007110596
    16.1749992370605 8.5004997253418
    16.5289993286133 8.6598596572876
    17.0879993438721 8.84640026092529
    17.6709995269775 9.0178804397583
    18.1669998168945 9.16009998321533
    18.5279998779297 9.26504039764404
    19.0410003662109 9.35410022735596
    19.5400009155273 9.40719985961914
    20 9.48999977111816
  };
  \addlegendentry{Interpolation}
  \addplot [semithick, red]
  table {%
    0 1.88999998569489
    10 4.48999977111816
  };
  \addlegendentry{Tangentes}
  \addplot [semithick, red, forget plot]
  table {%
    7.99900007247925 6.1297402381897
    17.9990005493164 8.52974033355713
  };
  \addplot [semithick, blue]
  table {%
    5 4.29999017715454
    12.9989995956421 6.29973983764648
  };
  \addlegendentry{Moyenne}
  \addplot [semithick, steelblue31119180, mark=*, mark size=3, mark options={solid}, only marks]
  table {%
    7.21500015258789 4.86549997329712
    7.21500015258789 4.86549997329712
  };
  \addlegendentry{PE}
  \end{axis}

\end{tikzpicture}
}
\subfloat[PE 2\label{fig:tan-pe2}]{% This file was created with tikzplotlib v0.10.1.
\begin{tikzpicture}

\definecolor{darkgray176}{RGB}{176,176,176}
\definecolor{green}{RGB}{0,128,0}
\definecolor{lightgray204}{RGB}{204,204,204}
\definecolor{steelblue31119180}{RGB}{31,119,180}

\begin{axis}[
height=10cm,
legend cell align={left},
legend style={
  fill opacity=0.8,
  draw opacity=1,
  text opacity=1,
  at={(0.03,0.97)},
  anchor=north west,
  draw=lightgray204
},
tick align=outside,
tick pos=left,
width=10cm,
x grid style={darkgray176},
xlabel={Volume de titrant \(\displaystyle [mL]\)},
xmajorgrids,
xmin=10, xmax=25,
xtick style={color=black},
y grid style={darkgray176},
ylabel={pH},
ymajorgrids,
ymin=6, ymax=11,
ytick style={color=black}
]
\addplot [semithick, green]
table {%
10 6.6399998664856
10.5869998931885 6.77088022232056
11.08899974823 6.88957977294922
12.5450000762939 7.21169996261597
13.0410003662109 7.33901977539062
13.5209999084473 7.44630002975464
14.0419998168945 7.6042799949646
14.5220003128052 7.76923990249634
15.0459995269775 7.99115991592407
15.5920000076294 8.24048042297363
16.0830001831055 8.45818042755127
16.5149993896484 8.65509986877441
17.0459995269775 8.83380031585693
17.5919990539551 8.99575996398926
18.0809993743896 9.13430023193359
18.5149993896484 9.26270008087158
19.0230007171631 9.35229969024658
19.5219993591309 9.40396022796631
20.0459995269775 9.49643993377686
20.5799999237061 9.56960010528564
21.0270004272461 9.62162017822266
21.5429992675781 9.65429973602295
22.0860004425049 9.71032047271729
22.5440006256104 9.76352024078369
23.0790004730225 9.8047399520874
24.0890007019043 9.86711978912354
25 9.9399995803833
};
\addlegendentry{Interpolation}
\addplot [semithick, red]
table {%
9.92929267883301 6.63444423675537
17 8.1899995803833
};
\addlegendentry{Tangentes}
\addplot [semithick, red, forget plot]
table {%
14.5 8.5
24.5 10.3000001907349
};
\addplot [semithick, blue]
table {%
12 7.57000017166138
19.5 9.06999969482422
};
\addlegendentry{Moyenne}
\addplot [semithick, steelblue31119180, mark=*, mark size=3, mark options={solid}, only marks]
table {%
15.789999961853 8.32759952545166
15.789999961853 8.32759952545166
};
\addlegendentry{PE}
\end{axis}

\end{tikzpicture}
}
\caption{Recherche de deux des trois points d'équilibre (PE) via la méthode des tangentes}\label{fig:cont-tan}
\end{figure}

En divisant le volume de \chemform{NaOH} utilisé pour atteindre le point d'équivalence par deux, on obtient
le point de demi-équivalence.
Ces points remarquables sont résumés dans la Figure~\ref{fig:sum-pe}.

\begin{figure}[H]
\centering
% This file was created with tikzplotlib v0.10.1.
\begin{tikzpicture}

\definecolor{cyan}{RGB}{0,255,255}
\definecolor{darkgray176}{RGB}{176,176,176}
\definecolor{green}{RGB}{0,128,0}
\definecolor{lightgray204}{RGB}{204,204,204}
\definecolor{orange}{RGB}{255,165,0}
\definecolor{yellow}{RGB}{255,255,0}

\begin{axis}[
height=10cm,
legend cell align={left},
legend style={
  fill opacity=0.8,
  draw opacity=1,
  text opacity=1,
  at={(0.97,0.03)},
  anchor=south east,
  draw=lightgray204
},
tick align=outside,
tick pos=left,
width=15cm,
x grid style={darkgray176},
xlabel={Volume de titrant \(\displaystyle [mL]\)},
xmajorgrids,
xmin=0, xmax=20,
xtick style={color=black},
y grid style={darkgray176},
ylabel={pH},
ymajorgrids,
ymin=0, ymax=11,
ytick style={color=black}
]
\addplot [semithick, green]
table {%
0 2.42000007629395
1.66299998760223 2.58956003189087
2.16700005531311 2.65337991714478
3.16799998283386 2.79688000679016
3.66799998283386 2.88023996353149
4.08400011062622 2.95847988128662
4.55700016021729 3.06595993041992
5.1710000038147 3.23446011543274
5.52799987792969 3.33064007759094
6.00799989700317 3.51640009880066
6.51100015640259 3.92298007011414
7.01000022888184 4.51700019836426
7.50899982452393 5.3593602180481
8.00899982452393 5.87468004226685
8.5310001373291 6.14239978790283
9.08800029754639 6.36168003082275
9.53499984741211 6.51910018920898
10.085000038147 6.65869998931885
10.6719999313354 6.79127979278564
11.1680002212524 6.90696001052856
12.5830001831055 7.22158002853394
13.085000038147 7.34870004653931
13.5419998168945 7.45260000228882
14.0869998931885 7.61957979202271
14.543999671936 7.778480052948
15.08899974823 8.0109395980835
15.6809997558594 8.27964019775391
16.1770000457764 8.50142002105713
16.5300006866455 8.66020011901855
17.0869998931885 8.84609985351562
17.6749992370605 9.01900005340576
18.1669998168945 9.16009998321533
18.5289993286133 9.26521968841553
19.0419998168945 9.35420036315918
19.5410003662109 9.40738010406494
20 9.48999977111816
};
\addlegendentry{Interpolation}
\addplot [semithick, blue, mark=*, mark size=3, mark options={solid}, only marks]
table {%
7.21500015258789 4.86549997329712
7.21500015258789 4.86549997329712
};
\addlegendentry{PE 1}
\addplot [semithick, cyan, mark=*, mark size=3, mark options={solid}, only marks]
table {%
3.60800004005432 2.86944007873535
3.60800004005432 2.86944007873535
};
\addlegendentry{PE/2 1}
\addplot [semithick, orange, mark=*, mark size=3, mark options={solid}, only marks]
table {%
15.789999961853 8.32759952545166
15.789999961853 8.32759952545166
};
\addlegendentry{PE 2}
\addplot [semithick, yellow, mark=*, mark size=3, mark options={solid}, only marks]
table {%
7.89599990844727 5.76183986663818
7.89599990844727 5.76183986663818
};
\addlegendentry{PE/2 2}
\draw (axis cs:7.215,0) node[
  scale=0.5,
  anchor=south,
  text=black,
  rotate=0.0
]{7.22};
\draw (axis cs:0,4.8655) node[
  scale=0.5,
  anchor=west,
  text=black,
  rotate=0.0
]{4.87};
\draw (axis cs:3.608,0) node[
  scale=0.5,
  anchor=south,
  text=black,
  rotate=0.0
]{3.61};
\draw (axis cs:0,2.86944) node[
  scale=0.5,
  anchor=west,
  text=black,
  rotate=0.0
]{2.87};
\draw (axis cs:15.79,0) node[
  scale=0.5,
  anchor=south,
  text=black,
  rotate=0.0
]{15.79};
\draw (axis cs:0,8.3276) node[
  scale=0.5,
  anchor=west,
  text=black,
  rotate=0.0
]{8.33};
\draw (axis cs:7.896,0) node[
  scale=0.5,
  anchor=south,
  text=black,
  rotate=0.0
]{7.9};
\draw (axis cs:0,5.76184) node[
  scale=0.5,
  anchor=west,
  text=black,
  rotate=0.0
]{5.76};
\end{axis}

\end{tikzpicture}

\caption{Points d'équivalence (PE) et de demi-équivalence (PE/2)}\label{fig:sum-pe}
\end{figure}

La table~\ref{tab:resum-ph-pka} résume les divers points trouvés, ainsi que les valeurs issues des tables de
référence.
Pour chaque point est indiqué la molécule dont on parle, les résultats pertinents, ainsi que le $pK_a$.
Pour les mesures, le $pK_a$ correspond au $pH$ au point de demi-équivalence.
Il est également consigné l'erreur $\frac{|mesure - théorie|}{théorie}$ entre le $pK_a$ théorique et mesuré.

\begin{table}[H]
\centering
\begin{tabular}{|>{\columncolor{gray}}c|c|>{\columncolor{lightgray}}c|c|>{\columncolor{lightgray}}c|c|>{\columncolor{lightgray}}c|}
\hline
\rowcolor{gray} Point & Molécule & $V_{NaOH} \ [mL]$ & $pH$ & $pK_a$ & $pK_a$ théorique & Erreur $[\%]$ \\
\hline
Équivalence 1 & \chemform{H_3PO_4} & 7.22 & 4.87 & \cellcolor{black!70} & 2.16 & \cellcolor{black!70} \\
\hline
Demi-équivalence 1 & \chemform{H_3PO_4} & 3.61 & 2.87 & 2.87 & 2.16 & 32.85 \\
\hline
Équivalence 2 & \chemform{H_2PO_4^-} & 15.79 & 8.33 & \cellcolor{black!70} & 7.21 & \cellcolor{black!70} \\
\hline
Demi-équivalence 2 & \chemform{H_2PO_4^-} & 7.9 & 5.76 & 5.76 & 7.21 & 20.09 \\
\hline
3e point (non-mesuré) & \chemform{HPO_4^{2-}} & \cellcolor{black!70} & \cellcolor{black!70} & \cellcolor{black!70} & 12.32 & \cellcolor{black!70} \\
\hline
\end{tabular}
\caption{Résumé des résultats, calcul d'erreur sur le $pK_a$}\label{tab:resum-ph-pka}
\end{table}

Finalement, on peut estimer de ces résultats la concentration d'acide phosphorique dans le Coca-Cola.
À noter qu'il est calculé selon les valeurs officielles à $0.006 \ [M]$ au début de la section.
Pour le faire via nos mesures, on va utiliser la formule du $pH$ d'un acide faible, en considérant le
triacide phosphorique comme un monoacide, car les autres $K_a$ sont assez éloignés pour ne pas devoir être pris en
compte ($K_{a_1} = 6.92 \cdot 10^{-3} \gg K_{a_2} = 6.17 \cdot 10^{-8} \gg K_{a_3} = 4.79 \cdot 10^{-13} $).\\
On a donc $pH = \frac{1}{2} (pK_a - \log(C_{HA})) \Leftrightarrow C_{HA} = 10^{pK_a - 2 \cdot pH}$.
D'où via la $pK_a$, et le $pH$ mesuré avant de mettre le produit titrant (voir point 0, table~\ref{tab:ph}) :
\[ C_{H_3PO_4} \approx 10^{2.87 - 2 \cdot 2.42} \approx 0.012 \ [M] \]
En faisant la même chose pour la valeur théorique en utilisant à la place le $pK_a$ des tables de référence, on obtient :
\[ C_{H_3PO_4} \approx 10^{2.16 - 2 \cdot 2.42} \approx 0.002 \ [M] \]
Ce qui nous donne par rapport au premier résultat une concentration $2 \times$ plus élevée, respectivement
$3 \times$ plus basse.

\section{Discussion}\label{sec:discussion}

Cette expérience permet d'encadrer correctement la concentration d'acide phosphorique dans le Coca-Cola.
En effet, en mesurant son $pH$, on a pu obtenir une première approximation de sa concentration en utilisant le $pK_a$
de l'acide phosphorique issu des tables de référence.
Dans un deuxième temps, la courbe de titration nous a permis de trouver les points d'équivalence et de demi-équivalence,
puis de calculer le $pK_a$ d'acide dans notre solution de Coca-Cola, en considérant que cela n'était que du \chemform{H_3PO_4}.
En comparant les deux résultats à celui issu du site officiel de Coca-Cola, on peut voir que la concentration
d'acide phosphorique dans le Coca-Cola est bien de l'ordre de $0.006 \ [M]$.
Le premier nous en indique $0.002 \ [M]$ et le second $0.012 \ [M]$ et la concentration officielle se situe bien
dans cet ordre de grandeur.\\ \\
Ces différences relativement grandes proportionnellement à la taille ne sont néanmoins pas trop surprenantes, et en
valeur concrète, elles ne sont pas si grandes que cela.
Tout d'abord, il faut noter les erreurs de mesure et celles dues par le dispositif expérimental.
Celles-ci sont multiples, a commencé par les produits utilisés, comme le Coca-Cola, qui a été stocké plusieurs semaines
avant l'expérience, et qui a donc pu être altéré, ou encore le produit titrant, qui a été préparé et utilisé
par d'autres personnes avant nous, et qui a donc pu être contaminé.\\ \\
Les erreurs dues au matériel sont minimes, mais il faut quand même noté que le dernier calibrage du pH-mètre date du
29 juin 2022 pour une expérience menée le 27 avril 2023.
Dans le déroulement de l'expérience, on peut noter que l'on a fait dégazer le Coca-Cola après avoir mesuré son volume,
et l'on a donc pu modifier son volume à cause de l'évaporation.
De plus, on n'a pas attendu avant de continuer l'expérience, et donc le Coca était à 43 degrés Celsius lors de la
première mesure de $pH$, ce qui peut expliquer l'écart avec le $pK_a$ théorique qui est mesuré à 25 degrés Celsius.
Pour continuer dans ce registre, le dégazage n'est pas forcément complet, et donc il peut rester du dioxyde de carbone
dans le Coca-Cola, ce qui peut modifier le $pH$ car le Coca serait plus acide (ce qui semble assez plausible au vu
de la comparaison avec la courbe théorique de la figure~\ref{fig:data-plot}).\\ \\
Finalement, le temps de mesure était assez limité, et donc le $pH$ n'était pas forcément stabilisé lors de la mesure,
ce qui peut expliquer le pH plus bas, et qui peut aussi justifier la présence de gaz encore dissous dans le Coca-Cola.
On peut aussi noter le panel de mesure trop bas pour voir la troisième équivalence.\\
De plus, il faut notifier les approximations dans le traitement des résultats, comme le fait de considérer le triacide
phosphorique comme un monoacide, ou encore l'interpolation linéaire des mesures plutôt que l'approximation par une
fonction, qui aurait permis une meilleure dérivée et donc une méthode des tangentes plus précise, mais cela n'offrait pas
un avantage suffisant pour justifier le temps supplémentaire que cela aurait pris, car ces fonctions sont paramétriques
avec comme borne les points recherchés\footnote{\href{https://www.epfl.ch/labs/gdp/wp-content/uploads/2019/09/ERC_18_Chapitre_7C.pdf}{Issu d'un exemple de titrage acide-base}}.\\

\subsection*{Réponses aux questions du protocole}
\begin{enumerate}
    \item \textit{Écrire les équations de dissociation de l'acide phosphorique.}\\
        Voir les équations de dissociation de l'acide phosphorique dans l'eau dans la section~\ref{sec:introduction}.
    \item \textit{Quelles sont les valeurs de $pH$ au point de demi-équivalence ?}\\
        Les valeurs sont visibles sur le graphique de la figure~\ref{fig:sum-pe} et la table~\ref{tab:resum-ph-pka}.
        Pour la troisième demi-équivalence, la plage de résultat de l'expérience ne permet pas de la voir.
    \item \textit{Quelles sont les valeurs de $pH$ au point d'équivalence ?
        Déterminer les volumes équivalents et les $pK_a$.}\\
        Toutes ces valeurs sont également visibles sur les graphiques et tables cités à la question précédente.
        Pour déterminer le $pK_a$, on a utilisé la propriété d'égalité entre le $pK_a$ et le $pH$ au point de
        demi-équivalence.
    \item \textit{Comparer les volumes nécessaire de base pour neutraliser le premier et le deuxième ion \chemform{H^+}
        et expliquer les éventuelles différences.}\\
        Les volumes semblent assez équivalents (voir table~\ref{tab:resum-ph-pka}), soit le volume du second point
        d'équivalence est environ le double du premier.
        Cela semble cohérent avec les courbes de références (figure~\ref{fig:data-plot} et note de la section \ref{sec:discussion}).
        Cela peut être justifié que chaque mole de \chemform{H_3PO_4} va créer en libérant son proton hydron une mole
        de \chemform{H_2PO_4^-}, qui va donc être compensée par la même quantité de base.
    \item \textit{Calculer la concentration de l'acide phosphorique dans le Coca.}\\
        Voir les différentes valeurs discutées dans la section~\ref{sec:resultats}.
\end{enumerate}

\section{Conclusion}\label{sec:conclusion}
En conclusion, on peut dire que l'expérience a été concluante.
Elle a permis de déterminer correctement la concentration d'acide phosphorique dans le Coca-Cola, ainsi que de se
familiariser avec les notions de titrage via des courbes de titrage.
Néanmoins, l'expérience pourrait être améliorée avec plus de temps à disposition, pour pouvoir faire plus de mesures
et plus précises, en laissant le $pH$ se stabiliser avant de le mesurer, et en essayant de mieux dégazer le Coca-Cola.
Cela nous aurait également permis avec plus de solution titrante d'obtenir le troisième point d'équivalence.
De plus, il aurait été intéressant de connaitre la fonction générale de manière continue afin d'établir une courbe de
tendance et donc de trouver les points d'équivalence plus précisément.
Il faut conclure en disant que malgré les erreurs (voir table~\ref{tab:resum-ph-pka}), les résultats sont assez
satisfaisants, car les valeurs sont à une échelle très petite et il est donc difficile de les mesurer avec précision,
et donc avoir une telle précision avec le matériel à disposition est déjà très appréciable.

\end{document}