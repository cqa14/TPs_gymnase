\begin{figure}[H]
    \centering
    \psscalebox{1 1} % Change this value to rescale the drawing.
    {
        \begin{pspicture}(0,-12.514695)(15.79,0.53591245)
            \definecolor{colour7}{rgb}{0.8,0.8,0.8}
            \definecolor{colour11}{rgb}{0.9019608,0.9019608,0.9019608}
            \definecolor{colour0}{rgb}{0.4,0.4,0.4}
            \definecolor{colour1}{rgb}{0.8,1.0,1.0}
            \definecolor{colour2}{rgb}{0.5019608,0.5019608,0.5019608}
            \definecolor{colour3}{rgb}{0.2,0.2,0.2}
            \definecolor{colour4}{rgb}{1.0,0.4,0.4}
            \definecolor{colour5}{rgb}{0.7019608,0.7019608,0.7019608}
            \definecolor{colour6}{rgb}{0.6,0.3019608,0.0}
            \definecolor{colour8}{rgb}{0.5019608,0.6,1.0}
            \definecolor{colour9}{rgb}{0.8,0.9019608,1.0}
            \definecolor{colour10}{rgb}{0.0,0.101960786,0.5019608}
            \psrotate(8.06, -6.704696){-80.12114}{\psframe[linecolor=black, linewidth=0.04, fillstyle=solid,fillcolor=black, dimen=outer](10.96,-6.3546963)(5.16,-7.054696)}
            \psframe[linecolor=colour7, linewidth=0.04, fillstyle=solid,fillcolor=colour11, dimen=outer](8.52,-4.694696)(3.98,-5.554696)
            \psframe[linecolor=colour7, linewidth=0.04, fillstyle=solid,fillcolor=colour11, dimen=outer](5.18,-3.594696)(4.24,-10.054696)
            \psbezier[linecolor=black, linewidth=0.16](3.48,-9.574696)(1.08,-4.254696)(4.56,-0.65469605)(7.6,-3.974696044921875)
            \pspolygon[linecolor=black, linewidth=0.04, fillstyle=solid](8.88,0.505304)(9.55,0.505304)(9.55,-3.9946961)(9.306364,-7.234696)(9.154091,-7.234696)(8.88,-3.9946961)
            \pspolygon[linecolor=black, linewidth=0.04, fillstyle=solid,fillcolor=colour1](9.04,-5.694696)(9.46,-5.694696)(9.56,-4.134696)(9.54,-0.15469605)(8.9,-0.15469605)(8.86,-3.934696)
            \pspolygon[linecolor=colour0, linewidth=0.04, fillstyle=solid,fillcolor=colour0](8.76,-5.6944256)(8.78,-6.108615)(9.68,-6.131588)(9.7,-6.574696)(9.88,-6.576858)(9.86,-5.144696)(9.72,-5.1521287)(9.68,-5.703345)
            \rput(9.38,-5.174696){\psaxes[linecolor=black, linewidth=0.04, tickstyle=full, axesstyle=axes, labels=none, ticks=y, ticksize=0.1432cm, dx=1.0cm, dy=1.0cm, Dy=10.0](0,0)(0,0)(0,5)}
            \rput[bl](9.02,-0.27469605){0}
            \rput[bl](8.96,-5.174696){50}
            \psline[linecolor=black, linewidth=0.04, fillstyle=solid, arrowsize=0.05291667cm 2.0,arrowlength=1.4,arrowinset=0.0]{->}(8.1,-1.6946961)(9.06,-2.594696)
            \rput[bl](6.02,-1.654696){NaOH $0.020$ $[M]$}
            \psframe[linecolor=colour2, linewidth=0.04, fillstyle=solid,fillcolor=colour2, dimen=outer](10.91,-10.994696)(6.59,-11.374696)
            \psframe[linecolor=colour3, linewidth=0.04, fillstyle=solid,fillcolor=colour3, dimen=outer](11.1,-11.294696)(6.32,-12.514696)
            \psframe[linecolor=colour0, linewidth=0.04, fillstyle=solid,fillcolor=colour0, dimen=outer](9.42,-11.494696)(8.86,-11.774696)
            \pscircle[linecolor=colour4, linewidth=0.04, fillstyle=solid,fillcolor=colour4, dimen=outer](9.24,-11.6146965){0.06}
            \pscircle[linecolor=colour5, linewidth=0.04, fillstyle=solid,fillcolor=colour5, dimen=outer](10.38,-11.914696){0.38}
            \psline[linecolor=black, linewidth=0.04, fillstyle=solid, arrowsize=0.05291667cm 2.0,arrowlength=1.4,arrowinset=0.0]{->}(12.2,-11.634696)(11.12,-11.694696)
            \rput[bl](12.38,-11.674696){Agitateur magnétique}
            \pscircle[linecolor=colour5, linewidth=0.04, fillstyle=solid,fillcolor=colour5, dimen=outer](7.92,-11.954696){0.38}
            \psframe[linecolor=colour0, linewidth=0.04, fillstyle=solid,fillcolor=colour0, dimen=outer](7.32,-11.534696)(6.76,-11.814696)
            \pscircle[linecolor=colour4, linewidth=0.04, fillstyle=solid,fillcolor=colour4, dimen=outer](7.16,-11.674696){0.06}
            \rput[bl](12.4,-12.014696){et plaque chauffante}
            \psframe[linecolor=black, linewidth=0.04, dimen=outer](9.86,-7.274696)(7.48,-10.974696)
            \psframe[linecolor=black, linewidth=0.04, fillstyle=solid,fillcolor=colour6, dimen=outer](9.88,-9.354696)(7.5,-10.974696)
            \psellipse[linecolor=black, linewidth=0.04, fillstyle=gradient, gradlines=2000, gradbegin=colour0, gradend=colour5, dimen=outer](8.62,-10.834696)(0.56,0.12)
            \psframe[linecolor=black, linewidth=0.04, fillstyle=solid,fillcolor=colour7, dimen=outer](5.54,-9.254696)(0.0,-12.494696)
            \psframe[linecolor=black, linewidth=0.04, fillstyle=gradient, gradlines=2000, gradbegin=colour8, gradend=colour9, dimen=outer](3.88,-9.794696)(2.0,-10.874696)
            \psframe[linecolor=black, linewidth=0.04, fillstyle=solid,fillcolor=colour10, dimen=outer](2.66,-10.934696)(2.04,-11.574696)
            \psframe[linecolor=black, linewidth=0.04, fillstyle=solid,fillcolor=colour10, dimen=outer](3.84,-10.934696)(3.2,-11.574696)
            \psframe[linecolor=black, linewidth=0.04, fillstyle=solid,fillcolor=colour10, dimen=outer](2.66,-11.694696)(2.04,-12.334696)
            \psframe[linecolor=black, linewidth=0.04, fillstyle=solid,fillcolor=colour10, dimen=outer](3.84,-11.694696)(3.2,-12.334696)
            \psline[linecolor=black, linewidth=0.04, arrowsize=0.05291667cm 2.0,arrowlength=1.4,arrowinset=0.0]{->}(11.44,-2.854696)(9.68,-4.174696)
            \psline[linecolor=black, linewidth=0.04, arrowsize=0.05291667cm 2.0,arrowlength=1.4,arrowinset=0.0]{->}(1.34,-6.974696)(2.06,-9.074696)
            \psline[linecolor=black, linewidth=0.04, arrowsize=0.05291667cm 2.0,arrowlength=1.4,arrowinset=0.0]{->}(11.76,-5.934696)(9.98,-7.894696)
            \psline[linecolor=black, linewidth=0.04, arrowsize=0.05291667cm 2.0,arrowlength=1.4,arrowinset=0.0]{->}(10.78,-8.134696)(9.22,-9.914696)
            \psline[linecolor=black, linewidth=0.04, arrowsize=0.05291667cm 2.0,arrowlength=1.4,arrowinset=0.0]{->}(10.52,-10.634696)(9.22,-10.834696)
            \rput[bl](11.48,-2.914696){Burette}
            \rput[bl](10.78,-8.274696){Coca-Cola degazé}
            \rput[bl](11.78,-6.014696){Bécher}
            \rput[bl](0.52,-6.934696){pH-mètre}
            \rput[bl](10.5,-10.754696){Barre d'agitation}
        \end{pspicture}
    }
    \caption{Schéma de l'expérience}\label{fig:schema}
\end{figure}