\section{Conclusion}\label{sec:conclusion}
L'expérience s'avère concluante : une valeur plausible comme coefficient de viscosité à été trouvée,
puis les simulations sont cohérentes quelles que soient les méthodes.
Néanmoins, il faut noter une incertitude non négligeable sur les mesures : les marques sur le tube,
ainsi que la mesure de leur distance n'est pas très précise ;
l'observateur peut être perturbé par l'image de la bille dans la glycérine qui se voit
via les effets optiques de diffraction par exemple (et autres effets de perspectives) ;
ainsi que les incertitudes négligeables tels que ceux lié au modèle utilisé.
L'expérience peut être exploitée différement et de manière peut-être plus précise via une vidéo de
la chute de la bille puis une analyse sur un logiciel adapté.
Pour conclure, nous sommes satisfaits des résultats obtenus et des exploitations possibles de
l'expérience via par exemple les simulations de vitesse effectuées.